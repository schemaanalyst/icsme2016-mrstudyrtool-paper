\begin{abstract}

  % GMK NOTE: I cut these phrases (some after modifying them) to streamline the existing abstract

  % As an alternative to integrating reduction approaches into existing (and complex) mutation testing systems and then
  % subsequently evaluating each one,

  % to evaluate an approach on the data collected from a single execution of the entire
  %   mutant set.

  % The retrospective study performed by the \mr~tool provides a framework for comparing the effectiveness of
  % the existing mutant reduction techniques.

  Mutation testing is a well-known method for measuring a test suite's quality. However, due to its computational
  expense and intrinsic difficulties (e.g., detecting equivalent mutants and checking the mutant's status for each
  test), mutation testing is often challenging to use in practice. To control the computational expense of mutation
  testing, many reduction strategies have been proposed (e.g., uniform random sampling over either mutants or
  operators).  However, a stand-alone, open-source tool to compare the efficiency and effectiveness of these methods is
  not yet available. Since existing mutation testing tools are often complex and language-dependent, this paper proposes
  a tool, called \mr, that enables the ``retrospective'' study of many mutant reduction methods using the data collected
  from the analysis of all potential mutants. Focusing on the mutation operators and the mutants that they produce, the
  presented tool allows developers to quickly prototype and study mutant reduction techniques without being burdened by
  implementation details.

\end{abstract}
