\begin{abstract}

  Mutation testing is a well-known method for measuring a test suite's quality.  However, due to its computational
  expense and intrinsic difficulties (e.g., detecting equivalent mutants and checking the original program's output
  with each test), mutation testing is often challenging to utilise in practice. To control the computational
  expense of mutation testing, various reduction strategies have been proposed (e.g., uniform random sampling and
  uniform random sampling over operators).  Although prior work has established reduction techniques for mutation
  testing, a tool to retrospectively analyse the performance of these techniques is not available.  Instead of
  integrating every reduction approach into an existing mutation testing system and individually executing each one,
  this paper proposes a tool, \mr, to evaluate an approach on the data collected from a single execution of the entire
  mutant set.  The retrospective study performed by the \mr~tool provides a framework for comparing the effectiveness
  of the existing mutant reduction techniques.

\end{abstract}
