\begin{abstract}

  % GMK NOTE: I cut these phrases (some after modifying them) to streamline the existing abstract

  % As an alternative to integrating reduction approaches into existing (and complex) mutation testing systems and then
  % subsequently evaluating each one,

  % to evaluate an approach on the data collected from a single execution of the entire
  %   mutant set.

  % The retrospective study performed by the \mr~tool provides a framework for comparing the effectiveness of
  % the existing mutant reduction techniques.

  Mutation testing is a well-known method for measuring a test suite's quality.  However, due to its computational
  expense and intrinsic difficulties (e.g., detecting equivalent mutants and potentially checking a mutant's status for
  each test), mutation testing is often challenging to practically use. To control the computational expense of mutation
  testing, many reduction strategies have been proposed (e.g., uniform random sampling over mutants).  Yet, a
  stand-alone tool to compare the efficiency and effectiveness of these methods is heretofore unavailable. Since
  existing mutation testing tools are often complex and language-dependent, this paper presents a tool, called \mr, that
  enables the ``retrospective'' study of mutant reduction methods using the data collected from a prior analysis of all
  mutants. Focusing on the mutation operators and the mutants that they produce, the presented tool allows developers to
  prototype and study mutant reduction methods without being burdened by the details of mutation testing tools. In
  addition to describing the \mr's design and overviewing the experimental results from using it, this paper inaugurates
  the public release of this open-source tool.

\end{abstract}

\vspace*{-.075in}

