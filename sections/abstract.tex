\begin{abstract}
    Mutation testing is a well-known method for measuring a test suite's quality.
    However, due to its computational expense and intrinsic difficulties (e.g.,
    detecting equivalent mutants, checking the original program's output with each test case),
    mutation testing is often infeasible in practice. To control the demands of
    mutation testing, various reduction strategies have been proposed
    (e.g., Mutant Sampling, Selective Mutation).
    Although reduction techniques for mutation testing have been described
    in written form, a tool for retrospectively analysing the performance of
    the approaches is not available.
    Instead of integrating every reduction approach into an existing mutation testing system and
    individually executing each one, we evaluate an approach on the data
    collected from a single execution of the entire mutant set.
    We provide a tool, \mr, which enables this retrospective
    comparison of effectiveness for existing mutation testing reduction techniques.
\end{abstract}
