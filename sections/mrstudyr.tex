%!TEX root=../icsme2016-mrstudyr.tex

\section{\textit{mrstudyr}: Mutant Reduction Studier}

\newcommand{\mx}[1]{\mathbf{\bm{#1}}} % Matrix command
\newcommand{\vc}[1]{\mathbf{\bm{#1}}} % Vector command

% Define the layers to draw the diagram
\pgfdeclarelayer{background}
\pgfdeclarelayer{foreground}
\pgfsetlayers{background,main,foreground}

% Define block styles used later

\tikzstyle{sensor}=[draw, fill=black!10, text width=5em,
    text centered, minimum height=2.5em,drop shadow]
\tikzstyle{small}=[draw, fill=black!10, text width=2em,
    text centered, minimum width=2em, minimum height=2.5em,drop shadow]
\tikzstyle{smallcir}=[draw, ellipse, fill=black!10, text width=5em,
    text centered, minimum height=2.5em, drop shadow]
\tikzstyle{circle}=[draw, ellipse, fill=black!10, text width=8em,
    text centered, minimum height=2.5em, drop shadow]
\tikzstyle{func}=[draw, fill=black!20, text width=5em,
    text centered, minimum height=2.5em,drop shadow]
\tikzstyle{mr}=[draw, fill=black!20, text width=5em,
    text centered, minimum height=13em, minimum width = 15em, drop shadow]
\tikzstyle{ann} = [above, text width=5em, text centered]
\tikzstyle{wa} = [sensor, text width=10em, fill=black!30,
    minimum height=6em, rounded corners, drop shadow]
\tikzstyle{sc} = [sensor, text width=13em, fill=red!30,
    minimum height=10em, rounded corners, drop shadow]


% Define distances for bordering
\def\blockdist{1.5}
\def\edgedist{2.5}

\begin{figure}[t]
  \centering
% SIMPLE
\begin{tikzpicture}[thick,scale=0.85, every node/.style={scale=0.85}]
    \node [mr] at (0, 0) (mr) {};
    \node [smallcir] at (0, 3.25) (md) {Mutant Data};
    \node [sensor] at (0, 1.5) (r) {Reduction Approach};
    \node [sensor] at (0, 0) (rmd) {Reduced Mutant Data};
    \node [small] at (-1.75, 0) (m1) {$ms'$};
    \node [small] at (1.75, 0) (m2) {$ms$};
    \node [func] at (0, -1.5) (calc) {Calculations};
    \node [circle] at (0, -3.25) (rec) {Policy Recommendation};

    \path [draw, ->] (md.south) -- node [above] {}
        (r.90);
    \path [draw, ->] (md.east) -| node [above] {}
        (m2.90);
    \path [draw, ->] (r.south) -- node [above] {}
        (rmd.90);
    \path [draw, ->] (rmd.west) -- node [above] {}
        (m1.0);
    \path [draw, ->] (m1.south) |- node [above] {}
        (calc.180);
    \path [draw, ->] (m2.south) |- node [above] {}
        (calc.0);
    \path [draw, ->] (calc.south) -- node [above] {}
        (rec.90);

% VERTICAL
% \begin{tikzpicture}[thick,scale=0.85, every node/.style={scale=0.85}]
%     \node [mr] at (0, 0) (mr) {};
%     % \node [] at (-3, -0.25) (left) {};
%     % \node [] at (0, 2.3) (text) {\textit{mrstudyr}};
%     \node [sensor] at (-1.25, 0) (g) {};
%     \node [sensor] at (1.25, 0) (gg) {};
%     \node [circle] at (0, 4.5) (md) {Mutant Data};
%     \node [circle] at (0, 2.75) (df) {Dataframe};
%     \node [] at (-1.25, 1.75) (a1) {$a_1$};
%     \node [] at (0, 1.75) (dots) {...};
%     \node [] at (1.25, 1.75) (a2) {$a_n$};
%     \node [] at (-1.75, 1) (t1) {$t_1$};
%     \node [] at (-1.25, 1) (dots2) {...};
%     \node [] at (-0.75, 1) (t30) {$t_{30}$};
%     \node [] at (0.75, 1) (tt1) {$t_1$};
%     \node [] at (1.25, 1) (dots3) {...};
%     \node [] at (1.75, 1) (tt30) {$t_{30}$};
%     \node [] at (-1.75, 0) (m1) {$ms_{1}'$};
%     \node [] at (-0.75, 0) (m2) {$ms_{30}'$};
%     \node [] at (0.75, 0) (mm1) {$ms_{1}'$};
%     \node [] at (1.75, 0) (mm2) {$ms_{30}'$};
%     \node [func] at (-1.75, -1.75) (corr) {Calculate Correlation};
%     \node [func] at (1.75, -1.75) (err) {Calculate Error};
%     \node [circle] at (0, -2.8) (df2) {Dataframe};
%
%     \path [draw, ->] (md.south) -- node [above] {}
%         (df.90);
%     \path [draw, ->] (df.south) -- node [above] {}
%         (a1.10);
%     \path [draw, ->] (df.south) -- node [above] {}
%         (a2.170);
%     \path [draw, ->] (a1.south) -- node [above] {}
%         (t1.50);
%     \path [draw, ->] (a1.south) -- node [above] {}
%         (t30.130);
%     \path [draw, ->] (a2.south) -- node [above] {}
%         (tt1.50);
%     \path [draw, ->] (a2.south) -- node [above] {}
%         (tt30.130);
%     \path [draw, ->] (t1.south) -- node [above] {}
%         (m1.90);
%     \path [draw, ->] (t30.south) -- node [above] {}
%         (m2.90);
%     \path [draw, ->] (tt1.south) -- node [above] {}
%         (mm1.90);
%     \path [draw, ->] (tt30.south) -- node [above] {}
%         (mm2.90);
%     \path [draw, ->] (g.south) -| +(0, -0.25) -| node [above] {}
%         (err.90);
%     \path [draw, ->] (gg.south) -| +(0, -0.25) -| node [above] {}
%         (corr.90);
%
\end{tikzpicture}
\caption{\label{fig:mrstudyr}The inputs and outputs of mutation analysis phase
of the mutation analysis process, performed by the \mr \,tool. \\
\textit{The darkest square in the background represents the \mr \,tool.
A rectangle represents a process. A rectangle with rounded edges represents a
calculation performed by \mr. An ellipse represents the output from a process.}}
\end{figure}


% Performing mutation testing, displayed in Figure~\ref{fig:process}, is often considered to be too expensive to take

% Review the process of mutation testing, starting with the first part at the top of the diagram

When performed with tools such as Major~\cite{Just2011b}, the process of mutation testing, as displayed in
Figure~\ref{fig:process}, involves the use of operators to generate mutants for a specific program and then the
execution of tests to determine how they kill the mutants. The outcome of this phase is the higher-is-better mutation
score, or the ratio of the number of killed mutants to the number of mutants generated~\cite{Just2011a}. In many cases,
as shown in Figure~\ref{fig:process}, it is necessary to perform various analyses of the mutants. For instance, testers
may want to see which mutants were not killed so as to determine if they are equivalent (i.e., semantically the same as
the original program) or, alternatively, prompts for ways to improve the test suite.

% Start to discuss the idea of retrospective mutant analysis for reduction techniques (questions you can answer)

This paper presents another type of mutant analysis: the retrospective study of mutant reduction techniques. Leveraging
data collected during mutation testing (e.g., the name of an operator that produced a mutant, the kill-status of a
mutant, and the costs of producing and analysing a mutant), this method supports the study of mutant reduction
techniques. This type of analysis allows testers to ask and answer questions like ``what would the mutation score be if
only a random 20\% of the mutants were executed?'' While questions of this nature could be executed through, for
instance, either a manual analysis or a bespoke program, this paper presents \mr, a tool that makes it easy to
effectively pose and answer questions about methods for mutant reduction.

% The first of its kind, \mr~is a tool for performing the retrospective analysis of mutant reduction techniques. Upon
% collecting generalised data from performing mutation testing on all mutants, \mr~is capable of determining efficiency
% and effectiveness of well-known mutant reduction techniques specific to the data.

By accepting a generalised input format, the \mr~tool can be utilised in a variety of domains to assess the efficiency
and effectiveness of mutant reduction techniques.  In this paper, we display that \mr~is versatile by conducting a
preliminary study of mutant reduction techniques on mutant data collected from testing database schemas.  However, the
\mr~tool is not limited to analysing mutant data from programs or databases, it can effectively be used in any number
domains. Where mutation testing can be performed, \mr~is capable of analysing reduction techniques.

% vim: ft=tex
%!TEX root=../icsme2016-mrstudyr.tex

\begin{table}[t!]
  \vspace{-0.75em}
  \caption{Sample data analysed in the empirical study.} \label{tbl:study-data}
  \footnotesize
  \centering
  \scalebox{\tablescalefactor}{
    \begin{tabular}{l@{\hskip 4pt}l@{\hskip 4pt}l@{\hskip 4pt}l@{\hskip 4pt}l@{\hskip 4pt}l@{\hskip 6pt}l@{\hskip 0pt}}
      {Identifier} & {DBMS} & {Schema} & {Operator} & {Type} & {Kill} & {Time} \\
      \toprule

      fpbpyn2cnbiinwfyy2mi & SQLite & CoffeeOrders & FKCColumnPairR & NORMAL & true & 59 \\
      fpbpyn2cnbiinwfyy2mi & SQLite & CoffeeOrders & FKCColumnPairR & NORMAL & true & 59 \\
      fpbpyn2cnbiinwfyy2mi & SQLite & CoffeeOrders & FKCColumnPairR & NORMAL & true & 88 \\
      fpbpyn2cnbiinwfyy2mi & SQLite & CoffeeOrders & FKCColumnPairR & NORMAL & true & 54 \\
      fpbpyn2cnbiinwfyy2mi & SQLite & CoffeeOrders & FKCColumnPairE & NORMAL & true & 56 \\
      fpbpyn2cnbiinwfyy2mi & SQLite & CoffeeOrders & FKCColumnPairE & NORMAL & true & 49 \\

      \bottomrule
    \end{tabular}
  }
  \vspace{-1em}
\end{table}

\input{tables/schemas}

The \mr~tool is not limited to the domains where it can be utilized, nor is it restricted
by the reduction approaches it can perform. In the current state, \mr~is able to perform
the most common mutant reduction technique, mutant sampling~\cite{gopinath2015mutation, jia2011analysis, gopinath2015empirical}.
The \mr~tool can be extended to perform specific reduction techniques such as E-Selective
~\cite{offutt1996experimental} --- where only mutants created by the following Mothra operators
are chosen: ABS, UOI, LCR, AOR and ROR~\cite{gopinath2015empirical} or complex reduction techniques
such as mutant clustering, where mutants are chosen based on clustering algorithms~\cite{jia2011analysis}.

In addition to being extensible in both the domains in which it can be utilized and the
reduction approaches it can perform, \mr~is accompanied with extensive documentation~\cite{tool}.
The documentation includes the necessary commands to get the reader ``up-and-running''
with \mr~quickly and obviates the need to read a poorly-formatted command-line manual page.
The various accepted function calls of the \mr~tool are also provided in the documentation
accompanied by their respective output. Finally, provided in the documentation is a live
asciinema~\cite{asciinema} screencast, displaying usage of the tool.

The \mr~tool has been released on the popular Git repository hosting service, GitHub~\cite{github}
under version 3 of the GNU General Public License~\cite{license}. The GNU General Public License
allows for commercial, patent and private use, as well as distribution of and modification to the tool
under the condition that anything derived from it is bound by the same conditions as the originally
released version~\cite{copyleft}.

% \subsection{Retrospective Analysis with \mr}

The next phase in the mutation analysis process is retrospective analysis. In this
phase, mutants statuses are determine. Where a mutant is considered
``dead'' if when executing the modified version of the source
code the output differed from the original version. Conversely, a mutant remains alive
if the output from the modified version of the source code does not differ from the
original version. More importantly, in this phase of the mutation analysis process,
the efficiency and effectiveness of various reduction techniques are evaluated.

To avoid the cost of performing mutation multiple times, the \mr tool retrospectively studies
each reduction technique's efficiency and effectiveness from the data collected from a prior analysis of all mutants.
Displayed in Figure \ref{fig:mrstudyr},
the \mr tool uses the data collected from mutation testing to conduct further analyses
regarding the effectiveness of a reduction technique.

% \subsection{Inputs and Output of \mr}

The data, at the very least, must  include the status of
a mutant after testing (e.g., dead or alive), the type of mutant (e.g., normal, duplicate, equivalent, or stillborn)
and the mutation operator used for each mutant. In addition to the minimum requirements, the data can include
more detail about each mutant. For example additional data that may be collected when performing mutation testing
on a database are the database
management system or schema under test and the cost of generating each mutant.

As indicated previously, the status of a mutant after testing must be provided
in the input data to the \mr tool.
Based on the number of dead and alive mutants after mutation testing,
the \mr tool calculates and associates a mutation score with a set of mutants.
The mutation score associated with a mutant set is a value representative of how well
the test suite is at identifying and killing the mutants included in that set.
The \mr tool first calculates the
mutation score for the non-reduced set of mutants and then later for each reduced set. The mutation
score for the non-reduced mutant set is used as a baseline for determining
each reduced set's performance.

% \subsection{Calculating Effectiveness with \mr}

\mr performs each reduction technique returning reduced mutant data,
displayed in Figure~\ref{fig:mrstudyr}. The effectiveness of a reduction technique is then evaluated
based on five metrics: mutation score, correlation, the creation-cost reduction percentage
and the mean absolute and root mean squared errors.

These five evaluation metrics help determine how effective a reduced set is at
representing all mutants. \mr calculates mutation score
by dividing the number of killed mutants by the total number of normal---not
equivalent, duplicate or stillborn---mutants. Then, provided by the ``Kendal'' R package,
\mr calculates Kendall's \taub~coefficient---because of its tie-awareness---to
determine the correlation between mutation scores of all mutants and each reduced set~\cite{mcleod2005kendall}.

Kendall's \taub~is a measurement of correlation between -1 and 1, representing
a strong negative or positive correlation, respectively. A measurement
of 0 means that there is no correlation. Additionally, we adopt the
Guildford scale to describe correlation measurements, with the absolute value
of a coefficient being described as ``low'', ``moderate'', ``high'' or ``very high'' when
it is less than 0.4, between 0.4 and 0.7, ranging from 0.7 to 0.9 or 0.9 and above, respectively \cite{inozemtseva2014coverage}.

% Is there a better way to re-word calculating a percentage than provided here?
Next, a percentage representing the reduction magnitude in creation cost for all mutants
and each reduced set is calculated. \mr calculates this by subtracting
the cost of creating the reduced set from the cost of creating all mutants, then
divides that by the cost of creating all mutants. This helps
determine whether performing a reduction technique is cost-effective.

Finally, \mr calculates both mean absolute (MAE) and root mean
squared errors (RMSE). MAE gives an equal weight to all errors allowing for
a high-level understanding, while RMSE emphasises large deviations in mutation scores.
These errors exploit the deviation between the mutation score of all mutants and
the reduced set.

