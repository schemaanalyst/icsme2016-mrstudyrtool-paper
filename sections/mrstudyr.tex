%!TEX root=../icsme2016-mrstudyr.tex

\section{\textit{mrstudyr}: Mutant Reduction Studier}

%!TEX root=../icsme2016-mrstudyr.tex

\newcommand{\mx}[1]{\mathbf{\bm{#1}}} % Matrix command
\newcommand{\vc}[1]{\mathbf{\bm{#1}}} % Vector command

% Define the layers to draw the diagram
\pgfdeclarelayer{background}
\pgfdeclarelayer{foreground}
\pgfsetlayers{background,main,foreground}

% Define block styles used later

\tikzstyle{sensor}=[draw, fill=black!5, text width=15em,
text centered, minimum height=2.5em,drop shadow]
\tikzstyle{smallsensor}=[draw, fill=black!5, text width=4em,
text centered, minimum height=2.5em,drop shadow]
\tikzstyle{box}=[draw, text width=8em,
text centered, minimum width=17.5em, minimum height=3.5em]
\tikzstyle{calc}=[draw, fill=black!5, text width=2.5em,
text centered, rounded corners, minimum height=2.5em,drop shadow]
\tikzstyle{circle}=[draw, ellipse, fill=black!5, text width=11em,
text centered, minimum height=2.5em, drop shadow]
\tikzstyle{mr}=[draw, fill=black!20, text width=5em,
text centered, minimum height=17em, minimum width = 18em, drop shadow]

% Define distances for bordering
\def\blockdist{1.5}
\def\edgedist{2.5}

\begin{figure}[t]

  \vspace{-.75em}

  \centering
  % SIMPLE
  \begin{tikzpicture}[thick,scale=0.85, every node/.style={scale=0.85}]
    \node [mr] at (0, 0) (mr) {};
    \node [circle] at (0, 3.7) (md) {Original Data};
    \node [sensor] at (0, 2.25) (r) {Reduction Techniques};
    \node [circle] at (0, 0.75) (rmd) {Cumulated Reduced Data};
    \node [sensor] at (0, -0.75) (ra) {Efficiency \& Effectiveness Analysis};
    \node [box] at (0, -2.25) (box) {};
    \node [calc] at (-2.25, -2.25) (ms) {MS};
    \node [calc] at (-0.75, -2.25) (err) {Red.};
    \node [calc] at (0.75, -2.25) (corr) {Corr.};
    \node [calc] at (2.25, -2.25) (err) {Err.};
    \node [sensor] at (0, -3.75) (he) {Human Examination};
    \node [circle] at (0, -5.25) (pr) {Policy Recommendation};

    \path [draw, ->] (md.south) -- node [above] {}
      (r.90);
    \path [draw, ->] (r.south) -- node [above] {}
      (rmd.90);
    \path [draw, ->] (rmd.south) -- node [above] {}
      (ra.90);
    \path [draw, ->] (ra.south) --  node [above] {}
      (box.90);
    \path [draw, ->] (box.south) --  node [above] {}
      (he.90);
    \path [draw, ->] (he.south) -- node [above] {}
      (pr.90);
  \end{tikzpicture}

  \caption{\label{fig:mrstudyr}The inputs and outputs of the \mr~tool.}

  \captionpara{0.5}{0.9}{0.5}{In this figure, the dark square represents the \mr~tool and its constituent parts, a
    rectangle stands for a process, a rectangle with rounded edges is a calculation performed by \mr, and an ellipse
  symbolises a process output.}

  \vspace{-1.8em}
\end{figure}



% Further explain the idea of retrospective analysis

Performing mutation testing, displayed in Figure~\ref{fig:process}, is often considered to be too expensive to take


The first of its kind, \mr~is a tool for performing the retrospective analysis of mutant reduction techniques. Upon
collecting generalised data from performing mutation testing on all mutants, \mr~is capable of determining efficiency
and effectiveness of well-known mutant reduction techniques specific to the data.

By accepting a generalised input format, the \mr~tool can be utilized in a variety of domains to assess the efficiency
and effectiveness of mutant reduction techniques.  In this paper, we display that \mr~is versatile by conducting a
preliminary study of mutant reduction techniques on mutant data collected from testing database schemas.  However, the
\mr~tool is not limited to analysing mutant data from programs or databases, it can effectively be used in any number
domains. Where mutation testing can be performed, \mr~is capable of analysing reduction techniques.

% vim: ft=tex
%!TEX root=../icsme2016-mrstudyr.tex

\begin{table}[t!]
  \vspace{-0.75em}
  \caption{Sample data analysed in the empirical study.} \label{tbl:study-data}
  \footnotesize
  \centering
  \scalebox{\tablescalefactor}{
    \begin{tabular}{l@{\hskip 4pt}l@{\hskip 4pt}l@{\hskip 4pt}l@{\hskip 4pt}l@{\hskip 4pt}l@{\hskip 6pt}l@{\hskip 0pt}}
      {Identifier} & {DBMS} & {Schema} & {Operator} & {Type} & {Kill} & {Time} \\
      \toprule

      fpbpyn2cnbiinwfyy2mi & SQLite & CoffeeOrders & FKCColumnPairR & NORMAL & true & 59 \\
      fpbpyn2cnbiinwfyy2mi & SQLite & CoffeeOrders & FKCColumnPairR & NORMAL & true & 59 \\
      fpbpyn2cnbiinwfyy2mi & SQLite & CoffeeOrders & FKCColumnPairR & NORMAL & true & 88 \\
      fpbpyn2cnbiinwfyy2mi & SQLite & CoffeeOrders & FKCColumnPairR & NORMAL & true & 54 \\
      fpbpyn2cnbiinwfyy2mi & SQLite & CoffeeOrders & FKCColumnPairE & NORMAL & true & 56 \\
      fpbpyn2cnbiinwfyy2mi & SQLite & CoffeeOrders & FKCColumnPairE & NORMAL & true & 49 \\

      \bottomrule
    \end{tabular}
  }
  \vspace{-1em}
\end{table}

% vim: ft=tex
%!TEX root=../icsme2016-mrstudyr.tex

\begin{table}[t!]
  % \vspace{-0.75em}
    \caption{Schemas analysed in the empirical study.}\label{tbl:study-schemas}
  \vspace{-1em}
  \footnotesize
  \centering
  \scalebox{\tablescalefactor}{
    \begin{tabular}{l@{\hskip -5pt}rrrrrrrr}
      {Schema} & \rot{Tables} & \rot{Columns} & \rot{Checks} & \rot{Foreign Keys} & \rot{Not Nulls} & \rot{Primary Keys} & \rot{Uniques} & \rot{$\sum$Constraints} \\
      \toprule

      CoffeeOrders & 5 & 20 & 0 & 4 & 10 & 5 & 0 & 19 \\
      Employee & 1 & 7 & 3 & 0 & 0 & 1 & 0 & 4 \\
      Inventory & 1 & 4 & 0 & 0 & 0 & 1 & 1 & 2 \\
      Iso3166 & 1 & 3 & 0 & 0 & 2 & 1 & 0 & 3 \\
      JWhoisServer & 6 & 49 & 0 & 0 & 44 & 6 & 0 & 50 \\
      MozillaPermissions & 1 & 8 & 0 & 0 & 0 & 1 & 0 & 1 \\
      NistWeather & 2 & 9 & 5 & 1 & 5 & 2 & 0 & 13 \\
      Person & 1 & 5 & 1 & 0 & 5 & 1 & 0 & 7 \\
      Products & 3 & 9 & 4 & 2 & 5 & 3 & 0 & 14 \\
      \midrule
      Total & 21 & 114 & 13 & 7 & 71 & 21 & 1 & 113 \\

      \bottomrule
    \end{tabular}
  }
  \vspace{-1em}
\end{table}


The \mr~tool is not limited to the domains where it can be utilized, nor is it restricted
by the reduction approaches it can perform. In the current state, \mr~is able to perform
the most common mutant reduction technique, mutant sampling~\cite{gopinath2015mutation, jia2011analysis, gopinath2015empirical}.
The \mr~tool can be extended to perform specific reduction techniques such as E-Selective
~\cite{offutt1996experimental} --- where only mutants created by the following Mothra operators
are chosen: ABS, UOI, LCR, AOR and ROR~\cite{gopinath2015empirical} or complex reduction techniques
such as mutant clustering, where mutants are chosen based on clustering algorithms~\cite{jia2011analysis}.

In addition to being extensible in both the domains in which it can be utilized and the
reduction approaches it can perform, \mr~is accompanied with extensive documentation~\cite{tool}.
The documentation includes the necessary commands to get the reader ``up-and-running''
with \mr~quickly and obviates the need to read a poorly-formatted command-line manual page.
The various accepted function calls of the \mr~tool are also provided in the documentation
accompanied by their respective output. Finally, provided in the documentation is a live
asciinema~\cite{asciinema} screencast, displaying usage of the tool.

The \mr~tool has been released on the popular Git repository hosting service, GitHub~\cite{github}
under version 3 of the GNU General Public License~\cite{license}. The GNU General Public License
allows for commercial, patent and private use, as well as distribution of and modification to the tool
under the condition that anything derived from it is bound by the same conditions as the originally
released version~\cite{copyleft}.

% \subsection{Retrospective Analysis with \mr}

The next phase in the mutation analysis process is retrospective analysis. In this
phase, mutants are identified as dead or alive. Where a mutant is considered
``dead''---or to have been killed---if when executing the modified version of the source
code the output differed from the original version. Conversely, a mutant remains alive
if the output from the modified version of the source code does not differ from the
original version. More importantly, in this phase of the mutation analysis process,
we are able to evaluate the effectiveness of various reduction techniques.

By retrospectively analysing each reduction technique's effectiveness, we are
able to avoid the cost of performing mutation testing more than once. To alleviate
the additional executions, data from the mutation testing phase needs to be collected.
Displayed in Figure \ref{fig:mrstudyr},
the \mr tool uses the data collected from mutation testing to conduct further analyses
regarding the effectiveness of a reduction technique.

% \subsection{Inputs and Output of \mr}
Displayed in Figure \ref{fig:mrstudyr}, \mr uses the data collected
from mutation testing to conduct further analyses
regarding the efficiency and effectiveness of a reduction technique.

The data provided to the \mr tool must minimally include mutant statuses (e.g., dead or alive),
mutant types (e.g., normal, duplicate, equivalent, or stillborn)
and operators. Moreover, the data can include futher detail about each mutant.
For example, when testing database schemas, the additional data may consist of
the database management system, schema and generation cost of each mutant.

Based on the number of dead and alive mutants after mutation testing,
the \mr tool calculates and associates a mutation score with a set of mutants.
A test suite's mutation score represents its ability to discover faults.
\mr calculates the mutation score for the entire set of mutants to be
used as a baseline for determining each reduced sets' effectiveness.

% \subsection{Calculating Effectiveness with \mr}

\mr performs each reduction technique returning reduced mutant data,
displayed in Figure~\ref{fig:mrstudyr}. The effectiveness of a reduction technique is then evaluated
based on five metrics: mutation score, correlation, the creation-cost reduction percentage
and the mean absolute and root mean squared errors.

These five evaluation metrics help determine how effective a reduced set is at
representing all mutants. \mr calculates mutation score
by dividing the number of killed mutants by the total number of normal---not
equivalent, duplicate or stillborn---mutants. Then, provided by the ``Kendal'' R package,
\mr calculates Kendall's \taub~coefficient---because of its tie-awareness---to
determine the correlation between mutation scores of all mutants and each reduced set~\cite{mcleod2015kendall}.

Kendall's \taub~is a measurement of correlation between -1 and 1, representing
a strong negative or positive correlation, respectively. A measurement
of 0 means that there is no correlation. Additionally, we adopt the
Guildford scale to describe correlation measurements, with the absolute value
of a coefficient being described as ``low'', ``moderate'', ``high'' or ``very high'' when
it is less than 0.4, between 0.4 and 0.7, ranging from 0.7 to 0.9 or 0.9 and above, respectively \cite{inozemtseva2014coverage}.

% Is there a better way to re-word calculating a percentage than provided here?
Next, a percentage representing the reduction magnitude in creation cost for all mutants
and each reduced set is calculated. \mr calculates this by subtracting
the cost of creating the reduced set from the cost of creating all mutants, then
divides that by the cost of creating all mutants. This helps
determine whether performing a reduction technique is cost-effective.

Finally, \mr calculates both mean absolute (MAE) and root mean
squared errors (RMSE). MAE gives an equal weight to all errors allowing for
a high-level understanding, while RMSE emphasises large deviations in mutation scores.
These errors exploit the deviation between the mutation score of all mutants and
the reduced set.

