\subsection{Tool Design}

The \mr tool was designed to empirically study the effectiveness of
various reduction techniques. The two reduction techniques mentioned earlier---uniform random sampling
and uniform random sampling over operators---require an arbitrary percentage, $x$, to be chosen as the maximum threshold
for the number of mutants to be analysed.

Where Wong and Mathur in their studies \cite{mathur1994empirical}, \cite{wong1993mutation} conducted
an experiment using random sampling with $x$ from $10\%$ to $40\%$ increasing by steps of $5\%$, we
chose to start $x$ at $10\%$ and increase by intervals of $10\%$ up to $90\%$. By lowering
the granularity of the experiment to $10\%$ intervals instead of
$1\%$ or $5\%$, we are able to reduce the cost of performing the retrospective analysis while observing
similar trends.

While reducing the costs of performing the retrospective analysis were a primary concern, it does not
outweigh producing accurate results. Although we conducted our experiments with $x$ increasing by
$10\%$ intervals, we did not see a significant enough loss in accuracy and in the amount of data collected to make
it worth running the additional trials with the smaller intervals (i.e. $1\%$ or $5\%$).

When it came to determining the number of trials necessary, we followed
the recommendations of Traeger et al. and Arcuri, by running 30 trials for each configuration of a reduction approach \cite{traeger2008nine, arcuri2014hitchhiker}.
The general rule in many fields of science is to at least perform 30 observations to show with high confidence
that the obtained results are statistically significant and to allow for the results to approximate a normal
distribution.

