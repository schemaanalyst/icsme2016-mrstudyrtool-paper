\subsection{Conducting Experiment Campaigns}

% The \mr~tool accepts a generalised input format enabling it to be used to retrospectively analyse mutation testing data
% from programs and also emerging domains such as databases.  Although domains under test may differ, like data can be
% collected (e.g., mutant status, mutant type and operator). While more data can optionally be collected from mutation
% testing, the additional data is more dependent on the domain under test. Data collected from performing mutation testing
% on the nine database schemas is displayed in Table~\ref{tbl:study-data}.  This is a snippet of the data used in the
% preliminary study of effectiveness of \mr.  The additional data are unique identifier for the database management system
% / schema combination (Identifier), DBMS (DBMS), schema (Schema) and mutant creation time (Time).

The structure of the experiment function for the two reduction sub-techniques is displayed in
Figure~\ref{fig:experiment-structure}. \mr~was designed to empirically study the efficiency and effectiveness of various
reduction techniques in retrospect. The reduction technique, mutant sampling, requires an arbitrary $x$, to be chosen as
the maximum fractional threshold for the number of mutants to be analysed from a set.

Where Wong and Mathur in their studies~\cite{mathur1994empirical, wong1993mutation} conducted experiments using mutant
sampling with $x$ from $10\%$ to $40\%$ increasing by steps of $5\%$, we chose to analyse $x$ at $1\%$ and then increase
by $10\%$ intervals up to $90\%$. By lowering the granularity of the experiment to $10\%$ intervals instead of $1\%$ or
$5\%$, we reduce the cost of performing retrospective analysis with \mr, while observing similar trends.

Following the recommendations of Traeger~\etal~and Arcuri, \mr~runs 30 trials for each configuration of a reduction
approach~\cite{traeger2008nine, arcuri2014hitchhiker}. The general rule in many fields of science is to minimally
perform 30 observations to show with high confidence that the results are statistically significant and to allow for the
results to approximate a normal distribution.

\subsection{Implementation as an R Package}

The R programming language for statistical data analysis was specifically designed to be a full statistical language for
conducting empirical studies. Following its predecessor, R is a powerful means for data analysts to express
computations~\cite{ihaka1996r}.  In R, the fundamental way to share code is via a package.

%!TEX root=../icsme2016-mrstudyr.tex

\begin{figure}
\centering

\vspace{-.75em}

\scalebox{\algorithmfigurescalefactor}{
\begin{tabular}{|l|}
\hline

% TODO: Review this algorithm and fix all of the other issues pointed out in the marked version of the paper
% --> How is the variable p incorporated?

\begin{tabular}{l}
\vspace{-2ex} \\

{\bf function} analyze ($data$) \\

\atab {\bf where} \\
\atab \atab \tinybullet $\mbox{{\em data}}$ is data collected from performing mutation testing on all mutants \\
\atab \atab \tinybullet $x$ is the maximum percentage for the number of mutants \\
\atab \atab \tinybullet {\em percentages} is the list of chosen $x$ values for mutant sampling \\
\atab \atab \tinybullet $t$ is the trial index of the experiment \\
\atab \atab \tinybullet {\em trials} is the total number of trials set for each configuration of technique \\
\atab \atab \tinybullet {\em reduced} is the reduced mutant data returned from a single reduction method \\
\atab \atab \tinybullet $d$ is the data from the configuration of a reduction method and $x$ value\\

\vspace*{-.75em} \\

% \hspace{1em} \atab {\bf initialise} $d$ \smallvspace \\

\hspace{1em} \atab $d \leftarrow \emptyset$, $t \leftarrow 0$ \smallvspace \\

% \hspace{1em} \atab $t \leftarrow 0$ \smallvspace \\

\hspace{1em} \atab {\bf for each} $x \in \mbox{{\em percentages}}$ \smallvspace \\

\hspace{3em} \atab {\bf for} $t \leq \mbox{{\em trials}}$ \smallvspace \\

\hspace{4em} \atab {\bf let} $\mbox{{\em reduced}} \leftarrow \mbox{{\em reduction\,technique}}\,(\mbox{{\em data}},\,x)$ \smallvspace \\

\hspace{4em} \atab {\bf let} $d \leftarrow$ {\bf append}($d$, $\mbox{{\em reduced}}$) \smallvspace \\

\hspace{4em} \atab $t \leftarrow t + 1$ \smallvspace \\

\hspace{1em} \atab {\bf return} $d$ \smallvspace \\

{\bf end function} \\
\vspace{-0.5em} \\

\end{tabular} \\

\hline
\end{tabular}}

\caption{\label{fig:experiment-structure} An experiment function for studying mutant sampling.}

\vspace{-1em}

\end{figure}



% TODO: This paragraph will need Dr. Kapfhammer's magic to fix (I cannot stand the spacing)

R packages are easy to distribute and include code, data, documentation and tests~\cite{wickham2015r}.  Since \mr~has
been released as an R package, installing the tool requires four commands in the R console.  First,
\texttt{install.packages("devtools")}, then {\small\texttt{library(devtools)}} to install and load the
{\small\texttt{devtools}}~\cite{devtools} package, respectively.  The \texttt{devtools} package is necessary because it
provides functions for installing and maintaining \mr~as well as its dependencies.  Finally, to install \mr~\cite{tool}
from the popular Git repository hosting service, GitHub~\cite{github}, use the following command:
{\small\texttt{devtools::install\_github("mccurdyc/mrstudyr")}}.  Then, finally, load \mr~using
\texttt{library(mrstudyr)}.


