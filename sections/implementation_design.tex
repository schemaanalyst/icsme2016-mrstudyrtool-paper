\subsection{Tool Design}

The \mr tool accepts a generalised input format enabling it to be used to retrospectively analyse mutation testing
data from programs and also emerging domains, i.e., databases. Although domains under test may differ,
like data can be collected (e.g., mutant status, mutant type and operator). While more data
can optionally be collected from mutation testing, the additional data is more
dependent on the domain. Sample data collected from performing mutation testing on database schemas
is included in Table~\ref{tbl:study-data}, where the additional data are
unique identifier for the database management system / schema combination (Identifier),
DBMS (DBMS), schema (Schema) and mutant creation time (Time).

Additionally, in our experiments, we chose to analyse a select number of schemas, shown in Table~\ref{tbl:study-schemas}.
We chose these nine schemas because they range in triviality. For example, the MozillaPermissions schema contains
a single constraint, where as the JWhoisServer schema has a total of 50 constraints. This allowed us to evaluate the effectiveness
of the \mr tool for schemas of varying complexities.

\mr was designed to empirically study the efficiency and effectiveness of
various reduction techniques in retrospect. The two reduction techniques mentioned earlier---which are often considered
to be the most cost-effective---are random sampling and sampling over operators. Both
of these techniques require an arbitrary percentage, $x$, to be chosen as the maximum threshold
for the number of mutants to be analysed from a set. The structure of the experiment function for the two
reduction techniques is displayed in Figure~\ref{fig:experiment-structure}.

Where Wong and Mathur in their studies~\cite{mathur1994empirical, wong1993mutation} conducted
experiments using mutant sampling with $x$ from $10\%$ to $40\%$ increasing by steps of $5\%$, we
chose to analyse $x$ at $1\%$ and then increase by $10\%$ intervals up to $90\%$. By lowering
the granularity of the experiment to $10\%$ intervals instead of
$1\%$ or $5\%$, we reduce the cost of performing retrospective analysis with \mr, while observing
similar trends.

Following the recommendations of Traeger et al. and Arcuri, we ran 30 trials for each configuration
of a reduction approach in \mr~\cite{traeger2008nine, arcuri2014hitchhiker}.
The general rule in many fields of science is to minimally perform 30 observations to show with high confidence
that the results are statistically significant and to allow for the results to approximate a normal distribution.
