\subsection{Tool Design}

% The \mr~tool accepts a generalised input format enabling it to be used to retrospectively analyse mutation testing data
% from programs and also emerging domains such as databases.  Although domains under test may differ, like data can be
% collected (e.g., mutant status, mutant type and operator). While more data can optionally be collected from mutation
% testing, the additional data is more dependent on the domain under test. Data collected from performing mutation testing
% on the nine database schemas is displayed in Table~\ref{tbl:study-data}.  This is a snippet of the data used in the
% preliminary study of effectiveness of \mr.  The additional data are unique identifier for the database management system
% / schema combination (Identifier), DBMS (DBMS), schema (Schema) and mutant creation time (Time).

We chose these nine schemas because they range in triviality. For example, the MozillaPermissions schema contains a
single constraint, where as the JWhoisServer schema has a total of 50 constraints. This allowed us to evaluate the
policy recommendations made based on the effectiveness of reduction techniques analysed by \mr~for schemas of varying
complexities.

The structure of the experiment function for the two reduction sub-techniques is displayed in
Figure~\ref{fig:experiment-structure}. \mr~was designed to empirically study the efficiency and effectiveness of various
reduction techniques in retrospect. The reduction technique, mutant sampling, requires an arbitrary $x$, to be chosen as
the maximum fractional threshold for the number of mutants to be analysed from a set.

Where Wong and Mathur in their studies~\cite{mathur1994empirical, wong1993mutation} conducted experiments using mutant
sampling with $x$ from $10\%$ to $40\%$ increasing by steps of $5\%$, we chose to analyse $x$ at $1\%$ and then increase
by $10\%$ intervals up to $90\%$. By lowering the granularity of the experiment to $10\%$ intervals instead of $1\%$ or
$5\%$, we reduce the cost of performing retrospective analysis with \mr, while observing similar trends.

Following the recommendations of Traeger~\etal~and Arcuri, \mr~runs 30 trials for each configuration of a reduction
approach~\cite{traeger2008nine, arcuri2014hitchhiker}. The general rule in many fields of science is to minimally
perform 30 observations to show with high confidence that the results are statistically significant and to allow for the
results to approximate a normal distribution.
