\subsection{Conducting Experiment Campaigns}

% The \mr~tool accepts a generalised input format enabling it to be used to retrospectively analyse mutation testing data
% from programs and also emerging domains such as databases.  Although domains under test may differ, like data can be
% collected (e.g., mutant status, mutant type and operator). While more data can optionally be collected from mutation
% testing, the additional data is more dependent on the domain under test. Data collected from performing mutation testing
% on the nine database schemas is displayed in Table~\ref{tbl:study-data}.  This is a snippet of the data used in the
% preliminary study of effectiveness of \mr.  The additional data are unique identifier for the database management system
% / schema combination (Identifier), DBMS (DBMS), schema (Schema) and mutant creation time (Time).

% The general rule in many fields of science is to minimally perform 30 observations to show with high confidence that the
% results are statistically significant and to allow for the results to approximate a normal distribution.

% The structure of the experiment function for the two reduction sub-techniques is displayed in

% Explain how to conduct a compaign of experiments using the figure with an algorithm's psuedo code
% TODO: Fix the fact that the last sentence in this paragraph does not take up enough space

% Experiment Structure Diagram: This diagram in its current state is pretty specific to our experiment and testing database
% schemas. As you will notice, it iterates over a list of schemas. Additionally, it is specific to mutant sampling
% with the diagram showing that a technique iterates over a list of percentages. We can keep the diagram the way it
% is or remove it. I think by making it generalised (removing list of schemas, and percentages) it reduces it to the
% point where it isn't worth showing because it will then consist of iterate through a number of trials, perform a reduction
% technique, return the data collected from every trial. If it were simplified, I think it would just be easier to
% describe the technique it writing rather than taking up so much space with a diagram.

% I know in GH we talked about objective-level discussion then increase granularity. With that in mind, this first
% sentence might be too specific for this section and may need to move to a section specific to the results displayed
% in this paper. Or, going back to the previous comment, we could remove the 'iterate over schemas' from the diagram
% and say that this is the structure for the campaign of experiments for mutant sampling with mrstudyr (I will do this!)

Figure~\ref{fig:experiment-structure} gives the structure for the campaign of experiments that \mr~conducts to evaluate
the efficiency and effectiveness of mutant reduction techniques. Since the presented tool currently focuses on mutant
sampling, the algorithm requires as input an arbitrary $x$, to be chosen as the maximum fractional threshold for the
number of mutants to be analysed from a set.  Following the recommendations of Traeger~\etal~and Arcuri and Briand,
\mr~runs 30 trials for each configuration of a reduction approach~\cite{traeger2008nine, arcuri2014hitchhiker}, thereby
controlling for the randomness inherent in a reduction method's behavior and execution time.

\subsection{Implementation as an R Package}

The R programming language for statistical data analysis was specifically designed to be a full statistical language for
conducting empirical studies. Following its predecessor, R is a powerful means for data analysts to express
computations~\cite{ihaka1996r}.  In R, the fundamental way to share code is via a package.  R packages are easy to
distribute and include code, data, documentation and tests~\cite{wickham2015r}.  Since \mr~has been released as an R
package, installing the tool requires four commands in the R console.  First, \texttt{install.packages("devtools")},
then {\small\texttt{library(devtools)}} to install and load the {\small\texttt{devtools}}~\cite{devtools} package,
respectively.  The \texttt{devtools} package is necessary because it provides functions for installing and maintaining
\mr~as well as its dependencies.  Finally, to install \mr~\cite{tool} from the popular Git repository hosting service,
GitHub~\cite{github}, use the following command: {\small\texttt{devtools::install\_github("mccurdyc/mrstudyr")}}.  Then,
finally, load \mr~using \texttt{library(mrstudyr)}.

%!TEX root=../icsme2016-mrstudyr.tex

\begin{figure}
\centering

\vspace{-.75em}

\scalebox{\algorithmfigurescalefactor}{
\begin{tabular}{|l|}
\hline

% TODO: Review this algorithm and fix all of the other issues pointed out in the marked version of the paper
% --> How is the variable p incorporated?

\begin{tabular}{l}
\vspace{-2ex} \\

{\bf function} analyze ($data$) \\

\atab {\bf where} \\
\atab \atab \tinybullet $\mbox{{\em data}}$ is data collected from performing mutation testing on all mutants \\
\atab \atab \tinybullet $x$ is the maximum percentage for the number of mutants \\
\atab \atab \tinybullet {\em percentages} is the list of chosen $x$ values for mutant sampling \\
\atab \atab \tinybullet $t$ is the trial index of the experiment \\
\atab \atab \tinybullet {\em trials} is the total number of trials set for each configuration of technique \\
\atab \atab \tinybullet {\em reduced} is the reduced mutant data returned from a single reduction method \\
\atab \atab \tinybullet $d$ is the data from the configuration of a reduction method and $x$ value\\

\vspace*{-.75em} \\

% \hspace{1em} \atab {\bf initialise} $d$ \smallvspace \\

\hspace{1em} \atab $d \leftarrow \emptyset$, $t \leftarrow 0$ \smallvspace \\

% \hspace{1em} \atab $t \leftarrow 0$ \smallvspace \\

\hspace{1em} \atab {\bf for each} $x \in \mbox{{\em percentages}}$ \smallvspace \\

\hspace{3em} \atab {\bf for} $t \leq \mbox{{\em trials}}$ \smallvspace \\

\hspace{4em} \atab {\bf let} $\mbox{{\em reduced}} \leftarrow \mbox{{\em reduction\,technique}}\,(\mbox{{\em data}},\,x)$ \smallvspace \\

\hspace{4em} \atab {\bf let} $d \leftarrow$ {\bf append}($d$, $\mbox{{\em reduced}}$) \smallvspace \\

\hspace{4em} \atab $t \leftarrow t + 1$ \smallvspace \\

\hspace{1em} \atab {\bf return} $d$ \smallvspace \\

{\bf end function} \\
\vspace{-0.5em} \\

\end{tabular} \\

\hline
\end{tabular}}

\caption{\label{fig:experiment-structure} An experiment function for studying mutant sampling.}

\vspace{-1em}

\end{figure}



