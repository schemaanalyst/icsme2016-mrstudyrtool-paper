%!TEX root=../icsme2016-mrstudyr.tex

\section{Introduction}

% Determining the quality of a test suite illustrates how stringently the
% given program was tested.

% Introduce software testing

Software developers may introduce errors into a program's source code that could result in a human
fatality~\cite{vicente2003programming}. Running a set of tests, frequently called a test suite, often aids in detecting
the faults that cause a program to function incorrectly~\cite{wagner2005comparing}. Yet, simply testing a program is not
sufficient. In order for testing to establish a high confidence in the correctness of the program under test, the test
suite needs to be of ``high quality''.

% Introduce mutation testing and its benefits

Mutation testing is a widely-recognized technique for assessing the quality of a test suite~\cite{gopinath2015mutation}.
While there are many potential faults for a program, mutation testing focuses on those that are ``close'' to the correct
version, with the expectation that they will be representative of all faults~\cite{jia2011analysis}. Of the methods for
evaluating test quality, mutation testing is widely considered the strongest test criterion in terms of its capability
to necessitate the creation of tests that find many faults~\cite{ammann2008introduction}.

% Talk about the limitations of mutation testing

Although mutation testing effectively requires tests to detect faults, it possesses major drawbacks in its computational
cost and the amount of human interaction it requires, often making it impractical to use~\cite{gopinath2015mutation,
wong1995reducing, gopinath2015empirical}.  A major computational cost of mutation testing comes from executing each test
case in a test suite for the many generated mutants~\cite{offutt1993experimental}. Executing a small, representative set
of mutants against the test suite has previously been proposed as a technique to reduce the cost of mutation
testing~\cite{jia2011analysis, wong1995reducing, offutt1993experimental}; this reduction strategy is categorized by
Offutt and Untch as a ``do fewer'' approach~\cite{offutt2001mutation}.

% Introduce the technique call random mutant sampling

% Due to its simplicity, mutant sampling is considered to be the most cost-effective reduction technique by avoiding
% implementation complexities~\cite{gopinath2015mutation}.  Although it is simple to implement, mutant sampling has still
% been empirically shown to outperform other techniques~\cite{zhang2010operator}.

There are several mutant reduction techniques in the ``do fewer'' category, with mutant sampling being a simple method
that randomly selects a subset of all mutants~\cite{wong1995reducing}. In addition to being conceptually
simple~\cite{gopinath2015mutation}, mutant sampling has been experimentally shown to outperform other more sophisticated
methods~\cite{zhang2010operator}.  Two sub-techniques within mutant sampling are called uniform random sampling and
sampling over operators~\cite{gopinath2015mutation, gopinath2015empirical}.  For both of these sub-techniques, a
threshold for the maximum percentage of selected mutants is set to $x$, which is then either applied to the entire set
of mutants or to each set of mutants produced by an operator~\cite{gopinath2015mutation, gopinath2015empirical}.

% GMK NOTE: Cut this sentence as it is not precisely the point that we want to main in this paper

% While this way of evaluating reduction techniques is accurate, it is both expensive to implement and perform due to the
% number of times mutation testing needs to be executed.

% GMK NOTE: Here is the source code analysis of the PIT tool using the cloc program (see script in the bin/ directory)

% ./cloc-git https://github.com/hcoles/pitest
% Cloning into 'temp-linecount-repo'...
% remote: Counting objects: 1240, done.
% remote: Compressing objects: 100% (931/931), done.
% remote: Total 1240 (delta 297), reused 557 (delta 122), pack-reused 0
% Receiving objects: 100% (1240/1240), 583.98 KiB | 0 bytes/s, done.
% Resolving deltas: 100% (297/297), done.
% Checking connectivity... done.
% ('temp-linecount-repo' will be deleted automatically)

     % 843 text files.
     % 833 unique files.
     %  63 files ignored.

% http://cloc.sourceforge.net v 1.60  T=1.79 s (456.9 files/s, 36462.9 lines/s)
% -------------------------------------------------------------------------------
% Language                     files          blank        comment           code
% -------------------------------------------------------------------------------
% Java                           780          11126           8087          43030
% Maven                           29            136             23           2287
% XML                              3             33             69            377
% CSS                              1              4              0             36
% Groovy                           2             10              0             32
% YAML                             1              4              1             21
% Bourne Shell                     2              0              0              2
% -------------------------------------------------------------------------------
% SUM:                           818          11313           8180          45785
% -------------------------------------------------------------------------------

% Discuss the challenges associated with these types of analyses, drawing on the code example of PIT

Prior work has found the smallest value of $x$ that still produces a representative set of
mutants~\cite{jia2011analysis, mathur1994empirical}. Yet, these efforts normally required the experimenters to integrate
a reduction technique into an existing mutation testing system before performing a mutation testing
experiment~\cite{demillo1988extended, king1991fortran}. Since mutation testing tools are often complex --- according to
the ``Count Lines of Code'' tool the  well-known PIT mutation testing system contains over 43,000 lines of non-commented
Java code and thousands of lines of build and configuration files --- this approach to studying mutant reduction methods
has a high upfront cost. That is, researchers must grasp the complexities of a mutation testing tool before they can
experimentally evaluate new reduction techniques.

% Retrospectively studying the data collected from a single execution of mutation testing is superior because of its
% minimal upfront costs and the implementation complexities of a target environment are obviated.

This paper presents

Retrospective analysis is more cost-effective than applying a mutant reduction technique and then evaluating it.  This
is the case because the expense of mutation testing in retrospective analysis is incurred only once, rather than for
every reduction technique. Analysing the effectiveness of an approach retrospectively is possible because the necessary
data can be collected from a prior analysis of all mutants.  This paper introduces \mr, the first and only tool that
analyses reduction techniques retrospectively.

\mr~is a tool for evaluating the effectiveness of mutation analysis reduction techniques in retrospect. Using data in a
generalised format from a single execution of mutation testing, \mr~employs each reduction strategy and performs
calculations to determine the efficiency and effectiveness. By accepting a generalised input format, \mr~is capable of
retrospectively analysing mutant reduction techniques for various domains.  Additionally, \mr~is modifiable,
well-documented and has been released on GitHub under and open-source license. The key contributions of this work are:


Since studying the data retrospectively removes the need to fathom the complexities of a target environment, mutation
testing can be extended to new domains (e.g., testing database schemas)~\cite{mcminn2016virtual,
mcminn2015effectiveness, wright2013efficient}.  Testing that a relational database's schema has correctly specified
integrity constraints is important because these restrictions ensure that only valid data enters the database. However,
little work has sought to test these integrity constraints~\cite{mcminn2016virtual}.

Databases are utilized pervasively for storing everything from batting and pitching statistics of baseball
players~\cite{lahmanbaseball} to fingerprints and scars of criminals~\cite{ngi}.  Relational databases are an efficient
and reliable way to store large amounts of data.  The acceptance of relationships between data entries, as well as
strict structural constraints upheld by the schema are what make relational databases unique.

The database's schema is the artefact responsible for specifying the data to be stored and how it should be structured
into tables. A series of integrity constraints is what make the database's schema able to ensure values are unique,
present, subject to domain-specific conditions and maintain referential integrity~\cite{mcminn2016virtual}.  Although
the schema is the last line of defence for the data contained within a database, little work has sought to test the
correctness of the integrity constraints~\cite{mcminn2015effectiveness}.

%!TEX root=../icsme2016-mrstudyr.tex

% Define the layers to draw the diagram
\pgfdeclarelayer{background}
\pgfdeclarelayer{foreground}
\pgfsetlayers{background,main,foreground}

% Define block styles used later

\tikzstyle{sensor}=[text width=5em,
    text centered, minimum height=2.5em]
\tikzstyle{darksensor}=[draw, fill=black!10, text width=6em,
    text centered, minimum height=2.5em,drop shadow]

% Define distances for bordering
\def\blockdist{1.5}
\def\edgedist{2.5}

\begin{figure}[t]
  \centering
\begin{tikzpicture}[thick,scale=0.85, every node/.style={scale=0.85}]
    \node [] at (0, 5) (map) {\textbf{Mutation Analysis Process}};
    \draw (-4, 4.75) -- (4, 4.75);
    \node [sensor] at (0, 4) (m1) {Generate Mutants};
    \node [sensor] at (0, 2.5) (m2) {Execute Generated Mutants};
    \node [sensor] at (3.25, 3.25) (t1) {Mutation Testing};
    \node [sensor] at (3.25, 1) (t2) {Retrospective Analysis};
    \draw (2, 4.4) -- (2, 1.9);
    \draw (1.7, 4.4) -- (2, 4.4);
    \draw (1.7, 1.9) -- (2, 1.9);
    \node [darksensor] at (0, 1) (m3) {Retrospective Analysis};
    \draw (2, 1.7) -- (2, 0.25);
    \draw (1.7, 1.7) -- (2, 1.7);
    \draw (1.7, 0.25) -- (2, 0.25);

    \path [draw, ->] (m1.south) -- node [above] {}
        (m2.90);
    \path [draw, ->] (m2.south) -- node [above] {}
        (m3.90);
    % \path [draw] (m1.east) -| +(0.3,0) |-node [above] {}
    %     (m2.0);
\end{tikzpicture}
\caption{\label{fig:process}The phases of the mutation analysis process.}
\vspace{-1.5em}
\end{figure}


% of database schemas~\cite{stackexchange}. Therefore, ensuring that

The little work testing the integrity constraints of a schema is not because of a sufficient understanding of the topic.
The 941,910 questions posted on the technical question and answer website, StackExchange, indicate the demand for
support of database schemas~\footnote{http://goo.gl/eZF1mK}. Therefore, ensuring that the schemas that are created are
also tested is important in upholding the integrity of the data contained within the database.

Due to its versatility, mutation testing can be used to detect errors made by programmers in many domains.  A domain
that mutation testing has become more prominent in recently is testing database schemas~\cite{mcminn2016virtual,
mcminn2015effectiveness, wright2013efficient}. However, due to its computation expense, it is not widely adopted in
industry to test database schemas.  Although mutation testing is used in a variety of domains, the reduction techniques
can still be utilized, including retrospective analysis~\cite{jia2011analysis, wong1995reducing, offutt1993experimental,
offutt2001mutation}.

% may need to fix these up a bit
    \begin{itemize}
        \item A tool to study mutant reduction techniques in retrospect.
        \item A tool that accepts a generalised input format.
        \item A tool released as a free and open-source R package.
    \end{itemize}
