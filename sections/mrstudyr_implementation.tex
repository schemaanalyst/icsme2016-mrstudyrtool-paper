\subsection{Implementation}

The R programming language for statistical data analysis was specifically designed to
be a full statistical language for conducting empirical studies. Following its predecessor, the S programming
language, R is a powerful means for data analysts to express computations \cite{ihaka1996r}.
Additionally, in R, the fundamental way to share code is via a package.

An R package includes code, data, documentation, tests and are easy to distribute \cite{wickham2015r}.
Since the \mr tool has been released as and R package, three commands in the R console
is all that are need to install the tool.
First, \texttt{install.packages("devtools")}, then \texttt{library(devtools)}
to install and load the \texttt{devtools}\cite{devtools} package, respectively.
The \texttt{devtools} package is necessary because it provides a framework for installing
and maintaining the tool. Finally, to obtain the
\mr\hspace*{-1em}\cite{mrstudyr} tool, use the \texttt{devtools} package to install it
from the popular Git repository hosting service, GitHub\cite{github},
using the following command: {\small\texttt{devtools::install\_github("mccurdyc/mrstudyr")}}.
In addition to being easy to install, the \mr tool accepts a generalised input format making
it extensible to various domains.

Accepting a generalised input format allows the \mr tool to be used to retrospectively analyse mutation testing
data from familiar and also emerging domains, such as databases. Although mutation
testing can be utilized in various domains, similar data can be collected from any
domain (e.g., mutation operators, mutant statuses, type of mutant). While more data
can optionally be collected from mutation testing, the additional data is more
dependent on the domain under observation.
