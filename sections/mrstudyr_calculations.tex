\subsection{Calculating Effectiveness with \mr}

\mr performs each reduction technique returning reduced mutant data,
displayed in Figure~\ref{fig:mrstudyr}. The effectiveness of a reduction technique is then evaluated
based on five metrics: mutation score, correlation, the creation-cost reduction percentage
and the mean absolute and root mean squared errors.

These five evaluation metrics help determine how effective a reduced set is at
representing all mutants. \mr calculates mutation score
by dividing the number of killed mutants by the total number of normal---not
equivalent, duplicate or stillborn---mutants. Then, provided by the ``Kendal'' R package,
\mr calculates Kendall's \taub~coefficient---because of its tie-awareness---to
determine the correlation between mutation scores of all mutants and each reduced set~\cite{mcleod2005kendall}.

Kendall's \taub~is a measurement of correlation between -1 and 1, representing
a strong negative or positive correlation, respectively. A measurement
of 0 means that there is no correlation. Additionally, we adopt the
Guildford scale to describe correlation measurements, with the absolute value
of a coefficient being described as ``low'', ``moderate'', ``high'' or ``very high'' when
it is less than 0.4, between 0.4 and 0.7, ranging from 0.7 to 0.9 or 0.9 and above, respectively \cite{inozemtseva2014coverage}.

% Is there a better way to re-word calculating a percentage than provided here?
Next, a percentage representing the reduction magnitude in creation cost for all mutants
and each reduced set is calculated. \mr calculates this by subtracting
the cost of creating the reduced set from the cost of creating all mutants, then
divides that by the cost of creating all mutants. This helps
determine whether performing a reduction technique is cost-effective.

Finally, \mr calculates both mean absolute (MAE) and root mean
squared errors (RMSE). MAE gives an equal weight to all errors allowing for
a high-level understanding, while RMSE emphasises large deviations in mutation scores.
These errors exploit the deviation between the mutation score of all mutants and
the reduced set.
