\subsection{Calculating Effectiveness with \mr}

In order to calculate the effectiveness of a reduction technique, the technique
must first be exercised on the data collected from mutation testing. The \mr tool
performs each reduction technique returning a new, reduced mutant set,
displayed in Figure \ref{fig:mrstudyr}. The effectiveness of a reduction technique is then evaluated
based on five metrics: mutation score, the correlation of the reduced and non-reduced sets' mutation scores,
the scale of creation-cost reduction and the mean absolute and root mean squared error.

These five evaluation metrics help determine how representative each reduced
set is of the non-reduced set of mutants. Mutation score is calculated
by dividing the number of killed mutants over the total number of normal---not
equivalent, duplicate or stillborn---mutants. Then, as provided by the ``Kendal'' R package,
we use Kendall's $\tau_b$ correlation coefficient---because of its tie-awareness---to
determine the correlation value between mutation scores of the non-reduced and reduced sets \cite{mcleod2005kendall}.

Kendall's $\tau_b$ returns a measurement of correlation between -1 and 1, representing
a strong negative or positive correlation between two sets, respectively. A return value
of 0 means that there is no correlation between the sets. Additionally, we adopt the
Guildford scale to describe correlation values between sets, with the absolute value
of a coefficient being described as ``low'', ``moderate'', ``high'' or ``very high'' when
it is less than 0.4, between 0.4 and 0.7, ranging from 0.7 to 0.9 or 0.9 and above, respectively \cite{inozemtseva2014coverage}.

Next, a value representing the magnitude of diminution in the cost of
creating the mutants between the entire set
and reduced set is calculated. This value is calculated by subtracting
the cost of creating the reduced set from the cost of creating all mutants. Then
dividing that value by the cost of creating all mutants. This helps
determine whether it is worth performing a reduction technique---even if the
reduced set has a very highly correlated mutation score with the non-reduced set.

Finally, we chose to calculate both mean absolute (MAE) and root mean
squared errors (RMSE) giving equal weight to
all errors and extra weight to larger errors, respectively. This provides
us with the ability to see the deviation between the mutation score of the reduced and
non-reduced set for a given reduction technique. By using both MAE and RMSE,
we are able to calculate the error in the mutation score for each reduced set
at a high-level with MAE and with an emphasis where large deviations occur by using RMSE.
