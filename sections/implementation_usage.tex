\subsection{Tool Usage}

Releasing an R tool via a package makes installing and loading the tool
a matter of four commands, but this does not guarantee that the usability
of the tool itself will be as easy. The \mr tool was designed to be simplistic,
yet perform stringent empirical analyses on mutant reduction techniques. Before
analysing the reduction techniques, the data collected from the mutation testing
phase must be read into the \mr tool, this can be done by typing \small{\texttt{read\_data("filename")}}
into the R console after installing and loading \mr. This function expects the data
to be located in the \texttt{inst/extdata} directory and stored as a comma-separated values
file (CSV).

Then, immediately following reading in the data, analysis begins. Currently, the two most common
sub-techniques of mutant sampling are performed using the following:
\texttt{analyse\_random\_sampling($data$)} and \texttt{analyse\_across\_operators($data$)},
for performing random sampling and random sampling across operators for the provided data, respectively.
These functions will return
