\subsection{Tool Usage}

Releasing an R tool via a package makes installing and loading the tool
a matter of a few commands, but this does not guarantee that the usability
of the tool will be as simple. \mr~was designed to be simplistic,
yet perform stringent empirical analyses on mutant reduction techniques.
To display the results from \mr~analysing mutant reduction strategies for the
testing of real-world database schemas, the following commands will be tailored
toward the data that we collected from performing mutation testing.
The collected data can be read in via the following: {\small\texttt{data <- read\_data("sqlite-avmdefaults.dat")}}.
This function expects the data to be located in the \texttt{inst/extdata} folder
and stored as a comma-separated value file.

Immediately following providing the data to \mr, analysis begins. Currently,
the two most common sub-techniques of mutant sampling are performed using the following:
\texttt{analyse\_random\_sampling(data)} and \texttt{analyse\_across\_operators(data)},
for random sampling and sampling across operators, respectively.
These functions will both return a new dataframe containing the trial, evaluated
percentage, reduced and original numerator and denominator counts,
reduced and original time to generate mutants and the reduced and original mutation scores.
The output from these function is provided in the accompanying README file on the
tool's GitHub page~\cite{tool}. After performing a reduction technique its efficiency
and effectiveness is evaluated based on five calculations.

The five metrics that evaluate the efficiency and effectiveness are:
mutation score, correlation, percent reduction in creation cost and error
--- MAE and RMSE --- between the reduced and original mutation scores.
The five aforementioned calculations are all contained within a single ``parent'' function.

The parent function, \texttt{analyse\_calculations}, performs all five child
calculations, returning a single dataframe --- that can be used throughout the human
examination phase --- rather than five individual dataframes begin generated for
each calculation. This reduces the number of function calls that experimenters
need to make while using the \mr~tool to analyse reduction approaches.

Additionally, the \texttt{analyse\_calculations} function can be used to calculate the
efficiency and effectiveness for any number of reduction approaches; it is not structurally
limited to a single reduction method. As input, the \texttt{analyse\_calculations} function
expects the dataframe returned from performing a reduction approach.

The human examination phase still requires use of \mr~for clearly visualising the trends
in the data collected from the calculation phase --- specifically the \texttt{analyse\_calculations}
function. The \mr~tool takes advantage of the work of Hadley Wickham with widely-used
graphing package \texttt{ggplot2}~\cite{ggplot2}.

All of the visualisation functions provided by \mr~and their output can be found on the GitHub
page~\cite{tool}. These visualisations will help human examiners to construct a policy recommendation
as to which reduction technique should be used for the specific domain and data under observation.
