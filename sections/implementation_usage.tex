\subsection{Tool Usage}

% CJM: Moving this figure here to allow reader to follow the process of the mrstudr tool

%!TEX root=../icsme2016-mrstudyr.tex

\newcommand{\mx}[1]{\mathbf{\bm{#1}}} % Matrix command
\newcommand{\vc}[1]{\mathbf{\bm{#1}}} % Vector command

% Define the layers to draw the diagram
\pgfdeclarelayer{background}
\pgfdeclarelayer{foreground}
\pgfsetlayers{background,main,foreground}

% Define block styles used later

\tikzstyle{sensor}=[draw, fill=black!5, text width=15em,
text centered, minimum height=2.5em,drop shadow]
\tikzstyle{smallsensor}=[draw, fill=black!5, text width=4em,
text centered, minimum height=2.5em,drop shadow]
\tikzstyle{box}=[draw, text width=8em,
text centered, minimum width=17.5em, minimum height=3.5em]
\tikzstyle{calc}=[draw, fill=black!5, text width=2.5em,
text centered, rounded corners, minimum height=2.5em,drop shadow]
\tikzstyle{circle}=[draw, ellipse, fill=black!5, text width=11em,
text centered, minimum height=2.5em, drop shadow]
\tikzstyle{mr}=[draw, fill=black!20, text width=5em,
text centered, minimum height=17em, minimum width = 18em, drop shadow]

% Define distances for bordering
\def\blockdist{1.5}
\def\edgedist{2.5}

\begin{figure}[t]

  \vspace{-.75em}

  \centering
  % SIMPLE
  \begin{tikzpicture}[thick,scale=0.85, every node/.style={scale=0.85}]
    \node [mr] at (0, 0) (mr) {};
    \node [circle] at (0, 3.7) (md) {Original Data};
    \node [sensor] at (0, 2.25) (r) {Reduction Techniques};
    \node [circle] at (0, 0.75) (rmd) {Cumulated Reduced Data};
    \node [sensor] at (0, -0.75) (ra) {Efficiency \& Effectiveness Analysis};
    \node [box] at (0, -2.25) (box) {};
    \node [calc] at (-2.25, -2.25) (ms) {MS};
    \node [calc] at (-0.75, -2.25) (err) {Red.};
    \node [calc] at (0.75, -2.25) (corr) {Corr.};
    \node [calc] at (2.25, -2.25) (err) {Err.};
    \node [sensor] at (0, -3.75) (he) {Human Examination};
    \node [circle] at (0, -5.25) (pr) {Policy Recommendation};

    \path [draw, ->] (md.south) -- node [above] {}
      (r.90);
    \path [draw, ->] (r.south) -- node [above] {}
      (rmd.90);
    \path [draw, ->] (rmd.south) -- node [above] {}
      (ra.90);
    \path [draw, ->] (ra.south) --  node [above] {}
      (box.90);
    \path [draw, ->] (box.south) --  node [above] {}
      (he.90);
    \path [draw, ->] (he.south) -- node [above] {}
      (pr.90);
  \end{tikzpicture}

  \caption{\label{fig:mrstudyr}The inputs and outputs of the \mr~tool.}

  \captionpara{0.5}{0.9}{0.5}{In this figure, the dark square represents the \mr~tool and its constituent parts, a
    rectangle stands for a process, a rectangle with rounded edges is a calculation performed by \mr, and an ellipse
  symbolises a process output.}

  \vspace{-1.8em}
\end{figure}



% CJM: I think we should move 'data' to the same line as the function call

Releasing an R tool via a package makes installing and loading the tool a matter of a few commands,
but this does not guarantee tool usability. \mr~was designed to be simplistic, yet perform stringent
empirical analyses on mutant reduction techniques. To display the results from \mr~analysing mutant
reduction strategies for the testing of real-world database schemas, the following commands will be
tailored toward the data that we collected from performing mutation testing. The collected data can
be read in via the following: {\small\texttt{data <- read\_data("sqlite-avmdefaults.dat")}}. This
function expects the data to be located in the \texttt{inst/extdata} folder and stored as a
comma-separated value file.

% CJM: I noticed in section 2.C (Conducting Experiment Campaigns) that it already explains that mrstudyr
% focuses on mutant sampling. Therefore, I am going to remove the sentences in this section restating
% this fact. These sentences are commented-out immediately follow this comment.

% Currently, the two most common sub-techniques of mutant sampling are performed using the following:
% \texttt{analyse\_random\_sampling(data)} and \texttt{analyse\_across\_operators(data)},
% for random sampling and sampling across operators, respectively.

% CJM: Additionally, the attributes of the data that I describe are specific to mutant sampling. I will
% remove these attributes.

% CJM: Is including the following statement entirely necessary? I removed it for now.

% Sample output from the \texttt{analyse} function is provided in the accompanying README file on the
% tool's GitHub page~\cite{tool}.

The reduction techniques, reffered to in Figure~\ref{fig:mrstudyr}, are performed following the provision of mutant data collected from mutation testing ---
the ``original data'' --- to \mr. Using \mr~to perform  reduction techniques is a single command:
\texttt{analyse(data)}. Shown in Figure~\ref{fig:mrstudyr},  each reduction technique performed by the \texttt{analyse} function in \mr, returns
reduced data which is then cumulated into a data set containing the data from all of the reduction approaches.
This data includes the trial, the total number of mutants analysed, the number of
killed mutants, the generation time of each mutant and the mutation scores for the reduced and
original data. After performing the reduction techniques, their efficiency and effectiveness is evaluated based
on four calculations in the ``E \& E Analysis'' phase displayed in Figure~\ref{fig:mrstudyr}.

% CJM: I want to present the evaluation metrics of efficiency and effectiveness, then define them, then finally,
% how to perform them with mrstudyr.
The evaluation metrics in ``E \& E Analysis'' are: mutation score, percent reduction in creation cost,
correlation, and error. The mutation score is calculated by dividing the number of killed mutants by the total number
of mutants~\cite{}. Percent reduction in creation cost of a set of mutants is the cost of the reduced set subtracted
from the cost of the original set, divided by the cost of the original set. The correlation between the original and reduced sets' mutation score is calculated using Kendall's~\taub, supported
by the ``Kendall'' R package~\cite{mcleod2015kendall}. Finally, the error --- MAE and RMSE --- between the original and reduced sets' mutation score
is calculated using the existing ``Metrics'' R package~\cite{metrics}.

Three of the four aforementioned metrics are calculated by a single function in \mr, \texttt{analyse\_calculations}.
The \texttt{analyse\_calculations} function returns data with the values of the respective calculation.
The rather than five individual dataframes begin generated for
each calculation. This reduces the number of function calls that experimenters
need to make while using the \mr~tool to analyse reduction approaches.

Additionally, the \texttt{analyse\_calculations} function can be used to calculate the
efficiency and effectiveness for any number of reduction approaches; it is not structurally
limited to a single reduction method. As input, the \texttt{analyse\_calculations} function
expects the dataframe returned from performing a reduction approach.

The human examination phase still requires use of \mr~for clearly visualising the trends
in the data collected from the calculation phase --- specifically the \texttt{analyse\_calculations}
function. The \mr~tool takes advantage of the work of Hadley Wickham with widely-used
graphing package \texttt{ggplot2}~\cite{ggplot2}.

All of the visualisation functions provided by \mr~and their output can be found on the GitHub
page~\cite{tool}. These visualisations will help human examiners to construct a policy recommendation
as to which reduction technique should be used for the specific domain and data under observation.
