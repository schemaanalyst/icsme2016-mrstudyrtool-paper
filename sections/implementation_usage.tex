\subsection{Tool Usage}

% CJM: Moving this figure here to allow reader to follow the process of the mrstudr tool

%!TEX root=../icsme2016-mrstudyr.tex

\newcommand{\mx}[1]{\mathbf{\bm{#1}}} % Matrix command
\newcommand{\vc}[1]{\mathbf{\bm{#1}}} % Vector command

% Define the layers to draw the diagram
\pgfdeclarelayer{background}
\pgfdeclarelayer{foreground}
\pgfsetlayers{background,main,foreground}

% Define block styles used later

\tikzstyle{sensor}=[draw, fill=black!5, text width=15em,
text centered, minimum height=2.5em,drop shadow]
\tikzstyle{smallsensor}=[draw, fill=black!5, text width=4em,
text centered, minimum height=2.5em,drop shadow]
\tikzstyle{box}=[draw, text width=8em,
text centered, minimum width=17.5em, minimum height=3.5em]
\tikzstyle{calc}=[draw, fill=black!5, text width=2.5em,
text centered, rounded corners, minimum height=2.5em,drop shadow]
\tikzstyle{circle}=[draw, ellipse, fill=black!5, text width=11em,
text centered, minimum height=2.5em, drop shadow]
\tikzstyle{mr}=[draw, fill=black!20, text width=5em,
text centered, minimum height=17em, minimum width = 18em, drop shadow]

% Define distances for bordering
\def\blockdist{1.5}
\def\edgedist{2.5}

\begin{figure}[t]

  \vspace{-.75em}

  \centering
  % SIMPLE
  \begin{tikzpicture}[thick,scale=0.85, every node/.style={scale=0.85}]
    \node [mr] at (0, 0) (mr) {};
    \node [circle] at (0, 3.7) (md) {Original Data};
    \node [sensor] at (0, 2.25) (r) {Reduction Techniques};
    \node [circle] at (0, 0.75) (rmd) {Cumulated Reduced Data};
    \node [sensor] at (0, -0.75) (ra) {Efficiency \& Effectiveness Analysis};
    \node [box] at (0, -2.25) (box) {};
    \node [calc] at (-2.25, -2.25) (ms) {MS};
    \node [calc] at (-0.75, -2.25) (err) {Red.};
    \node [calc] at (0.75, -2.25) (corr) {Corr.};
    \node [calc] at (2.25, -2.25) (err) {Err.};
    \node [sensor] at (0, -3.75) (he) {Human Examination};
    \node [circle] at (0, -5.25) (pr) {Policy Recommendation};

    \path [draw, ->] (md.south) -- node [above] {}
      (r.90);
    \path [draw, ->] (r.south) -- node [above] {}
      (rmd.90);
    \path [draw, ->] (rmd.south) -- node [above] {}
      (ra.90);
    \path [draw, ->] (ra.south) --  node [above] {}
      (box.90);
    \path [draw, ->] (box.south) --  node [above] {}
      (he.90);
    \path [draw, ->] (he.south) -- node [above] {}
      (pr.90);
  \end{tikzpicture}

  \caption{\label{fig:mrstudyr}The inputs and outputs of the \mr~tool.}

  \captionpara{0.5}{0.9}{0.5}{In this figure, the dark square represents the \mr~tool and its constituent parts, a
    rectangle stands for a process, a rectangle with rounded edges is a calculation performed by \mr, and an ellipse
  symbolises a process output.}

  \vspace{-1.8em}
\end{figure}



% CJM: I think we should move 'data' to the same line as the function call

Release as an R package makes installing and loading a tool easy, but does not guarantee usability.
\mr~was designed to be simplistic, yet perform stringent empirical analyses on mutant reduction techniques. To
display the results from \mr~analysing mutant reduction strategies for the testing of real-world database schemas,
the following commands will be tailored toward the data from performing mutation testing. The collected data can
be read in via the following: {\small\texttt{data <- read\_data("sqlite-avmdefaults.dat")}}. This
function expects the data to be located in the \texttt{inst/extdata} folder and stored as a
comma-separated value file.

% CJM: I noticed in section 2.C (Conducting Experiment Campaigns) that it already explains that mrstudyr
% focuses on mutant sampling. Therefore, I am going to remove the sentences in this section restating
% this fact. These sentences are commented-out immediately follow this comment.

% Currently, the two most common sub-techniques of mutant sampling are performed using the following:
% \texttt{analyse\_random\_sampling(data)} and \texttt{analyse\_across\_operators(data)},
% for random sampling and sampling across operators, respectively.

% CJM: Additionally, the attributes of the data that I describe are specific to mutant sampling. I will
% remove these attributes.

% CJM: Is including the following statement entirely necessary? I removed it for now.

% Sample output from the \texttt{analyse} function is provided in the accompanying README file on the
% tool's GitHub page~\cite{tool}.

% This data includes the trial, total number of mutants analysed, number of killed mutants, mutant set generation time
% and the mutation scores for the reduced and original sets. After performing the reduction techniques,

``Reduction Techniques'', as referred to in Figure~\ref{fig:mrstudyr}, are performed following the provision
of mutant data collected from mutation testing on all mutants --- the ``Original Data'' --- to \mr. Using
\mr~to perform reduction techniques is a single command: \texttt{analyse(data)}. Each reduction technique performed by
the \texttt{analyse} function returns reduced data which is then cumulated into a single data set, shown in
Figure~\ref{fig:mrstudyr}. After performing a reduction technique, it is evaluated in ``Efficiency and Effectiveness
Analysis'', displayed in Figure~\ref{fig:mrstudyr}.
% CJM: I want to present the evaluation metrics of efficiency and effectiveness, then define them, then finally,
% how to perform them with mrstudyr.
The evaluation calculations in efficiency and effectiveness analysis, and their respective abbreviations, are as
follows: mutation score
(MS), corresponding cost reduction (Red.), correlation (Corr.), and error (Err.).

% Cut due to space constraints
% The mutation score of a set is calculated
% by dividing the number of killed mutants by the total number of mutants~\cite{wong1995reducing}. The reduction
% in creation cost for a set of mutants is the cost of the reduced set subtracted from the cost of the original set,
% divided by the cost of the original set. The original and reduced sets' mutation score correlation is calculated
% using Kendall's~\taub, supported by the ``Kendall'' R package~\cite{mcleod2015kendall}. Finally, the error --- both
% MAE and RMSE --- between the original and reduced sets' mutation score is calculated using the existing ``Metrics''
% R package~\cite{metrics}.

The aforementioned metrics, with the exception of correlation, are calculated by a single function in \mr,
\texttt{analyse\_calculations(data)}, where the input is the cumulated reduced data and the output is a data set with
the values of the respective calculation. To calculate correlation between reduced and original mutation scores, the
\texttt{analyse\_correlation(data)} function is used, where the input to this function, is again, the cumulated reduced
data and the output is a data set containing only Kendall's \taub~correlation coefficient~\cite{mcminn2016virtual}. The
simplicity of \mr~is upheld by only requiring the use of two commands to analyse the efficiency and effectiveness of
reducers. This same simplicity is continued in the final ``Human Examination'' phase.

% CJM: removed this paragraph because I no longer think that it is needed and to save space for preliminary study
% Additionally, the \texttt{analyse\_calculations} function can be used to calculate the
% efficiency and effectiveness for any number of reduction approaches; it is not structurally
% limited to a single reduction method. As input, the \texttt{analyse\_calculations} function
% expects the dataframe returned from performing a reduction approach.

The human examination phase still requires use of \mr~for visualising the trends in the data collected from efficiency
and effectiveness analysis. The \mr~tool takes advantage of Hadley Wickham's widely-used graphing package,
\texttt{ggplot2}~\cite{ggplot2}. The visualisation functions provided by \mr~and an screencast demonstrating the use of
the tool~\cite{asciinema} can be found on \mr's GitHub page~\cite{tool}. The documentation, tutorial, screencast, and
visualisations will help researchers and industrialists construct a policy recommendation regarding which reduction
technique should be used for the specific domain and data under observation.

% vim: ft=tex
%!TEX root=../icsme2016-mrstudyr.tex

\begin{table}[t!]
  % \vspace{-0.75em}
    \caption{Schemas analysed in the empirical study.}\label{tbl:study-schemas}
  \vspace{-1em}
  \footnotesize
  \centering
  \scalebox{\tablescalefactor}{
    \begin{tabular}{l@{\hskip -5pt}rrrrrrrr}
      {Schema} & \rot{Tables} & \rot{Columns} & \rot{Checks} & \rot{Foreign Keys} & \rot{Not Nulls} & \rot{Primary Keys} & \rot{Uniques} & \rot{$\sum$Constraints} \\
      \toprule

      CoffeeOrders & 5 & 20 & 0 & 4 & 10 & 5 & 0 & 19 \\
      Employee & 1 & 7 & 3 & 0 & 0 & 1 & 0 & 4 \\
      Inventory & 1 & 4 & 0 & 0 & 0 & 1 & 1 & 2 \\
      Iso3166 & 1 & 3 & 0 & 0 & 2 & 1 & 0 & 3 \\
      JWhoisServer & 6 & 49 & 0 & 0 & 44 & 6 & 0 & 50 \\
      MozillaPermissions & 1 & 8 & 0 & 0 & 0 & 1 & 0 & 1 \\
      NistWeather & 2 & 9 & 5 & 1 & 5 & 2 & 0 & 13 \\
      Person & 1 & 5 & 1 & 0 & 5 & 1 & 0 & 7 \\
      Products & 3 & 9 & 4 & 2 & 5 & 3 & 0 & 14 \\
      \midrule
      Total & 21 & 114 & 13 & 7 & 71 & 21 & 1 & 113 \\

      \bottomrule
    \end{tabular}
  }
  \vspace{-1em}
\end{table}

