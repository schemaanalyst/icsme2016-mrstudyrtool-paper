\subsection{Tool Usage}

Releasing an R tool via a package makes installing and loading the tool
a matter of a few commands, but this does not guarantee that the usability
of the tool will be as simple. \mr~was designed to be simplistic,
yet perform stringent empirical analyses on mutant reduction techniques.
To display the results from \mr~analysing mutant reduction strategies for the
testing of real-world database schemas, the following commands will be tailored
toward the data that we collected from performing mutation testing.
The collected data can be read in via the following: {\small\texttt{data <- read\_data("sqlite-avmdefaults")}}.
This function expects the data to be located in the \texttt{inst/extdata} folder
and stored as a comma-separated value file.

Immediately following providing the data to \mr, analysis begins. Currently,
the two most common sub-techniques of mutant sampling are performed using the following:
\texttt{analyse\_random\_sampling(data)} and \texttt{analyse\_across\_operators(data)},
for performing random sampling and random sampling across operators, respectively.
These functions will both return a new dataframe containing the trial, evaluated
percentage, reduced and original numerator and denominator counts,
reduced and original time to generate mutants and the reduced and original mutation scores.
The output from these function is provided in the accompanying README file on the
tool's GitHub page~\cite{tool}. After performing a reduction technique its efficiency
and effectiveness is evaluated based on five calculation functions.

The five functions that evaluate the efficiency and effectiveness are:
mutation score, correlation coefficient, magnitude in reduction of creation cost
(\%) and error---MAE and RMSE---between the reduced and original mutation scores.
The calculation functions are as follows, in order:

\begin{itemize}
    \item {\small\texttt{analyse\_mutation\_score(data)}}
    \item {\small\texttt{analyse\_correlation(data)}}
    \item {\small\texttt{analyse\_reduction(data)}}
    \item {\small\texttt{analyse\_percents\_error(data)}}
\end{itemize}

These functions each return a dataframe including the result of the calculation.
Again, the output from the calculation functions can be found on
the tool's GitHub page~\cite{tool}. Following efficiency and effectiveness calculations
is human examination, only succeeded by the establishment of a policy recommendation.

The human examination phase still requires use of \mr~for clearly visualising the trends
in the data collected from the calculation phase. The \mr~tool takes advantage of the
widely-used graphing package, \texttt{ggplot2}~\cite{ggplot2}. All of the visualisation
functions provided and their output can be found on the GitHub page~\cite{tool}.
These visualisations will help human examiners to construct a policy recommendation as
to which reduction technique should be used for the specific domain and data under observation.
