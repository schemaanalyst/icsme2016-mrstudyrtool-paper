\subsection{Tool Usage}

Releasing an R tool via a package makes installing and loading the tool
a matter of four commands, but this does not guarantee that the usability
of the tool itself will be as easy. The \mr tool was designed to be simplistic,
yet perform stringent empirical analyses on mutant reduction techniques. Before
analysing the reduction techniques, the data collected from the mutation testing
phase must be read into the \mr tool, this can be done by typing {\small\texttt{read\_data("filename")}}
into the R console after installing and loading \mr. This function expects the data
to be located in the \texttt{inst/extdata} directory and stored as a comma-separated values
file (CSV).

Then, immediately following reading in the data, analysis begins. Currently, the two most common
sub-techniques of mutant sampling are performed using the following:
\texttt{analyse\_random\_sampling($data$)} and \texttt{analyse\_across\_operators($data$)},
for performing random sampling and random sampling across operators for the provided $data$, respectively.
These functions will both return a new dataframe containing the trial, percentage being evaluated, reduced
numerator and denominator counts, the original numerator and denominator counts, the reduced and original time
to generate all mutants and the reduced and original mutation scores. Example output is provided in the accompanying
README file on the tool's GitHub page~\cite{mrstudyr}. After performing a reduction technique its effectiveness
is evaluated.

There are four calculations that evaluate the effectiveness of a reduction technique: mutation score, correlation coefficient
and error---MAE and RMSE---between the reduced and original mutation scores. The calculation functions are as follows (in order):
analyse\_mutation\_score($data$), analyse\_correlation($data$) and analyse\_percents\_error($data$). These functions all return
a dataframe containing the data from the performed calculation. Sample output from the calculation function can be found on
the tool's GitHub page~\cite{mrstudyr}. Following the effectiveness calculation phase is the human examination phase.

The human examination phase still requires use of the \mr tool for clearly visualising the trends in the data
collected from the effectiveness calculation phase. The \mr tool takes advantage of an existing graphing package, \texttt{ggplot2}~\cite{ggplot2}.
All of the visualisation functions provided by the \mr tool can be found on the GitHub page~\cite{mrstudyr}. These
visualisations will help a human examiner to construct a policy recommendation as to which reduction technique
should be used for the specific domain and data under observation.
