\subsection{Tool Usage}

Releasing an R tool via a package makes installing and loading the tool
a matter of a few commands, but this does not guarantee that the usability
of the tool will be as simple. \mr was designed to be simplistic,
yet perform stringent empirical analyses on mutant reduction techniques. Before
analysing the reduction techniques, the data collected from mutation testing
must be read in, this can be done in the R console by typing {\small\texttt{data <- read\_data("filename")}}.
This function expects the data to be located in the \texttt{inst/extdata} folder
and stored as a comma-separated value file.

Then, immediately following reading in the data, analysis begins. Currently, the two most common
sub-techniques of mutant sampling are performed using the following:
\texttt{analyse\_random\_sampling(data)} and \texttt{analyse\_across\_operators(data)},
for performing random sampling and random sampling across operators, respectively. \texttt{data} refers to the
newly assigned variable containing the input data in a dataframe.
These functions will both return a new dataframe containing the trial, evaluated percentage, reduced
numerator and denominator counts, original numerator and denominator counts, reduced and original time
to generate mutants and the reduced and original mutation scores. Example output is provided in the accompanying
README file on the tool's GitHub page~\cite{tool}. After performing a reduction technique its efficiency and
effectiveness is evaluated based on five calculation functions.

The five functions that evaluate the effectiveness of a reduction technique are: mutation score, correlation coefficient, magnitude
in reduction of creation cost (\%) and error---MAE and RMSE---between the reduced and original mutation scores.
The calculation functions are as follows, in order:
\begin{itemize}
\item \texttt{analyse\_mutation\_score(data)}
\item \texttt{analyse\_correlation(data)}
\item \texttt{analyse\_reduction(data)}
\item \texttt{analyse\_percents\_error(data)}
\end{itemize}
These functions each return a dataframe including the result of the calculation.
Sample output from the calculation functions can be found on
the tool's GitHub page~\cite{tool}. Following efficiency and effectiveness calculations is human examination.

The human examination phase still requires use of \mr for clearly visualising the trends in the data
collected from the calculation phase. The \mr tool takes advantage of the widely-used graphing package, \texttt{ggplot2}~\cite{ggplot2}.
All of the visualisation functions provided and their output can be found on the GitHub page~\cite{tool}.
These visualisations will help human examiners to construct a policy recommendation as to which reduction technique
should be used for the specific domain and data under observation.
