\subsection{Tool Usage}

%!TEX root=../icsme2016-mrstudyr.tex

\newcommand{\mx}[1]{\mathbf{\bm{#1}}} % Matrix command
\newcommand{\vc}[1]{\mathbf{\bm{#1}}} % Vector command

% Define the layers to draw the diagram
\pgfdeclarelayer{background}
\pgfdeclarelayer{foreground}
\pgfsetlayers{background,main,foreground}

% Define block styles used later

\tikzstyle{sensor}=[draw, fill=black!5, text width=15em,
text centered, minimum height=2.5em,drop shadow]
\tikzstyle{smallsensor}=[draw, fill=black!5, text width=4em,
text centered, minimum height=2.5em,drop shadow]
\tikzstyle{box}=[draw, text width=8em,
text centered, minimum width=17.5em, minimum height=3.5em]
\tikzstyle{calc}=[draw, fill=black!5, text width=2.5em,
text centered, rounded corners, minimum height=2.5em,drop shadow]
\tikzstyle{circle}=[draw, ellipse, fill=black!5, text width=11em,
text centered, minimum height=2.5em, drop shadow]
\tikzstyle{mr}=[draw, fill=black!20, text width=5em,
text centered, minimum height=17em, minimum width = 18em, drop shadow]

% Define distances for bordering
\def\blockdist{1.5}
\def\edgedist{2.5}

\begin{figure}[t]

  \vspace{-.75em}

  \centering
  % SIMPLE
  \begin{tikzpicture}[thick,scale=0.85, every node/.style={scale=0.85}]
    \node [mr] at (0, 0) (mr) {};
    \node [circle] at (0, 3.7) (md) {Original Data};
    \node [sensor] at (0, 2.25) (r) {Reduction Techniques};
    \node [circle] at (0, 0.75) (rmd) {Cumulated Reduced Data};
    \node [sensor] at (0, -0.75) (ra) {Efficiency \& Effectiveness Analysis};
    \node [box] at (0, -2.25) (box) {};
    \node [calc] at (-2.25, -2.25) (ms) {MS};
    \node [calc] at (-0.75, -2.25) (err) {Red.};
    \node [calc] at (0.75, -2.25) (corr) {Corr.};
    \node [calc] at (2.25, -2.25) (err) {Err.};
    \node [sensor] at (0, -3.75) (he) {Human Examination};
    \node [circle] at (0, -5.25) (pr) {Policy Recommendation};

    \path [draw, ->] (md.south) -- node [above] {}
      (r.90);
    \path [draw, ->] (r.south) -- node [above] {}
      (rmd.90);
    \path [draw, ->] (rmd.south) -- node [above] {}
      (ra.90);
    \path [draw, ->] (ra.south) --  node [above] {}
      (box.90);
    \path [draw, ->] (box.south) --  node [above] {}
      (he.90);
    \path [draw, ->] (he.south) -- node [above] {}
      (pr.90);
  \end{tikzpicture}

  \caption{\label{fig:mrstudyr}The inputs and outputs of the \mr~tool.}

  \captionpara{0.5}{0.9}{0.5}{In this figure, the dark square represents the \mr~tool and its constituent parts, a
    rectangle stands for a process, a rectangle with rounded edges is a calculation performed by \mr, and an ellipse
  symbolises a process output.}

  \vspace{-1.8em}
\end{figure}



% Explain how you have to store the data in the mrstudyr package

We designed the \mr~tool to make it simple to perform a thorough and automated empirical analysis of mutant reduction
techniques. The common structure of an R package expects that externally-collected data is stored in the
\texttt{inst/extdata} folder. This is the location where \mr~expects to find the mutation data, stored as a
comma-separated value file; in this paper, we ran the \sa~tool~\cite{McMinn2016} to generate the data used as input
to \mr.

% To display the results from \mr~performing mutant reduction strategies for the testing of real-world
% database schemas, the following commands will be tailored toward the data from performing mutation testing.

% Release as an R package makes installing and loading a tool easy, but does not guarantee usability.

The ``Reduction Techniques'', as referenced in Figure~\ref{fig:mrstudyr}, are performed following the provision of
mutant data, the ``Original Data'', to \mr. Using \mr~to perform analyses and create visualisations is a single command
per-technique.  For random and over operator sampling, the commands are {\texttt{create\_random\_sampling\_graphs()}}
and {\texttt{create\_operator\_sampling\_graphs()}}, respectively. Both functions accumulate the analysed reduced data
over 30 trials into a single data set, as shown by the ``Accumulated Reduced Data'' ellipse in
Figure~\ref{fig:mrstudyr}. After performing a reduction technique, it is evaluated in ``Efficiency and Effectiveness
Analysis''. The evaluation calculations in efficiency and effectiveness analysis, and their respective abbreviations,
are as follows: mutation score (MS), cost reduction (Red.), correlation (Corr.), and error (Err.).

% Due to space constraints, did not discuss the difference between RMSE and MAE.

While mutation score and cost reduction are calculated in the function performing the analysis, Kendall's
\taub~correlation coefficient~\cite{mcminn2016virtual} and the chosen errors are calculated by
\texttt{analyse\_calculations()}, where the input is the accumulated reduced data from a reduction technique
and the output is a new data set with the values of the respective calculation.  The errors calculated by
\mr~are the root mean square error (RMSE) and the mean absolute error (MAE), both ``lower-is-better'' metrics.
These error calculations were chosen due to their wide adoption to evaluate model performance~\cite{chai2014root}.
Demonstrating the usability of \mr~is the fact that it requires only two commands to perform the analysis and
calculate model evaluation. This same simplicity also remains true for the final phase, ``Human Examination''.

The human examination phase still requires use of \mr~to present the results of the efficiency and
effectiveness analysis. \mr~takes advantage of Hadley Wickham's widely-used graphing package,
\texttt{ggplot2}~\cite{ggplot2}, to create comprehensible visualisations, which will help researchers and
industrialists construct a policy recommendation regarding which reduction technique should be used for the
data under observation. The documentation, a screencast~\cite{asciinema}, and useful visualisations can be
found on \mr's GitHub page~\cite{tool}.

% vim: ft=tex
%!TEX root=../icsme2016-mrstudyr.tex

\begin{table}[t!]
  % \vspace{-0.75em}
    \caption{Schemas analysed in the empirical study.}\label{tbl:study-schemas}
  \vspace{-1em}
  \footnotesize
  \centering
  \scalebox{\tablescalefactor}{
    \begin{tabular}{l@{\hskip -5pt}rrrrrrrr}
      {Schema} & \rot{Tables} & \rot{Columns} & \rot{Checks} & \rot{Foreign Keys} & \rot{Not Nulls} & \rot{Primary Keys} & \rot{Uniques} & \rot{$\sum$Constraints} \\
      \toprule

      CoffeeOrders & 5 & 20 & 0 & 4 & 10 & 5 & 0 & 19 \\
      Employee & 1 & 7 & 3 & 0 & 0 & 1 & 0 & 4 \\
      Inventory & 1 & 4 & 0 & 0 & 0 & 1 & 1 & 2 \\
      Iso3166 & 1 & 3 & 0 & 0 & 2 & 1 & 0 & 3 \\
      JWhoisServer & 6 & 49 & 0 & 0 & 44 & 6 & 0 & 50 \\
      MozillaPermissions & 1 & 8 & 0 & 0 & 0 & 1 & 0 & 1 \\
      NistWeather & 2 & 9 & 5 & 1 & 5 & 2 & 0 & 13 \\
      Person & 1 & 5 & 1 & 0 & 5 & 1 & 0 & 7 \\
      Products & 3 & 9 & 4 & 2 & 5 & 3 & 0 & 14 \\
      \midrule
      Total & 21 & 114 & 13 & 7 & 71 & 21 & 1 & 113 \\

      \bottomrule
    \end{tabular}
  }
  \vspace{-1em}
\end{table}

