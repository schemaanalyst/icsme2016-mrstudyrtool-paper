\subsection{Tool Usage}

\newcommand{\mx}[1]{\mathbf{\bm{#1}}} % Matrix command
\newcommand{\vc}[1]{\mathbf{\bm{#1}}} % Vector command

% Define the layers to draw the diagram
\pgfdeclarelayer{background}
\pgfdeclarelayer{foreground}
\pgfsetlayers{background,main,foreground}

% Define block styles used later

\tikzstyle{sensor}=[draw, fill=black!10, text width=5em,
    text centered, minimum height=2.5em,drop shadow]
\tikzstyle{small}=[draw, fill=black!10, text width=2em,
    text centered, minimum width=2em, minimum height=2.5em,drop shadow]
\tikzstyle{smallcir}=[draw, ellipse, fill=black!10, text width=5em,
    text centered, minimum height=2.5em, drop shadow]
\tikzstyle{circle}=[draw, ellipse, fill=black!10, text width=8em,
    text centered, minimum height=2.5em, drop shadow]
\tikzstyle{func}=[draw, fill=black!20, text width=5em,
    text centered, minimum height=2.5em,drop shadow]
\tikzstyle{mr}=[draw, fill=black!20, text width=5em,
    text centered, minimum height=13em, minimum width = 15em, drop shadow]
\tikzstyle{ann} = [above, text width=5em, text centered]
\tikzstyle{wa} = [sensor, text width=10em, fill=black!30,
    minimum height=6em, rounded corners, drop shadow]
\tikzstyle{sc} = [sensor, text width=13em, fill=red!30,
    minimum height=10em, rounded corners, drop shadow]


% Define distances for bordering
\def\blockdist{1.5}
\def\edgedist{2.5}

\begin{figure}[t]
  \centering
% SIMPLE
\begin{tikzpicture}[thick,scale=0.85, every node/.style={scale=0.85}]
    \node [mr] at (0, 0) (mr) {};
    \node [smallcir] at (0, 3.25) (md) {Mutant Data};
    \node [sensor] at (0, 1.5) (r) {Reduction Approach};
    \node [sensor] at (0, 0) (rmd) {Reduced Mutant Data};
    \node [small] at (-1.75, 0) (m1) {$ms'$};
    \node [small] at (1.75, 0) (m2) {$ms$};
    \node [func] at (0, -1.5) (calc) {Calculations};
    \node [circle] at (0, -3.25) (rec) {Policy Recommendation};

    \path [draw, ->] (md.south) -- node [above] {}
        (r.90);
    \path [draw, ->] (md.east) -| node [above] {}
        (m2.90);
    \path [draw, ->] (r.south) -- node [above] {}
        (rmd.90);
    \path [draw, ->] (rmd.west) -- node [above] {}
        (m1.0);
    \path [draw, ->] (m1.south) |- node [above] {}
        (calc.180);
    \path [draw, ->] (m2.south) |- node [above] {}
        (calc.0);
    \path [draw, ->] (calc.south) -- node [above] {}
        (rec.90);

% VERTICAL
% \begin{tikzpicture}[thick,scale=0.85, every node/.style={scale=0.85}]
%     \node [mr] at (0, 0) (mr) {};
%     % \node [] at (-3, -0.25) (left) {};
%     % \node [] at (0, 2.3) (text) {\textit{mrstudyr}};
%     \node [sensor] at (-1.25, 0) (g) {};
%     \node [sensor] at (1.25, 0) (gg) {};
%     \node [circle] at (0, 4.5) (md) {Mutant Data};
%     \node [circle] at (0, 2.75) (df) {Dataframe};
%     \node [] at (-1.25, 1.75) (a1) {$a_1$};
%     \node [] at (0, 1.75) (dots) {...};
%     \node [] at (1.25, 1.75) (a2) {$a_n$};
%     \node [] at (-1.75, 1) (t1) {$t_1$};
%     \node [] at (-1.25, 1) (dots2) {...};
%     \node [] at (-0.75, 1) (t30) {$t_{30}$};
%     \node [] at (0.75, 1) (tt1) {$t_1$};
%     \node [] at (1.25, 1) (dots3) {...};
%     \node [] at (1.75, 1) (tt30) {$t_{30}$};
%     \node [] at (-1.75, 0) (m1) {$ms_{1}'$};
%     \node [] at (-0.75, 0) (m2) {$ms_{30}'$};
%     \node [] at (0.75, 0) (mm1) {$ms_{1}'$};
%     \node [] at (1.75, 0) (mm2) {$ms_{30}'$};
%     \node [func] at (-1.75, -1.75) (corr) {Calculate Correlation};
%     \node [func] at (1.75, -1.75) (err) {Calculate Error};
%     \node [circle] at (0, -2.8) (df2) {Dataframe};
%
%     \path [draw, ->] (md.south) -- node [above] {}
%         (df.90);
%     \path [draw, ->] (df.south) -- node [above] {}
%         (a1.10);
%     \path [draw, ->] (df.south) -- node [above] {}
%         (a2.170);
%     \path [draw, ->] (a1.south) -- node [above] {}
%         (t1.50);
%     \path [draw, ->] (a1.south) -- node [above] {}
%         (t30.130);
%     \path [draw, ->] (a2.south) -- node [above] {}
%         (tt1.50);
%     \path [draw, ->] (a2.south) -- node [above] {}
%         (tt30.130);
%     \path [draw, ->] (t1.south) -- node [above] {}
%         (m1.90);
%     \path [draw, ->] (t30.south) -- node [above] {}
%         (m2.90);
%     \path [draw, ->] (tt1.south) -- node [above] {}
%         (mm1.90);
%     \path [draw, ->] (tt30.south) -- node [above] {}
%         (mm2.90);
%     \path [draw, ->] (g.south) -| +(0, -0.25) -| node [above] {}
%         (err.90);
%     \path [draw, ->] (gg.south) -| +(0, -0.25) -| node [above] {}
%         (corr.90);
%
\end{tikzpicture}
\caption{\label{fig:mrstudyr}The inputs and outputs of mutation analysis phase
of the mutation analysis process, performed by the \mr \,tool. \\
\textit{The darkest square in the background represents the \mr \,tool.
A rectangle represents a process. A rectangle with rounded edges represents a
calculation performed by \mr. An ellipse represents the output from a process.}}
\end{figure}


Release as an R package makes installing and loading a tool easy, but does not guarantee usability.
\mr~was designed to be simple, yet perform stringent empirical analyses on mutant reduction techniques.
To display the results from \mr~performing mutant reduction strategies for the testing of real-world
database schemas, the following commands will be tailored toward the data from performing mutation
testing.

The common structure of an R package is to have externally-collected data in the \texttt{inst/extdata}
folder. This is the location where~\mr expects to find the collect mutation data, stored as a
comma-separated value file.

``Reduction Techniques'', as referred to in Figure~\ref{fig:mrstudyr}, are performed following the
provision of mutant data collected from mutation testing on all mutants --- the ``Original Data'' ---
to \mr. Using \mr~to perform analyses and create visualisations is a single command per-technique.
For random and across-operator sampling, the commands are {\texttt{create\_random\_sampling\_graphs()}}
and {\texttt{create\_operator\_sampling\_graphs()}}, respectively. Both functions accumulate the analysed
reduced data over 30 trials into a single data set, as shown by the ``Accumulated Reduced Data'' ellipse
in Figure~\ref{fig:mrstudyr}. After performing a reduction technique, it is evaluated in ``Efficiency and
Effectiveness Analysis''. The evaluation calculations in efficiency and effectiveness analysis, and their
respective abbreviations, are as follows: mutation score (MS), cost reduction (Red.), correlation (Corr.),
and error (Err.).

% TODO: fix space at end of paragraph
% Due to space constraints, did not discuss the difference between RMSE and MAE.

While mutation score and cost reduction are calculated in the function performing the analysis, Kendall's
\taub~correlation coefficient~\cite{mcminn2016virtual} and the chosen errors are calculated by
\texttt{analyse\_calculations()}, where the input is the accumulated reduced data from a reduction technique
and the output is a new data set with the values of the respective calculation.  The errors calculated by
\mr~are the root mean square error (RMSE) and the mean absolute error (MAE).  These error calculations were
chosen due to their wide adoption to evaluate model performance~\cite{chai2014root}. Performing the analysis
and calculating these model evaluation metrics demonstrates the simplicity of \mr~ by requiring only two
commands. This same theme is continued to the final phase, ``Human Examination''.

The human examination phase still requires use of \mr~to present the results of the efficiency and
effectiveness analysis. \mr~takes advantage of Hadley Wickham's widely-used graphing package,
\texttt{ggplot2}~\cite{ggplot2}, to create comprehensible visualisations, which will help researchers and
industrialists construct a policy recommendation regarding which reduction technique should be used for the
data under observation. The documentation, a screencast~\cite{asciinema}, and useful visualisations can be
found on \mr's GitHub page~\cite{tool}.

\input{tables/schemas}
