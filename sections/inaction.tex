\vspace*{-0.25cm}
\section{\textit{mrstudyr} in Action (to be changed)}

include tables about the mutant data (alive, dead, total, type) \\
provide results from more than just Inventory schema \\
Table including characteristics of the schemas \\
Graph and explaining trends and meaning of everything in graph (QSIC2104)



% The data analyzed in this paper was generated from performing mutation testing on
% 34 database schemas using the SQLite database management system. The data
% contain 118860 mutants, which as a whole, take 67287543 milliseconds to analyze using mutation analysis.
%
% Reducing this cost is the primary focus of this analysis. However, reducing cost by analyzing fewer mutants
% leads to weakly correlated mutation scores and higher mean absolute and root mean squared errors (displayed
% in Figures \ref{fig:comp_corr} and \ref{fig:comp_error}, respectively). In this paper, I will
% analyze the graphs comparing mutation score, error and correlation (on the y-axis) to percentage (on the x-axis),
% specifically for the Inventory schema.
%
% As previously mentioned, the mutation score associated with a test suite or database schema is a value
% corresponding to the quality of the test suite or schema. When reducing the mutant set by fractional values,
% we are able to receive mutation scores that are very close to the original mutation score (e.g., both uniform random sampling
% and stratified random sampling over operators on only $30\%$ of the original mutants produce mutation scores ranging between 0.70 and 0.80, where the
% original mutation score is 0.75). As displayed
% in Figure \ref{fig:comp} and previously noted by Budd et al. in \cite{budd1980mutation}, even as small
% as $10\%$ of the original set can produce mutation scores that are considered to be highly correlated
% ($\tau_b \approx 0.73$ for uniform random sampling and $\tau_b \approx 0.76$ for stratified random sampling over operators).
% The correlation between the mutation scores of both reduced sets from both approaches and the original set for all fractional values---using Kendall's $\tau_b$---can
% observed in Figure \ref{fig:comp_corr}.
%
% Inozemtseva and Holmes adopt the Guildford scale to describe correlation values. I will similarly
% describe correlations between sets' mutation score using the same scale.
% The Guildford scale describes the absolute value of a coefficient as ``low'' when it is less than 0.4,
% ``moderate'' when it is between 0.4 and 0.7, ``high'' when ranging from 0.7 to 0.9, and ``very high''
% when it is greater than 0.9 \cite{inozemtseva2014coverage}.
%
% Finally, when observing mean absolute and root mean squared errors displayed in Figure \ref{fig:comp_error},
% I noticed that both approaches experience a similar decrease in error as the reduced sets
% get larger---increase in fractional value. When analyzing the errors experienced for sets between
% $1 - 30\%$ of the original mutant set, both uniform random sampling and stratified random sampling over mutation operators
% encounter a rapid decrease in error, or increase in accuracy.
%
% After performing selective mutation analysis I have observed that all reduced sets from both
% uniform random sampling and stratified random sampling over mutation operators produce sets with mutation scores that are
% considered to be ``highly'' correlated, with the only exception
% being the reduced set produced by random sampling with $1\%$, which produces a reduced set with a mutation score that is considered
% to be weakly correlated to that of the mutation score of the entire set of mutations.
%
% Although we are able to produce
% reduced sets with highly correlated mutation scores, we are not able to produced sets with
% very highly correlated mutation scores using either of the selective approaches being analyzed.
%
% I chose only to provide the visualizations for the Inventory schema because it effectively displays
% the consequences that selective mutation analysis can have on the mutation score. Additionally,
% there are nine schemas that are mutation adequate, meaning that no matter how small the test data
% are, data for the corresponding schema will \textit{always} produce mutation adequate sets. This is what I consider
% to be an ineffective display of the cause of selective mutation analysis.
%
% \begin{figure*}[!ht]
% \centering
% \subfloat[Uniform random sampling errors]{\includegraphics[width=3in]{inventory_error}
% \label{fig:iv_e}}
% \hfil
% \subfloat[Stratified random sampling over mutation operators errors]{\includegraphics[width=3in]{inventory_op_error}
% \label{fig:iv_op_e}}
% \caption{The errors for uniform random sampling and stratified random sampling over operators for the Inventory schema.}
% \label{fig:comp_error}
% \end{figure*}

