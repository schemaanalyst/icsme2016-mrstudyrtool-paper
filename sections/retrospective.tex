% GMK NOTE: To save space, I think that we can cut all of this content and instead write one short paragraph that will
% go at the start of the next section of the paper

% \section{Retrospective Analysis}

% Performing mutation testing, displayed in Figure~\ref{fig:process}, is often considered to be too expensive to take
% advantage of in industry. While reducing the number of mutants using techniques such as mutant sampling improve
% efficiency, they do not reduce the cost enough to make mutation testing affordable. Additionally, integrating the
% various methods to reduce the number of mutants into a mutation testing system requires the programmer to understand the
% intricacies of the testing tool. Furthermore, after integrating a reduction technique into a testing tool, the major
% computational expense of performing mutation testing is incurred. Then, finally, the efficiency and effectiveness of a
% reduction technique can be assessed.

% By analysing mutant reducing methods in retrospect, the efficiency and effectiveness can be determined without
% understanding the intricacies of complex mutation testing tools or having to integrate every technique a tool. The
% ``trial-and-error'' method~\cite{edwards2004using} of evaluating reduction techniques is now performed on the data,
% eliminating the expenses and complexities of integrating and testing every approach. Moreover, the minimal upfront costs
% of retrospective analysis enable it to be utilized to provide more rapid feedback to a tester regarding the
% effectiveness of reduction techniques. Heretofore, no tool existed that was capable of analysing mutant reduction
% techniques retrospectively.

