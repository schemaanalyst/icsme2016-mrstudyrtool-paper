% \vspace*{-.2em}

\section{Conclusions and Future Work}

% \vspace*{-.25em}

% Re-introduce the challenges of mutation testing and summarize what this tool provides

Although mutation testing is well-recognized as a way to assess test suite quality, it may be too costly to practically
use. As such, various methods have been developed to decrease the cost of mutation testing. Performing these reduction
techniques in the past has required researchers and experimenters to incorporate a reduction method into an, often
complex, mutation testing tool. The \mr~tool alleviates the burden of implementing each approach by analyzing reduction
techniques in retrospect, a potentially more cost-effective method.

% Summarize some of the key features of the tool and then talk about how it is released and enables future work

By retrospectively analyzing the data collected from a prior analysis of all mutants, the \mr~tool is able to decrease
the upfront human-implementation costs by obviating the need for researchers and industrialists to fully understand the
domain complexities associated with a mutation testing system.  Furthermore, \mr~provides an easy-to-use and rapid way
to assess the efficiency and effectiveness of mutant reduction methods. In addition to being detailed in this paper,
\mr~has been released under an open-source license on a GitHub site that features extensive documentation and a
screencast~\cite{tool}. In future work, we plan to extend the functionality of \mr~by integrating additional mutant
reduction techniques, thereby allowing for a more comprehensive comparison of the techniques' efficiency and
effectiveness.

