\subsection{Inputs and Output of \mr}

Displayed in Figure~\ref{fig:mrstudyr}, \mr uses the data
from mutation testing to conduct further analyses of reduction techniques.

The data provided to \mr must minimally include mutant statuses (e.g., dead or alive),
mutant types (e.g., normal, duplicate, equivalent, or stillborn)
and operators. Moreover, data may include further detail about each mutant (e.g.,
when testing databases; the database management system, schema and generation cost of each mutant).

Based on the number of dead and alive mutants after mutation testing,
\mr calculates and associates a mutation score with a mutant set.
A mutant set's mutation score represents its ability to discover faults.
The mutation score for all mutants if calculated by \mr as a baseline
for determining each reduced sets' effectiveness.
