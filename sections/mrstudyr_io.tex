\subsection{Inputs and Output of \mr}
Displayed in Figure \ref{fig:mrstudyr}, \mr uses the data collected
from mutation testing to conduct further analyses
regarding the efficiency and effectiveness of a reduction technique.

The data provided to the \mr tool must minimally include mutant statuses (e.g., dead or alive),
mutant types (e.g., normal, duplicate, equivalent, or stillborn)
and operators. Moreover, the data can include futher detail about each mutant.
For example, when testing database schemas, the additional data may consist of
the database management system, schema and generation cost of each mutant.

Based on the number of dead and alive mutants after mutation testing,
the \mr tool calculates and associates a mutation score with a set of mutants.
A test suite's mutation score represents its ability to discover faults.
\mr calculates the mutation score for the entire set of mutants to be
used as a baseline for determining each reduced sets' effectiveness.
