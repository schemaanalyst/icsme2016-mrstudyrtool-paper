\subsection{Inputs and Output of \mr}

The data, at the very least, must  include the status of
a mutant after testing (e.g., dead or alive), the type of mutant (e.g., normal, duplicate, equivalent, or stillborn)
and the mutation operator used for each mutant. In addition to the minimum requirements, the data can include
more detail about each mutant. For example additional data that may be collected when performing mutation testing
on a database are the database
management system or schema under test and the cost of generating each mutant.

As indicated previously, the status of a mutant after testing must be provided
in the input data to the \mr tool.
Based on the number of dead and alive mutants after mutation testing,
the \mr tool calculates and associates a mutation score with a set of mutants.
The mutation score associated with a mutant set is a value representative of how well
the test suite is at identifying and killing the mutants included in that set.
The \mr tool first calculates the
mutation score for the non-reduced set of mutants and then later for each reduced set. The mutation
score for the non-reduced mutant set is used as a baseline for determining
each reduced set's performance.
