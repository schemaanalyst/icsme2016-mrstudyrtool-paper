\subsection{Retrospective Analysis with \mr}

The next phase in the mutation analysis process is retrospective analysis. In this
phase, mutants statuses are determine. Where a mutant is considered
``dead'' if when executing the modified version of the source
code the output differed from the original version. Conversely, a mutant remains alive
if the output from the modified version of the source code does not differ from the
original version. More importantly, in this phase of the mutation analysis process,
the efficiency and effectiveness of various reduction techniques are evaluated.

To avoid the cost of performing mutation multiple times, the \mr tool retrospectively studies
each reduction technique's efficiency and effectiveness from the data collected from a prior analysis of all mutants.
Displayed in Figure \ref{fig:mrstudyr},
the \mr tool uses the data collected from mutation testing to conduct further analyses
regarding the effectiveness of a reduction technique.
