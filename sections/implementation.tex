\section{Implementation}

The R programming language for statistical data analysis was specifically designed to
be a full statistical language for conducting empirical studies. Following its predecessor, the S programming
language, R is a powerful means for data analysts to express computations \cite{ihaka1996r}.
Additionally, in R, the fundamental way to share code is via a package.

An R package includes code, data, documentation, tests and are easy to distribute \cite{wickham2015r}.
Since the \mr tool has been released as an R package, four commands in the R console
is all that is needed to install the tool.
First, \texttt{install.packages("devtools")}, then \texttt{library(devtools)}
to install and load the \texttt{devtools}\cite{devtools} package, respectively.
The \texttt{devtools} package is necessary because it provides a framework for installing
and maintaining the tool. Finally, to obtain the
\mr\cite{mrstudyr} tool itself, use the \texttt{devtools} package to install it
from the popular Git repository hosting service, GitHub\cite{github},
using the following command: {\small\texttt{devtools::install\_github("mccurdyc/mrstudyr")}}
and then load it using \texttt{library(mrstudyr)}.
In addition to being easy to install, the \mr tool accepts a generalised input format making
it extensible to various domains.

Accepting a generalised input format allows the \mr tool to be used to retrospectively analyse mutation testing
data from familiar and also emerging domains, such as databases. Although mutation
testing can be utilized in various domains, similar data can be collected from any
domain (e.g., mutation operators, mutant statuses, type of mutant). While more data
can optionally be collected from mutation testing, the additional data is more
dependent on the domain under observation.

\subsection{Tool Design}

The \mr tool was designed to empirically study the effectiveness of
various reduction techniques. The two reduction techniques mentioned earlier---uniform random sampling
and uniform random sampling over operators---require an arbitrary percentage, $x$, to be chosen as the maximum threshold
for the number of mutants to be analysed.

Where Wong and Mathur in their studies \cite{mathur1994empirical}, \cite{wong1993mutation} conducted
an experiment using random sampling with $x$ from $10\%$ to $40\%$ increasing by steps of $5\%$, we
chose to start $x$ at $10\%$ and increase by intervals of $10\%$ up to $90\%$. By lowering
the granularity of the experiment to $10\%$ intervals instead of
$1\%$ or $5\%$, we are able to reduce the cost of performing the retrospective analysis while observing
similar trends.

While reducing the costs of performing the retrospective analysis were a primary concern, it does not
outweigh producing accurate results. Although we conducted our experiments with $x$ increasing by
$10\%$ intervals, we did not see a significant enough loss in accuracy and in the amount of data collected to make
it worth running the additional trials with the smaller intervals (i.e. $1\%$ or $5\%$).

When it came to determining the number of trials necessary, we followed
the recommendations of Traeger et al. and Arcuri, by running 30 trials for each configuration of a reduction approach \cite{traeger2008nine, arcuri2014hitchhiker}.
The general rule in many fields of science is to at least perform 30 observations to show with high confidence
that the obtained results are statistically significant and to allow for the results to approximate a normal
distribution.


\subsection{Tool Usage}

Releasing an R tool via a package makes installing and loading the tool
a matter of a few commands, but this does not guarantee that the usability
of the tool will be as simple. \mr~was designed to be simplistic,
yet perform stringent empirical analyses on mutant reduction techniques.
To display the results from \mr~analysing mutant reduction strategies for the
testing of real-world database schemas, the following commands will be tailored
toward the data that we collected from performing mutation testing.
The collected data can be read in via the following: {\small\texttt{data <- read\_data("sqlite-avmdefaults")}}.
This function expects the data to be located in the \texttt{inst/extdata} folder
and stored as a comma-separated value file.

Immediately following providing the data to \mr, analysis begins. Currently,
the two most common sub-techniques of mutant sampling are performed using the following:
\texttt{analyse\_random\_sampling(data)} and \texttt{analyse\_across\_operators(data)},
for performing random sampling and random sampling across operators, respectively.
These functions will both return a new dataframe containing the trial, evaluated
percentage, reduced and original numerator and denominator counts,
reduced and original time to generate mutants and the reduced and original mutation scores.
The output from these function is provided in the accompanying README file on the
tool's GitHub page~\cite{tool}. After performing a reduction technique its efficiency
and effectiveness is evaluated based on five calculation functions.

The five functions that evaluate the efficiency and effectiveness are:
mutation score, correlation coefficient, magnitude in reduction of creation cost
(\%) and error---MAE and RMSE---between the reduced and original mutation scores.
The calculation functions are as follows, in order:

\begin{itemize}
    \item {\small\texttt{analyse\_mutation\_score(data)}}
    \item {\small\texttt{analyse\_correlation(data)}}
    \item {\small\texttt{analyse\_reduction(data)}}
    \item {\small\texttt{analyse\_percents\_error(data)}}
\end{itemize}

These functions each return a dataframe including the result of the calculation.
Again, the output from the calculation functions can be found on
the tool's GitHub page~\cite{tool}. Following efficiency and effectiveness calculations
is human examination, only succeeded by the establishment of a policy recommendation.

The human examination phase still requires use of \mr~for clearly visualising the trends
in the data collected from the calculation phase. The \mr~tool takes advantage of the
widely-used graphing package, \texttt{ggplot2}~\cite{ggplot2}. All of the visualisation
functions provided and their output can be found on the GitHub page~\cite{tool}.
These visualisations will help human examiners to construct a policy recommendation as
to which reduction technique should be used for the specific domain and data under observation.


