\section{Implementation}
\subsection{Tool Design}
% For analyzing the existing selective approaches discussed previously, I have created
% a tool, \texttt{selectms}. This tool reads in data, expecting the data to contain
% specific attributes: identifier, dbms, schema, operator, type, killed, time. The identifier attribute
% is a universally unique set of characters describing the combination of database management system (dmbs)
% and schema. The dbms attribute is as the name suggests, a character value representing the database management system under observation.
% The schema and operator attributes, both character values, are the schema and operator under observation. The type attribute, a set of characters,
% refers to the type of mutant: equivalent, stillborn or normal. However, as previously mentioned, we filter out all non-normal mutants.
% A boolean value, killed, is either true or false, if the mutant is dead or alive, respectively. Finally, the time attribute is an
% integer value representing the time it takes---in milliseconds---to create the mutant.

$x\%$ random selection with $x$ increasing by 10 \\
% As previously mentioned, I am comparing uniform random sampling and stratified random sampling over mutation operators.
% The fractional values that I analyzed for both of the selective approaches are:
% $\{\frac{1}{100}, \frac{10}{100}, \frac{20}{100}, \frac{30}{100}, \frac{40}{100}, \frac{50}{100}, \frac{60}{100}, \frac{70}{100}, \frac{80}{100}, \frac{90}{100}\}$.
% I chose to increase by 10\% rather than analyzing each percentage due to the computation expense of the analysis if performed for each percentage.

continue discussing why we chose 10\% increments (refer to one paper from annotated bib)

30 trials
% We ran each possible configuration of fractional value for uniform random sampling and stratified random sampling
% for 30 trials as suggested by Traeger and Zadok to allow the results to approximate a normal distribution \cite{traeger2008nine}. Additionally,
% We kept the fractional values and number of trials for each configuration of sampling approach consistent throughout the analysis.

% \begin{figure*}[!hb]
% \centering
% \subfloat[Uniform random sampling correlation]{\includegraphics[width=2.5in]{rs_corr}
% \label{fig:iv_c}}
% \hfil
% \subfloat[Stratified random sampling over mutation operators correlation]{\includegraphics[width=2.5in]{op_corr}
% \label{fig:iv_op_c}}
% \caption{The correlation of mutation scores across percentages for the uniform random sampling and stratified random sampling
% across operators approaches, specifically for the Inventory schema.}
% \label{fig:comp_corr}
% \end{figure*}
\subsection{Tool Usage}
provide some command-line commands or R function calls
