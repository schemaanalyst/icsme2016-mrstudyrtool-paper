\section{Implementation}

The R programming language for statistical data analysis was specifically designed to
be a full statistical language for conducting empirical studies. Following its predecessor, the S programming
language, R is a powerful means for data analysts to express computations \cite{ihaka1996r}.
Additionally, in R, the fundamental way to share code is via a package.

An R package includes code, data, documentation, tests and are easy to distribute \cite{wickham2015r}.
Since the \mr tool has been released as and R package, three commands in the R console
is all that are need to install the tool.
First, \texttt{install.packages("devtools")}, then \texttt{library(devtools)}
to install and load the \texttt{devtools}\cite{devtools} package, respectively.
The \texttt{devtools} package is necessary because it provides a framework for installing
and maintaining the tool. Finally, to obtain the
\mr\hspace*{-1em}\cite{mrstudyr} tool, use the \texttt{devtools} package to install it
from the popular Git repository hosting service, GitHub\cite{github},
using the following command: {\small\texttt{devtools::install\_github("mccurdyc/mrstudyr")}}.
In addition to being easy to install, the \mr tool accepts a generalised input format making
it extensible to various domains.

Accepting a generalised input format allows the \mr tool to be used to retrospectively analyse mutation testing
data from familiar and also emerging domains, such as databases. Although mutation
testing can be utilized in various domains, similar data can be collected from any
domain (e.g., mutation operators, mutant statuses, type of mutant). While more data
can optionally be collected from mutation testing, the additional data is more
dependent on the domain under observation.

\subsection{Tool Design}

The \mr tool was designed for empirically studying the effects of various $x\%$ for
the two aforementioned reduction techniques in Mutant Sampling and Uniform Sampling Over Mutation Operators.
Where Wong and Mathur in their studies \cite{mathur1994empirical}, \cite{wong1993mutation} conducted
an experiment using random sampling with $x$ from $10\%$ to $40\%$ increasing by steps of $5\%$, we
chose to start $x$ at $10\%$ and increase by intervals $10\%$ up to $90\%$. By increasing by $10\%$ intervals instead of
by $1\%$ or $5\%$, we are able to reduce the cost of performing the retrospective analysis.

While reducing the costs of performing the retrospective analysis were a primary concern, it does not
outweigh producing accurate results. Although we conducted our experiments with $x$ increasing by
$10\%$ intervals, we did not see a significant enough loss in accuracy and in the amount of data collected to make
it worth running the additional trials for with the smaller intervals (i.e. $1\%$ or $5\%$).

When it came to determining the number of trials necessary for each configuration, we followed
the recommendations of Traeger et al. and Arcuri, by running 30 trials \cite{traeger2008nine, arcuri2014hitchhiker}.
The general rule in many fields of science is to at least perform 30 observations to show with high confidence
that the obtained results are statistically significant and to allow for the results to approximate a normal
distribution.


\subsection{Tool Usage}
provide some command-line commands or R function calls



