\section{Implementation}
%!TEX root=../icsme2016-mrstudyr.tex

\newcommand{\mx}[1]{\mathbf{\bm{#1}}} % Matrix command
\newcommand{\vc}[1]{\mathbf{\bm{#1}}} % Vector command

% Define the layers to draw the diagram
\pgfdeclarelayer{background}
\pgfdeclarelayer{foreground}
\pgfsetlayers{background,main,foreground}

% Define block styles used later

\tikzstyle{sensor}=[draw, fill=black!5, text width=15em,
text centered, minimum height=2.5em,drop shadow]
\tikzstyle{smallsensor}=[draw, fill=black!5, text width=4em,
text centered, minimum height=2.5em,drop shadow]
\tikzstyle{box}=[draw, text width=8em,
text centered, minimum width=17.5em, minimum height=3.5em]
\tikzstyle{calc}=[draw, fill=black!5, text width=2.5em,
text centered, rounded corners, minimum height=2.5em,drop shadow]
\tikzstyle{circle}=[draw, ellipse, fill=black!5, text width=11em,
text centered, minimum height=2.5em, drop shadow]
\tikzstyle{mr}=[draw, fill=black!20, text width=5em,
text centered, minimum height=17em, minimum width = 18em, drop shadow]

% Define distances for bordering
\def\blockdist{1.5}
\def\edgedist{2.5}

\begin{figure}[t]

  \vspace{-.75em}

  \centering
  % SIMPLE
  \begin{tikzpicture}[thick,scale=0.85, every node/.style={scale=0.85}]
    \node [mr] at (0, 0) (mr) {};
    \node [circle] at (0, 3.7) (md) {Original Data};
    \node [sensor] at (0, 2.25) (r) {Reduction Techniques};
    \node [circle] at (0, 0.75) (rmd) {Cumulated Reduced Data};
    \node [sensor] at (0, -0.75) (ra) {Efficiency \& Effectiveness Analysis};
    \node [box] at (0, -2.25) (box) {};
    \node [calc] at (-2.25, -2.25) (ms) {MS};
    \node [calc] at (-0.75, -2.25) (err) {Red.};
    \node [calc] at (0.75, -2.25) (corr) {Corr.};
    \node [calc] at (2.25, -2.25) (err) {Err.};
    \node [sensor] at (0, -3.75) (he) {Human Examination};
    \node [circle] at (0, -5.25) (pr) {Policy Recommendation};

    \path [draw, ->] (md.south) -- node [above] {}
      (r.90);
    \path [draw, ->] (r.south) -- node [above] {}
      (rmd.90);
    \path [draw, ->] (rmd.south) -- node [above] {}
      (ra.90);
    \path [draw, ->] (ra.south) --  node [above] {}
      (box.90);
    \path [draw, ->] (box.south) --  node [above] {}
      (he.90);
    \path [draw, ->] (he.south) -- node [above] {}
      (pr.90);
  \end{tikzpicture}

  \caption{\label{fig:mrstudyr}The inputs and outputs of the \mr~tool.}

  \captionpara{0.5}{0.9}{0.5}{In this figure, the dark square represents the \mr~tool and its constituent parts, a
    rectangle stands for a process, a rectangle with rounded edges is a calculation performed by \mr, and an ellipse
  symbolises a process output.}

  \vspace{-1.8em}
\end{figure}



need to talk about meaning of everything in diagram.
talk about how we get reduced mutant data.
talk about what $ms$ and $ms'$ mean and how they are calculated.
talk about what the calculation block represents
what is a policy recommendation?

\subsection{Tool Design}
% For analyzing the existing selective approaches discussed previously, I have created
% a tool, \texttt{selectms}. This tool reads in data, expecting the data to contain
% specific attributes: identifier, dbms, schema, operator, type, killed, time. The identifier attribute
% is a universally unique set of characters describing the combination of database management system (dmbs)
% and schema. The dbms attribute is as the name suggests, a character value representing the database management system under observation.
% The schema and operator attributes, both character values, are the schema and operator under observation. The type attribute, a set of characters,
% refers to the type of mutant: equivalent, stillborn or normal. However, as previously mentioned, we filter out all non-normal mutants.
% A boolean value, killed, is either true or false, if the mutant is dead or alive, respectively. Finally, the time attribute is an
% integer value representing the time it takes---in milliseconds---to create the mutant.

$x\%$ random selection with $x$ increasing by 10 \\
% As previously mentioned, I am comparing uniform random sampling and stratified random sampling over mutation operators.
% The fractional values that I analyzed for both of the selective approaches are:
% $\{\frac{1}{100}, \frac{10}{100}, \frac{20}{100}, \frac{30}{100}, \frac{40}{100}, \frac{50}{100}, \frac{60}{100}, \frac{70}{100}, \frac{80}{100}, \frac{90}{100}\}$.
% I chose to increase by 10\% rather than analyzing each percentage due to the computation expense of the analysis if performed for each percentage.

continue discussing why we chose 10\% increments (refer to one paper from annotated bib)

30 trials
% We ran each possible configuration of fractional value for uniform random sampling and stratified random sampling
% for 30 trials as suggested by Traeger and Zadok to allow the results to approximate a normal distribution \cite{traeger2008nine}. Additionally,
% We kept the fractional values and number of trials for each configuration of sampling approach consistent throughout the analysis.

% \begin{figure*}[!hb]
% \centering
% \subfloat[Uniform random sampling correlation]{\includegraphics[width=2.5in]{rs_corr}
% \label{fig:iv_c}}
% \hfil
% \subfloat[Stratified random sampling over mutation operators correlation]{\includegraphics[width=2.5in]{op_corr}
% \label{fig:iv_op_c}}
% \caption{The correlation of mutation scores across percentages for the uniform random sampling and stratified random sampling
% across operators approaches, specifically for the Inventory schema.}
% \label{fig:comp_corr}
% \end{figure*}
\subsection{Tool Usage}
provide some command-line commands or R function calls

%!TEX root=../icsme2016-mrstudyr.tex

% Define the layers to draw the diagram
\pgfdeclarelayer{background}
\pgfdeclarelayer{foreground}
\pgfsetlayers{background,main,foreground}

% Define block styles used later

\tikzstyle{sensor}=[text width=5em,
    text centered, minimum height=2.5em]
\tikzstyle{darksensor}=[draw, fill=black!10, text width=6em,
    text centered, minimum height=2.5em,drop shadow]

% Define distances for bordering
\def\blockdist{1.5}
\def\edgedist{2.5}

\begin{figure}[t]
  \centering
\begin{tikzpicture}[thick,scale=0.85, every node/.style={scale=0.85}]
    \node [] at (0, 5) (map) {\textbf{Mutation Analysis Process}};
    \draw (-4, 4.75) -- (4, 4.75);
    \node [sensor] at (0, 4) (m1) {Generate Mutants};
    \node [sensor] at (0, 2.5) (m2) {Execute Generated Mutants};
    \node [sensor] at (3.25, 3.25) (t1) {Mutation Testing};
    \node [sensor] at (3.25, 1) (t2) {Retrospective Analysis};
    \draw (2, 4.4) -- (2, 1.9);
    \draw (1.7, 4.4) -- (2, 4.4);
    \draw (1.7, 1.9) -- (2, 1.9);
    \node [darksensor] at (0, 1) (m3) {Retrospective Analysis};
    \draw (2, 1.7) -- (2, 0.25);
    \draw (1.7, 1.7) -- (2, 1.7);
    \draw (1.7, 0.25) -- (2, 0.25);

    \path [draw, ->] (m1.south) -- node [above] {}
        (m2.90);
    \path [draw, ->] (m2.south) -- node [above] {}
        (m3.90);
    % \path [draw] (m1.east) -| +(0.3,0) |-node [above] {}
    %     (m2.0);
\end{tikzpicture}
\caption{\label{fig:process}The phases of the mutation analysis process.}
\vspace{-1.5em}
\end{figure}

