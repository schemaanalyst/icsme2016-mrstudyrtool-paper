\section{Implementation}

The R programming language for statistical data analysis was specifically designed to
be a full statistical language for conducting empirical studies. Following its predecessor, the S programming
language, R is a powerful means for data analysts to express computations \cite{ihaka1996r}.
Additionally, in R, the fundamental way to share code is via a package.

An R package includes code, data, documentation, tests and are easy to distribute \cite{wickham2015r}.
Since the \mr tool has been released as and R package, three commands in the R console
is all that are need to install the tool.
First, \texttt{install.packages("devtools")}, then \texttt{library(devtools)}
to install and load the \texttt{devtools}\cite{devtools} package, respectively.
The \texttt{devtools} package is necessary because it provides a framework for installing
and maintaining the tool. Finally, to obtain the
\mr\hspace*{-1em}\cite{mrstudyr} tool, use the \texttt{devtools} package to install it
from the popular Git repository hosting service, GitHub\cite{github},
using the following command: {\small\texttt{devtools::install\_github("mccurdyc/mrstudyr")}}.
In addition to being easy to install, the \mr tool accepts a generalised input format making
it extensible to various domains.

Accepting a generalised input format allows the \mr tool to be used to retrospectively analyse mutation testing
data from familiar and also emerging domains, such as databases. Although mutation
testing can be utilized in various domains, similar data can be collected from any
domain (e.g., mutation operators, mutant statuses, type of mutant). While more data
can optionally be collected from mutation testing, the additional data is more
dependent on the domain under observation.

\subsection{Conducting Experiment Campaigns}

% The \mr~tool accepts a generalised input format enabling it to be used to retrospectively analyse mutation testing data
% from programs and also emerging domains such as databases.  Although domains under test may differ, like data can be
% collected (e.g., mutant status, mutant type and operator). While more data can optionally be collected from mutation
% testing, the additional data is more dependent on the domain under test. Data collected from performing mutation testing
% on the nine database schemas is displayed in Table~\ref{tbl:study-data}.  This is a snippet of the data used in the
% preliminary study of effectiveness of \mr.  The additional data are unique identifier for the database management system
% / schema combination (Identifier), DBMS (DBMS), schema (Schema) and mutant creation time (Time).

% The general rule in many fields of science is to minimally perform 30 observations to show with high confidence that the
% results are statistically significant and to allow for the results to approximate a normal distribution.

% The structure of the experiment function for the two reduction sub-techniques is displayed in

% Explain how to conduct a compaign of experiments using the figure with an algorithm's psuedo code
% TODO: Fix the fact that the last sentence in this paragraph does not take up enough space

% Experiment Structure Diagram: This diagram in its current state is pretty specific to our experiment and testing database
% schemas. As you will notice, it iterates over a list of schemas. Additionally, it is specific to mutant sampling
% with the diagram showing that a technique iterates over a list of percentages. We can keep the diagram the way it
% is or remove it. I think by making it generalised (removing list of schemas, and percentages) it reduces it to the
% point where it isn't worth showing because it will then consist of iterate through a number of trials, perform a reduction
% technique, return the data collected from every trial. If it were simplified, I think it would just be easier to
% describe the technique it writing rather than taking up so much space with a diagram.

% I know in GH we talked about objective-level discussion then increase granularity. With that in mind, this first
% sentence might be too specific for this section and may need to move to a section specific to the results displayed
% in this paper. Or, going back to the previous comment, we could remove the 'iterate over schemas' from the diagram
% and say that this is the structure for the campaign of experiments for mutant sampling with mrstudyr (I will do this!)

% CJM: This paragraph will need major revisions now with the updated experiment structure diagram
Figure~\ref{fig:experiment-structure} gives the structure for the campaign of experiments that \mr~conducts to collect
the reduced mutant data from performing various reduction techniques. The ``analyse'' algorithm expects the mutant data from performing
mutation testing on all mutants. Since the presented tool currently focuses on mutant
sampling, the \textit{reduction technique} algorithm requires as input an arbitrary $x$, to be chosen as the maximum fractional threshold for the
number of mutants to be analysed from a set. Following the recommendations of Traeger~\etal~and Arcuri and Briand,
for the preliminary results presented in this paper, the maximum threshold for the number of trials that \mr~runs for each configuration of a reduction approach is set to 30~\cite{traeger2008nine, arcuri2014hitchhiker}, thereby
controlling for the randomness inherent in a reduction method's behavior and execution time.

\subsection{Implementation as an R Package}

The R programming language for statistical data analysis was specifically designed to be a full statistical language for
conducting empirical studies. Following its predecessor, R is a powerful means for data analysts to express
computations~\cite{ihaka1996r}.  In R, the fundamental way to share code is via a package.  R packages are easy to
distribute and include code, data, documentation and tests~\cite{wickham2015r}.  Since \mr~has been released as an R
package, installing the tool requires four commands in the R console.  First, \texttt{install.packages("devtools")},
then {\small\texttt{library(devtools)}} to install and load the {\small\texttt{devtools}}~\cite{devtools} package,
respectively.  The \texttt{devtools} package is necessary because it provides functions for installing and maintaining
\mr~as well as its dependencies.  Finally, to install \mr~\cite{tool} from the popular Git repository hosting service,
GitHub~\cite{github}, use the following command: {\small\texttt{devtools::install\_github("mccurdyc/mrstudyr")}}.  Then,
finally, load \mr~using \texttt{library(mrstudyr)}.

%!TEX root=../icsme2016-mrstudyr.tex

\begin{figure}
\centering

\vspace{-.75em}

\scalebox{\algorithmfigurescalefactor}{
\begin{tabular}{|l|}
\hline

% TODO: Review this algorithm and fix all of the other issues pointed out in the marked version of the paper
% --> How do you initialise d?
% --> How is the variable p incorporated?
% --> Does this algorithm closely follow the structure of the R code?
% --> Does this algorithm use consistent terminology for the variable x?

\begin{tabular}{l}
\vspace{-2ex} \\

{\bf function} analyse ($data$) \\

\atab {\bf where} \\
\atab \atab \tinybullet $\mbox{{\em data}}$ is data collected from performing mutation testing on all mutants \\
\atab \atab \tinybullet $x$ is the maximum fractional threshold for the number of mutants \\
\atab \atab \tinybullet {\em percentages} is the list of chosen $x$ values for mutant sampling \\
\atab \atab \tinybullet $t$ is the trial index of the experiment \\
\atab \atab \tinybullet {\em trials} is the total number of trials set for each configuration of technique \\
\atab \atab \tinybullet {\em red} is the reduced mutant data returned from a single reduction method \\
\atab \atab \tinybullet $d$ is the data from the configuration of a reduction method and $x$ value\\

\vspace*{-.75em} \\

% \hspace{1em} \atab {\bf initialise} $d$ \smallvspace \\

\hspace{1em} \atab $d \leftarrow \emptyset$, $t \leftarrow 0$ \smallvspace \\

% \hspace{1em} \atab $t \leftarrow 0$ \smallvspace \\

\hspace{1em} \atab {\bf for each} $x \in \mbox{{\em percentages}}$ \smallvspace \\

\hspace{3em} \atab {\bf for} $t \leq \mbox{{\em trials}}$ \smallvspace \\

\hspace{4em} \atab {\bf let} $red \leftarrow \mbox{{\em reduction\,technique}}\,(\mbox{{\em data}},\,p)$ \smallvspace \\

\hspace{4em} \atab {\bf let} $d \leftarrow$ {\bf append}($d$, $\mbox{{\em red}}$) \smallvspace \\

\hspace{4em} \atab $t \leftarrow t + 1$ \smallvspace \\

\hspace{1em} \atab {\bf return} $d$ \smallvspace \\

{\bf end function} \\
\vspace{-0.5em} \\

\end{tabular} \\

\hline
\end{tabular}}

\caption{\label{fig:experiment-structure} An experiment for studying mutant sampling.}

\vspace{-1em}

\end{figure}




\subsection{Tool Usage}

% CJM: Moving this figure here to allow reader to follow the process of the mrstudr tool

%!TEX root=../icsme2016-mrstudyr.tex

\newcommand{\mx}[1]{\mathbf{\bm{#1}}} % Matrix command
\newcommand{\vc}[1]{\mathbf{\bm{#1}}} % Vector command

% Define the layers to draw the diagram
\pgfdeclarelayer{background}
\pgfdeclarelayer{foreground}
\pgfsetlayers{background,main,foreground}

% Define block styles used later

\tikzstyle{sensor}=[draw, fill=black!5, text width=15em,
text centered, minimum height=2.5em,drop shadow]
\tikzstyle{smallsensor}=[draw, fill=black!5, text width=4em,
text centered, minimum height=2.5em,drop shadow]
\tikzstyle{box}=[draw, text width=8em,
text centered, minimum width=17.5em, minimum height=3.5em]
\tikzstyle{calc}=[draw, fill=black!5, text width=2.5em,
text centered, rounded corners, minimum height=2.5em,drop shadow]
\tikzstyle{circle}=[draw, ellipse, fill=black!5, text width=11em,
text centered, minimum height=2.5em, drop shadow]
\tikzstyle{mr}=[draw, fill=black!20, text width=5em,
text centered, minimum height=17em, minimum width = 18em, drop shadow]

% Define distances for bordering
\def\blockdist{1.5}
\def\edgedist{2.5}

\begin{figure}[t]

  \vspace{-.75em}

  \centering
  % SIMPLE
  \begin{tikzpicture}[thick,scale=0.85, every node/.style={scale=0.85}]
    \node [mr] at (0, 0) (mr) {};
    \node [circle] at (0, 3.7) (md) {Original Data};
    \node [sensor] at (0, 2.25) (r) {Reduction Techniques};
    \node [circle] at (0, 0.75) (rmd) {Cumulated Reduced Data};
    \node [sensor] at (0, -0.75) (ra) {Efficiency \& Effectiveness Analysis};
    \node [box] at (0, -2.25) (box) {};
    \node [calc] at (-2.25, -2.25) (ms) {MS};
    \node [calc] at (-0.75, -2.25) (err) {Red.};
    \node [calc] at (0.75, -2.25) (corr) {Corr.};
    \node [calc] at (2.25, -2.25) (err) {Err.};
    \node [sensor] at (0, -3.75) (he) {Human Examination};
    \node [circle] at (0, -5.25) (pr) {Policy Recommendation};

    \path [draw, ->] (md.south) -- node [above] {}
      (r.90);
    \path [draw, ->] (r.south) -- node [above] {}
      (rmd.90);
    \path [draw, ->] (rmd.south) -- node [above] {}
      (ra.90);
    \path [draw, ->] (ra.south) --  node [above] {}
      (box.90);
    \path [draw, ->] (box.south) --  node [above] {}
      (he.90);
    \path [draw, ->] (he.south) -- node [above] {}
      (pr.90);
  \end{tikzpicture}

  \caption{\label{fig:mrstudyr}The inputs and outputs of the \mr~tool.}

  \captionpara{0.5}{0.9}{0.5}{In this figure, the dark square represents the \mr~tool and its constituent parts, a
    rectangle stands for a process, a rectangle with rounded edges is a calculation performed by \mr, and an ellipse
  symbolises a process output.}

  \vspace{-1.8em}
\end{figure}



% CJM: I think we should move 'data' to the same line as the function call

Releasing an R tool via a package makes installing and loading the tool a matter of a few commands,
but does not guarantee tool usability. \mr~was designed to be simplistic, yet perform stringent
empirical analyses on mutant reduction techniques. To display the results from \mr~analysing mutant
reduction strategies for the testing of real-world database schemas, the following commands will be
tailored toward the data that we collected from performing mutation testing. The collected data can
be read in via the following: {\small\texttt{data <- read\_data("sqlite-avmdefaults.dat")}}. This
function expects the data to be located in the \texttt{inst/extdata} folder and stored as a
comma-separated value file.

% CJM: I noticed in section 2.C (Conducting Experiment Campaigns) that it already explains that mrstudyr
% focuses on mutant sampling. Therefore, I am going to remove the sentences in this section restating
% this fact. These sentences are commented-out immediately follow this comment.

% Currently, the two most common sub-techniques of mutant sampling are performed using the following:
% \texttt{analyse\_random\_sampling(data)} and \texttt{analyse\_across\_operators(data)},
% for random sampling and sampling across operators, respectively.

% CJM: Additionally, the attributes of the data that I describe are specific to mutant sampling. I will
% remove these attributes.

% CJM: Is including the following statement entirely necessary? I removed it for now.

% Sample output from the \texttt{analyse} function is provided in the accompanying README file on the
% tool's GitHub page~\cite{tool}.

The reduction techniques, reffered to in Figure~\ref{fig:mrstudyr}, are performed following the provision
of mutant data collected from mutation testing --- the ``Original Data'' --- to \mr. Using \mr~to perform
reduction techniques is a single command: \texttt{analyse(data)}. Each reduction technique performed by
the \texttt{analyse} function in \mr, returns reduced data which is then cumulated into a data set, shown
in Figure~\ref{fig:mrstudyr}, containing the data from all of the reduction approaches. This data includes
the trial, the total number of mutants analysed, the number of killed mutants, the generation time of each
mutant and the mutation scores for the reduced and original data. After performing the reduction techniques,
their efficiency and effectiveness is evaluated based on four calculations on the cumulated reduced data in
the ``E \& E Analysis'' phase.

% CJM: I want to present the evaluation metrics of efficiency and effectiveness, then define them, then finally,
% how to perform them with mrstudyr.
The evaluation calculations in E \& E analysis --- and their respective abreviations --- are: mutation score
(MS), reduction in creation cost (Red.), correlation (Corr.), and error (Err.). The mutation score is calculated
by dividing the number of killed mutants by the total number of mutants~\cite{wong1995reducing}. The reduction
in creation cost for a set of mutants is the cost of the reduced set subtracted from the cost of the original set,
divided by the cost of the original set. The original and reduced sets' mutation score correlation is calculated
using Kendall's~\taub, supported by the ``Kendall'' R package~\cite{mcleod2015kendall}. Finally, the error --- both
MAE and RMSE --- between the original and reduced sets' mutation score is calculated using the existing ``Metrics''
R package~\cite{metrics}.

The aforementioned metrics, with the exception of calculating correlation, are calculated by a single function
in \mr~: \texttt{analyse\_calculations(data)}. Where the input to \texttt{analyse\_calculations} is the cumulated
reduced data, displayed in Figure~\ref{fig:mrstudyr}, from performing the reduction techniques and returns a data
set with the values of the respective calculation. To calculate correlation, the \texttt{analyse\_correlation(data)}
function is used. Where the input data to this function is again the cumulated reduced data and the output is a
data set containing only the correlation values between the reduced and original mutation scores. The usability of
\mr~is supported by requiring experimenters to only call two functions to analyse the efficiency and effective of
reduction techniques.

% CJM: removed this paragraph because I no longer think that it is needed and to save space for preliminary study
% Additionally, the \texttt{analyse\_calculations} function can be used to calculate the
% efficiency and effectiveness for any number of reduction approaches; it is not structurally
% limited to a single reduction method. As input, the \texttt{analyse\_calculations} function
% expects the dataframe returned from performing a reduction approach.

The ``Human Examination'' phase, displayed in Figure~\ref{fig:mrstudyr}, still requires use of \mr~for clearly
visualising the trends in the data collected from the E \& E analysis. The \mr~tool takes advantage of the work
of Hadley Wickham with widely-used graphing package \texttt{ggplot2}~\cite{ggplot2}. All of the visualisation
functions provided by \mr~and their output can be found on the GitHub page~\cite{tool}. These visualisations
will help human examiners to construct a policy recommendation as to which reduction technique should be used
for the specific domain and data under observation.


