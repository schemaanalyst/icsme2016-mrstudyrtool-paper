\newcommand{\mx}[1]{\mathbf{\bm{#1}}} % Matrix command
\newcommand{\vc}[1]{\mathbf{\bm{#1}}} % Vector command

% Define the layers to draw the diagram
\pgfdeclarelayer{background}
\pgfdeclarelayer{foreground}
\pgfsetlayers{background,main,foreground}

% Define block styles used later

\tikzstyle{sensor}=[draw, fill=black!10, text width=5em,
    text centered, minimum height=2.5em,drop shadow]
\tikzstyle{small}=[draw, fill=black!10, text width=2em,
    text centered, minimum width=2em, minimum height=2.5em,drop shadow]
\tikzstyle{smallcir}=[draw, ellipse, fill=black!10, text width=5em,
    text centered, minimum height=2.5em, drop shadow]
\tikzstyle{circle}=[draw, ellipse, fill=black!10, text width=8em,
    text centered, minimum height=2.5em, drop shadow]
\tikzstyle{func}=[draw, fill=black!20, text width=5em,
    text centered, minimum height=2.5em,drop shadow]
\tikzstyle{mr}=[draw, fill=black!20, text width=5em,
    text centered, minimum height=13em, minimum width = 15em, drop shadow]
\tikzstyle{ann} = [above, text width=5em, text centered]
\tikzstyle{wa} = [sensor, text width=10em, fill=black!30,
    minimum height=6em, rounded corners, drop shadow]
\tikzstyle{sc} = [sensor, text width=13em, fill=red!30,
    minimum height=10em, rounded corners, drop shadow]


% Define distances for bordering
\def\blockdist{1.5}
\def\edgedist{2.5}

\begin{figure}[t]
  \centering
% SIMPLE
\begin{tikzpicture}[thick,scale=0.85, every node/.style={scale=0.85}]
    \node [mr] at (0, 0) (mr) {};
    \node [smallcir] at (0, 3.25) (md) {Mutant Data};
    \node [sensor] at (0, 1.5) (r) {Reduction Approach};
    \node [sensor] at (0, 0) (rmd) {Reduced Mutant Data};
    \node [small] at (-1.75, 0) (m1) {$ms'$};
    \node [small] at (1.75, 0) (m2) {$ms$};
    \node [func] at (0, -1.5) (calc) {Calculations};
    \node [circle] at (0, -3.25) (rec) {Policy Recommendation};

    \path [draw, ->] (md.south) -- node [above] {}
        (r.90);
    \path [draw, ->] (md.east) -| node [above] {}
        (m2.90);
    \path [draw, ->] (r.south) -- node [above] {}
        (rmd.90);
    \path [draw, ->] (rmd.west) -- node [above] {}
        (m1.0);
    \path [draw, ->] (m1.south) |- node [above] {}
        (calc.180);
    \path [draw, ->] (m2.south) |- node [above] {}
        (calc.0);
    \path [draw, ->] (calc.south) -- node [above] {}
        (rec.90);

% VERTICAL
% \begin{tikzpicture}[thick,scale=0.85, every node/.style={scale=0.85}]
%     \node [mr] at (0, 0) (mr) {};
%     % \node [] at (-3, -0.25) (left) {};
%     % \node [] at (0, 2.3) (text) {\textit{mrstudyr}};
%     \node [sensor] at (-1.25, 0) (g) {};
%     \node [sensor] at (1.25, 0) (gg) {};
%     \node [circle] at (0, 4.5) (md) {Mutant Data};
%     \node [circle] at (0, 2.75) (df) {Dataframe};
%     \node [] at (-1.25, 1.75) (a1) {$a_1$};
%     \node [] at (0, 1.75) (dots) {...};
%     \node [] at (1.25, 1.75) (a2) {$a_n$};
%     \node [] at (-1.75, 1) (t1) {$t_1$};
%     \node [] at (-1.25, 1) (dots2) {...};
%     \node [] at (-0.75, 1) (t30) {$t_{30}$};
%     \node [] at (0.75, 1) (tt1) {$t_1$};
%     \node [] at (1.25, 1) (dots3) {...};
%     \node [] at (1.75, 1) (tt30) {$t_{30}$};
%     \node [] at (-1.75, 0) (m1) {$ms_{1}'$};
%     \node [] at (-0.75, 0) (m2) {$ms_{30}'$};
%     \node [] at (0.75, 0) (mm1) {$ms_{1}'$};
%     \node [] at (1.75, 0) (mm2) {$ms_{30}'$};
%     \node [func] at (-1.75, -1.75) (corr) {Calculate Correlation};
%     \node [func] at (1.75, -1.75) (err) {Calculate Error};
%     \node [circle] at (0, -2.8) (df2) {Dataframe};
%
%     \path [draw, ->] (md.south) -- node [above] {}
%         (df.90);
%     \path [draw, ->] (df.south) -- node [above] {}
%         (a1.10);
%     \path [draw, ->] (df.south) -- node [above] {}
%         (a2.170);
%     \path [draw, ->] (a1.south) -- node [above] {}
%         (t1.50);
%     \path [draw, ->] (a1.south) -- node [above] {}
%         (t30.130);
%     \path [draw, ->] (a2.south) -- node [above] {}
%         (tt1.50);
%     \path [draw, ->] (a2.south) -- node [above] {}
%         (tt30.130);
%     \path [draw, ->] (t1.south) -- node [above] {}
%         (m1.90);
%     \path [draw, ->] (t30.south) -- node [above] {}
%         (m2.90);
%     \path [draw, ->] (tt1.south) -- node [above] {}
%         (mm1.90);
%     \path [draw, ->] (tt30.south) -- node [above] {}
%         (mm2.90);
%     \path [draw, ->] (g.south) -| +(0, -0.25) -| node [above] {}
%         (err.90);
%     \path [draw, ->] (gg.south) -| +(0, -0.25) -| node [above] {}
%         (corr.90);
%
\end{tikzpicture}
\caption{\label{fig:mrstudyr}The inputs and outputs of mutation analysis phase
of the mutation analysis process, performed by the \mr \,tool. \\
\textit{The darkest square in the background represents the \mr \,tool.
A rectangle represents a process. A rectangle with rounded edges represents a
calculation performed by \mr. An ellipse represents the output from a process.}}
\end{figure}


\section{\textit{mrstudyr}: Mutant Reduction Studier}

Performing the entire mutation analysis process, displayed in Figure \ref{fig:process},
is both expensive in terms of time and computational requirements. The mutation testing
phase, displayed in Figure \ref{fig:process}, of the mutation analysis
process consists of generating and executing the mutants. The primary
expense of the mutation analysis process is incurred in the mutation testing phase
due to the large number of mutants generated---even for programs with few lines of code \cite{offutt2001mutation}.

The next phase in the mutation analysis process is retrospective analysis. In this
phase, mutants are found to either be dead or alive. Where a mutant is considered
``dead''---or to have been killed---if when executing the modified version of the source
code the output differed from the original version. Conversely, a mutant remains alive
if the output from the modified version of the source code does not differ from the
original version. More importantly, in this phase of the mutation analysis process,
we are able to evaluate the effectiveness of various reduction techniques.

By retrospectively analysing each reduction technique's effectiveness, we are
able to avoid the cost of performing mutation testing more than once. To alleviate
the additional executions, data needs to be collected
from the first run. Displayed in Figure \ref{fig:mrstudyr},
the \mr tool uses the data collected from mutation testing to conduct further analyses
regarding the effectiveness of a reduction technique.

The data, at the very least, needs to include the status of
a mutant after testing (e.g., dead or alive), the type of mutant (e.g., normal, duplicate, equivalent, or stillborn)
and the operator used. In addition to the minimum requirements, the data can include much
more detail about the mutants. An example of additional data that may be collected are the database
management system or schema under test and the cost of generating each mutant.

As indicated previously, the \mr tool requires the status of a mutant after testing to be included
in the data provided.
Based on the number of dead and alive mutants after mutation testing,
the \mr tool calculates and associates a mutation score with a set of mutants.
The mutation score of a mutant set is a value representative of how well
a test suite is at identifying and killing mutants. The \mr tool calculates the
mutation scores for the non-reduced set of mutants and then later for each reduced set
to determine how well a reduced set represents the original set in terms of ability to kill mutants.

In order to calculate the effectiveness of a reduction technique, the technique
must first be exercised on the data collected from mutation testing. The \mr tool
performs the reduction techniques and returns the new, reduced mutant data,
displayed in Figure \ref{fig:mrstudyr}. The effectiveness---how well a reduced set is able to represent
the non-reduced set---of a reduction technique is then evaluated
based on four metrics: mutation score, the correlation of the reduced and non-reduced sets' mutation scores
and the mean absolute and root mean squared error.

DO I FURTHER DISCUSS THESE METRICS HERE?? \\
I think so.

\subsection{Released as a Standardised R Package}
\textit{note: do not inappropriately include too much detail}
talk about why we chose to use R (hitchhiker's guide paper / SBST2016)\\
talk about the importance of free and open-source software (Regression testing workshop paper)\\
talk about why reproducibility is important (Regression testing workshop paper)


\subsection{Generalised Input Format}
not limited to a single domain / extensible
allows for use in old and new domains with little modification
    \subsubsection{application to databases}
    talk about mutation analysis of database schema mutants
    \subsubsection{application to programs}

