%!TEX root=../icsme2016-mrstudyr.tex

% Numbers that should be presented in a consistent fashion
\newcommand{\numquestions}{971,373}

\newcommand{\mannwhitney}{Mann---Whitney {\it U\/} test\xspace}
\newcommand{\wilcoxon}{Wilcoxon rank-sum test\xspace}
\newcommand{\pvalue}{{\it p\/}-value\xspace}
\newcommand{\bfpvalue}{\textbf{\textit{p}-value}\xspace}
\newcommand{\atwelve}{\^{A}\textsubscript{12}\xspace}
\newcommand{\bfatwelve}{\textbf{\textit{\^{A}}}$\mathbf{_{12}}$\xspace}
\newcommand{\taub}{{\large$\tau_{b}$}}

\newcommand{\etal}{et al.\xspace}

% Textual shortcuts
\newcommand{\mr}{{\it mrstudyr\/}\xspace}
\newcommand{\MR}{\mr}
\newcommand{\mrstudyr}{\mr}

% Floats
\newcommand{\Figure}[1]{Figure~\ref{#1}\xspace}
\newcommand{\Section}[1]{Section~\ref{#1}\xspace}
\newcommand{\Table}[1]{Table~\ref{#1}\xspace}

%% To help ensure consistent float scaling
\def \algorithmfigurescalefactor{0.7}
\def \tablescalefactor{0.85}
\def \inlinescalefactor{0.82}

%% Rotated table column type
\newcolumntype{R}[2]{%
    >{\adjustbox{minipage=6em,angle=#1,lap=\width-(#2)}\bgroup}%
    l%
    <{\egroup}%
}
\newcommand*\rot{\multicolumn{1}{R{40}{1.5em}}}

%% Highlight command for coloured highlight in listings
\newcommand{\reducedstrut}{\vrule width 0pt height \ht\strutbox depth \dp\strutbox\relax}
\newcommand{\highlight}[1]{%
  \begingroup
  \setlength{\fboxsep}{0pt}%
  \colorbox{red!20}{\reducedstrut#1\/}%
  \endgroup
}

% Equations
\newcommand{\relnull}{\sql{NULL}}
\newcommand{\eqnull}{= \relnull}
\newcommand{\existsinrel}[2]{\exists \inrel{#1}{#2}}
\newcommand{\forallinrel}[2]{\forall \inrel{#1}{#2}}
\newcommand{\inrel}[2]{#1 \in \: #2}
\newcommand{\neqnull}{\neq \relnull}
\newcommand{\select}[2]{#1 (#2)}
\newcommand{\selecteqnull}[2]{\select{#1}{#2} \eqnull}
\newcommand{\selectneqnull}[2]{\select{#1}{#2} \neqnull}
\newcommand{\selectneq}[3]{\select{#1}{#3} \: \neq \: \select{#2}{#3}}
\newcommand{\selecteq}[3]{\select{#1}{#3} \: = \: \select{#2}{#3}}
\newcommand{\selecteqexp}[4]{\select{#1}{#3} \: = \: \select{#2}{#4}}
\newcommand{\smallwedge}{\: \wedge \:}
\newcommand{\smallvee}{\: \vee \:}

\let\oldemptyset\emptyset
\let\emptyset\varnothing

% Algorithms in figures
\newcommand{\tinybullet}{$\vcenter{\hbox{\tiny$\bullet$}} \;$}
\newcommand{\atab}{\hspace{3ex}}
\newcommand{\wheretab}{\hspace{2ex}}
\newcommand{\exptab}{\hspace{2.5ex}}
\newcommand{\smallvspace}{\vspace{0.5ex}}
\newcommand{\closeblock}{\vspace{1ex}}
\newcommand{\closemajorblock}{\vspace{2ex}}

% Comments
\newcommand{\rem}[1]{}

% RQs
\newcommand{\conclusion}[2]{\vspace{1mm} \noindent {\bf Conclusion for #1:} #2}

% Caption Paragraphs
\newcommand{\captionpara}[4]{
    \vspace{#1em}
    \makebox[\linewidth][c]{
        \begin{minipage}{#2\linewidth}
      \scriptsize
            #4
        \end{minipage}
    }
    \vspace{#3em}
}

\newcommand{\sa}{{\it SchemaAnalyst\/}\xspace}
\newcommand{\SA}{\sa}
\newcommand{\SchemaAnalyst}{\sa}
\newcommand{\schemaanalyst}{\sa}

\newcommand{\boxplotstats}{%
  Each box represents the inter-quartile range (IQR), or the measure of statistical dispersion that is the difference between
  the 1\textsuperscript{st} and 3\textsuperscript{rd} quartiles. In this plot the whiskers extend up to $1.5$
  times the IQR and the line across the middle of the box marks the median. Additionally, the triangle in
  the boxes denotes the mean and the filled circles extending beyond the whiskers correspond to outliers.}

