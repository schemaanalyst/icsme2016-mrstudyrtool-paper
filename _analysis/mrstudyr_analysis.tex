\documentclass[]{article}
\usepackage{lmodern}
\usepackage{amssymb,amsmath}
\usepackage{ifxetex,ifluatex}
\usepackage{fixltx2e} % provides \textsubscript
\ifnum 0\ifxetex 1\fi\ifluatex 1\fi=0 % if pdftex
  \usepackage[T1]{fontenc}
  \usepackage[utf8]{inputenc}
\else % if luatex or xelatex
  \ifxetex
    \usepackage{mathspec}
  \else
    \usepackage{fontspec}
  \fi
  \defaultfontfeatures{Ligatures=TeX,Scale=MatchLowercase}
\fi
% use upquote if available, for straight quotes in verbatim environments
\IfFileExists{upquote.sty}{\usepackage{upquote}}{}
% use microtype if available
\IfFileExists{microtype.sty}{%
\usepackage{microtype}
\UseMicrotypeSet[protrusion]{basicmath} % disable protrusion for tt fonts
}{}
\usepackage[margin=1in]{geometry}
\usepackage{hyperref}
\hypersetup{unicode=true,
            pdftitle={MRSTUDYR: Retrospectively Studying the Effectiveness of Mutant Reduction Techniques},
            pdfborder={0 0 0},
            breaklinks=true}
\urlstyle{same}  % don't use monospace font for urls
\usepackage{color}
\usepackage{fancyvrb}
\newcommand{\VerbBar}{|}
\newcommand{\VERB}{\Verb[commandchars=\\\{\}]}
\DefineVerbatimEnvironment{Highlighting}{Verbatim}{commandchars=\\\{\}}
% Add ',fontsize=\small' for more characters per line
\newenvironment{Shaded}{}{}
\newcommand{\KeywordTok}[1]{\textcolor[rgb]{0.00,0.44,0.13}{\textbf{{#1}}}}
\newcommand{\DataTypeTok}[1]{\textcolor[rgb]{0.56,0.13,0.00}{{#1}}}
\newcommand{\DecValTok}[1]{\textcolor[rgb]{0.25,0.63,0.44}{{#1}}}
\newcommand{\BaseNTok}[1]{\textcolor[rgb]{0.25,0.63,0.44}{{#1}}}
\newcommand{\FloatTok}[1]{\textcolor[rgb]{0.25,0.63,0.44}{{#1}}}
\newcommand{\ConstantTok}[1]{\textcolor[rgb]{0.53,0.00,0.00}{{#1}}}
\newcommand{\CharTok}[1]{\textcolor[rgb]{0.25,0.44,0.63}{{#1}}}
\newcommand{\SpecialCharTok}[1]{\textcolor[rgb]{0.25,0.44,0.63}{{#1}}}
\newcommand{\StringTok}[1]{\textcolor[rgb]{0.25,0.44,0.63}{{#1}}}
\newcommand{\VerbatimStringTok}[1]{\textcolor[rgb]{0.25,0.44,0.63}{{#1}}}
\newcommand{\SpecialStringTok}[1]{\textcolor[rgb]{0.73,0.40,0.53}{{#1}}}
\newcommand{\ImportTok}[1]{{#1}}
\newcommand{\CommentTok}[1]{\textcolor[rgb]{0.38,0.63,0.69}{\textit{{#1}}}}
\newcommand{\DocumentationTok}[1]{\textcolor[rgb]{0.73,0.13,0.13}{\textit{{#1}}}}
\newcommand{\AnnotationTok}[1]{\textcolor[rgb]{0.38,0.63,0.69}{\textbf{\textit{{#1}}}}}
\newcommand{\CommentVarTok}[1]{\textcolor[rgb]{0.38,0.63,0.69}{\textbf{\textit{{#1}}}}}
\newcommand{\OtherTok}[1]{\textcolor[rgb]{0.00,0.44,0.13}{{#1}}}
\newcommand{\FunctionTok}[1]{\textcolor[rgb]{0.02,0.16,0.49}{{#1}}}
\newcommand{\VariableTok}[1]{\textcolor[rgb]{0.10,0.09,0.49}{{#1}}}
\newcommand{\ControlFlowTok}[1]{\textcolor[rgb]{0.00,0.44,0.13}{\textbf{{#1}}}}
\newcommand{\OperatorTok}[1]{\textcolor[rgb]{0.40,0.40,0.40}{{#1}}}
\newcommand{\BuiltInTok}[1]{{#1}}
\newcommand{\ExtensionTok}[1]{{#1}}
\newcommand{\PreprocessorTok}[1]{\textcolor[rgb]{0.74,0.48,0.00}{{#1}}}
\newcommand{\AttributeTok}[1]{\textcolor[rgb]{0.49,0.56,0.16}{{#1}}}
\newcommand{\RegionMarkerTok}[1]{{#1}}
\newcommand{\InformationTok}[1]{\textcolor[rgb]{0.38,0.63,0.69}{\textbf{\textit{{#1}}}}}
\newcommand{\WarningTok}[1]{\textcolor[rgb]{0.38,0.63,0.69}{\textbf{\textit{{#1}}}}}
\newcommand{\AlertTok}[1]{\textcolor[rgb]{1.00,0.00,0.00}{\textbf{{#1}}}}
\newcommand{\ErrorTok}[1]{\textcolor[rgb]{1.00,0.00,0.00}{\textbf{{#1}}}}
\newcommand{\NormalTok}[1]{{#1}}
\usepackage{graphicx,grffile}
\makeatletter
\def\maxwidth{\ifdim\Gin@nat@width>\linewidth\linewidth\else\Gin@nat@width\fi}
\def\maxheight{\ifdim\Gin@nat@height>\textheight\textheight\else\Gin@nat@height\fi}
\makeatother
% Scale images if necessary, so that they will not overflow the page
% margins by default, and it is still possible to overwrite the defaults
% using explicit options in \includegraphics[width, height, ...]{}
\setkeys{Gin}{width=\maxwidth,height=\maxheight,keepaspectratio}
\IfFileExists{parskip.sty}{%
\usepackage{parskip}
}{% else
\setlength{\parindent}{0pt}
\setlength{\parskip}{6pt plus 2pt minus 1pt}
}
\setlength{\emergencystretch}{3em}  % prevent overfull lines
\providecommand{\tightlist}{%
  \setlength{\itemsep}{0pt}\setlength{\parskip}{0pt}}
\setcounter{secnumdepth}{5}
% Redefines (sub)paragraphs to behave more like sections
\ifx\paragraph\undefined\else
\let\oldparagraph\paragraph
\renewcommand{\paragraph}[1]{\oldparagraph{#1}\mbox{}}
\fi
\ifx\subparagraph\undefined\else
\let\oldsubparagraph\subparagraph
\renewcommand{\subparagraph}[1]{\oldsubparagraph{#1}\mbox{}}
\fi

%%% Use protect on footnotes to avoid problems with footnotes in titles
\let\rmarkdownfootnote\footnote%
\def\footnote{\protect\rmarkdownfootnote}

%%% Change title format to be more compact
\usepackage{titling}

% Create subtitle command for use in maketitle
\newcommand{\subtitle}[1]{
  \posttitle{
    \begin{center}\large#1\end{center}
    }
}

\setlength{\droptitle}{-2em}
  \title{MRSTUDYR: Retrospectively Studying the Effectiveness of Mutant Reduction
Techniques}
  \pretitle{\vspace{\droptitle}\centering\huge}
  \posttitle{\par}
  \author{}
  \preauthor{}\postauthor{}
  \date{}
  \predate{}\postdate{}

\begin{document}
\maketitle

\section{Installing the Ineffective Mutant Analysis
Package}\label{installing-the-ineffective-mutant-analysis-package}

This code will install the ineffectivemutants package from GitHub using
the \texttt{install\_github} function.

\begin{Shaded}
\begin{Highlighting}[]
\KeywordTok{options}\NormalTok{(}\DataTypeTok{scipen=}\DecValTok{10}\NormalTok{, }\DataTypeTok{width=}\DecValTok{200}\NormalTok{)}
\NormalTok{devtools::}\KeywordTok{install_github}\NormalTok{(}\StringTok{"mccurdyc/mrstudyr"}\NormalTok{)}
\end{Highlighting}
\end{Shaded}

\begin{quote}
\section{Comparing the Mutation
Scores}\label{comparing-the-mutation-scores}
\end{quote}

\subsection{Initialize the System}\label{initialize-the-system}

First, load in the libraries that are used in addition to those with the
mrstudyr package (i.e., load all of the packages not used by mrstudyr
but still used in this RMarkdown file). Note that right now the mrstudyr
package will automatically load all of the packages that it needs to
performs its various analyses. Now, we are ready to call the functions
from the ineffectivemutants package and produce the appropriate summary
tables and graphs.

\begin{Shaded}
\begin{Highlighting}[]
\KeywordTok{suppressPackageStartupMessages}\NormalTok{(}\KeywordTok{library}\NormalTok{(mrstudyr))}
\KeywordTok{suppressPackageStartupMessages}\NormalTok{(}\KeywordTok{library}\NormalTok{(knitr))}
\end{Highlighting}
\end{Shaded}

\subsection{Show the Schemas for the ICSME 2016
Paper}\label{show-the-schemas-for-the-icsme-2016-paper}

\begin{Shaded}
\begin{Highlighting}[]
\NormalTok{sqlite_data <-}\StringTok{ }\KeywordTok{read_sqlite_avmdefaults}\NormalTok{()}
\NormalTok{schemas <-}\StringTok{ }\NormalTok{sqlite_data %>%}\StringTok{ }\KeywordTok{collect_study_schemas}\NormalTok{() %>%}
\StringTok{           }\KeywordTok{collect_normal_data}\NormalTok{() %>%}\StringTok{ }\KeywordTok{select}\NormalTok{(schema) %>%}\StringTok{ }\KeywordTok{unique}\NormalTok{()}
\end{Highlighting}
\end{Shaded}

\subsection{Perform Reduction
Techniques}\label{perform-reduction-techniques}

First, performing the most common reduction technique, random sampling.

\begin{Shaded}
\begin{Highlighting}[]
\NormalTok{d <-}\StringTok{ }\KeywordTok{read_sqlite_avmdefaults}\NormalTok{()}
\NormalTok{normal_data <-}\StringTok{ }\NormalTok{d %>%}\StringTok{ }\KeywordTok{collect_study_schemas}\NormalTok{() %>%}\StringTok{ }\KeywordTok{collect_normal_data}\NormalTok{()}
\NormalTok{random_sampling_data <-}\StringTok{ }\NormalTok{normal_data %>%}\StringTok{ }\KeywordTok{analyze_random_sampling}\NormalTok{()}
\end{Highlighting}
\end{Shaded}

Next, applying the concepts of random sampling but, over operators.

\begin{Shaded}
\begin{Highlighting}[]
\NormalTok{d <-}\StringTok{ }\KeywordTok{read_sqlite_avmdefaults}\NormalTok{()}
\NormalTok{normal_data <-}\StringTok{ }\NormalTok{d %>%}\StringTok{ }\KeywordTok{collect_study_schemas}\NormalTok{() %>%}\StringTok{ }\KeywordTok{collect_normal_data}\NormalTok{()}
\NormalTok{operator_sampling_data <-}\StringTok{ }\NormalTok{normal_data %>%}\StringTok{ }\KeywordTok{analyze_across_operators}\NormalTok{()}
\end{Highlighting}
\end{Shaded}

\subsection{Visualize the Mutation Scores for the Random Sampling
Technique}\label{visualize-the-mutation-scores-for-the-random-sampling-technique}

Now, to plot the graph that shows the mutation score for the random
sampling mutant reduction technique for four user-specified percentages.
In this example, we will create a visualisation containing the
percentages found in the accompanying paper.

\begin{Shaded}
\begin{Highlighting}[]
\NormalTok{d <-}\StringTok{ }\KeywordTok{read_sqlite_avmdefaults}\NormalTok{()}
\NormalTok{normal_data <-}\StringTok{ }\NormalTok{d %>%}\StringTok{ }\KeywordTok{collect_study_schemas}\NormalTok{() %>%}\StringTok{ }\KeywordTok{collect_normal_data}\NormalTok{()}
\NormalTok{filtered_percents_data <-}\StringTok{ }\NormalTok{random_sampling_data %>%}\StringTok{ }\KeywordTok{collect_chosen_percent_data}\NormalTok{(}\KeywordTok{c}\NormalTok{(}\DecValTok{1}\NormalTok{, }\DecValTok{10}\NormalTok{, }\DecValTok{20}\NormalTok{, }\DecValTok{40}\NormalTok{))}
\end{Highlighting}
\end{Shaded}

To actually create the visualization, use the following function, giving
it the reduced data from four percentages.

\begin{Shaded}
\begin{Highlighting}[]
\KeywordTok{visualize_plot_mutation_score}\NormalTok{(filtered_percents_data)}
\end{Highlighting}
\end{Shaded}

\includegraphics{mrstudyr_analysis_files/figure-latex/mutation_score_random-1.pdf}

Analysis of the results in this graph:

\begin{itemize}
\tightlist
\item
  Great variability in mutation scores at percentage: 1
\item
  Even at percentage: 10, variability in mutation scores is reduced
\item
  The same trend remains true for percentage: 20 and 40, but a smaller
  reduction in variability of mutation scores
\end{itemize}

Now, plot the graph that shows the mutation score for the operator
sampling mutant reduction technique.

First, select four percentages to compare.

\begin{Shaded}
\begin{Highlighting}[]
\NormalTok{d <-}\StringTok{ }\KeywordTok{read_sqlite_avmdefaults}\NormalTok{()}
\NormalTok{normal_data <-}\StringTok{ }\NormalTok{d %>%}\StringTok{ }\KeywordTok{collect_study_schemas}\NormalTok{() %>%}\StringTok{ }\KeywordTok{collect_normal_data}\NormalTok{()}
\NormalTok{filtered_percents_data <-}\StringTok{ }\NormalTok{operator_sampling_data %>%}\StringTok{ }\KeywordTok{collect_chosen_percent_data}\NormalTok{(}\KeywordTok{c}\NormalTok{(}\DecValTok{1}\NormalTok{, }\DecValTok{10}\NormalTok{, }\DecValTok{20}\NormalTok{, }\DecValTok{40}\NormalTok{))}
\end{Highlighting}
\end{Shaded}

Then, plot those four percentages against each other.

\begin{Shaded}
\begin{Highlighting}[]
\KeywordTok{visualize_plot_mutation_score}\NormalTok{(filtered_percents_data)}
\end{Highlighting}
\end{Shaded}

\includegraphics{mrstudyr_analysis_files/figure-latex/mutation_score_operator-1.pdf}

Analysis of the results in this graph:

\begin{itemize}
\tightlist
\item
  Similar to random sampling, there is great variability in mutation
  scores at percentage: 1
\item
  At percentage: 10, variability in mutation scores is reduced
  substantially
\item
  This trend continues for percentage: 20 and 40, but with a smaller
  reduction in variability of mutation scores over these percentages
\end{itemize}


\end{document}
