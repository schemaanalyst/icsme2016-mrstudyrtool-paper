
%% bare_conf.tex
%% V1.4b
%% 2015/08/26
%% by Michael Shell
%% See:
%% http://www.michaelshell.org/
%% for current contact information.
%%
%% This is a skeleton file demonstrating the use of IEEEtran.cls
%% (requires IEEEtran.cls version 1.8b or later) with an IEEE
%% conference paper.
%%
%% Support sites:
%% http://www.michaelshell.org/tex/ieeetran/
%% http://www.ctan.org/pkg/ieeetran
%% and
%% http://www.ieee.org/

%%*************************************************************************
%% Legal Notice:
%% This code is offered as-is without any warranty either expressed or
%% implied; without even the implied warranty of MERCHANTABILITY or
%% FITNESS FOR A PARTICULAR PURPOSE! 
%% User assumes all risk.
%% In no event shall the IEEE or any contributor to this code be liable for
%% any damages or losses, including, but not limited to, incidental,
%% consequential, or any other damages, resulting from the use or misuse
%% of any information contained here.
%%
%% All comments are the opinions of their respective authors and are not
%% necessarily endorsed by the IEEE.
%%
%% This work is distributed under the LaTeX Project Public License (LPPL)
%% ( http://www.latex-project.org/ ) version 1.3, and may be freely used,
%% distributed and modified. A copy of the LPPL, version 1.3, is included
%% in the base LaTeX documentation of all distributions of LaTeX released
%% 2003/12/01 or later.
%% Retain all contribution notices and credits.
%% ** Modified files should be clearly indicated as such, including  **
%% ** renaming them and changing author support contact information. **
%%*************************************************************************


% *** Authors should verify (and, if needed, correct) their LaTeX system  ***
% *** with the testflow diagnostic prior to trusting their LaTeX platform ***
% *** with production work. The IEEE's font choices and paper sizes can   ***
% *** trigger bugs that do not appear when using other class files.       ***                          ***
% The testflow support page is at:
% http://www.michaelshell.org/tex/testflow/



\documentclass[conference]{IEEEtran}
% Some Computer Society conferences also require the compsoc mode option,
% but others use the standard conference format.
%
% If IEEEtran.cls has not been installed into the LaTeX system files,
% manually specify the path to it like:
% \documentclass[conference]{../sty/IEEEtran}





% Some very useful LaTeX packages include:
% (uncomment the ones you want to load)


% *** MISC UTILITY PACKAGES ***
%
%\usepackage{ifpdf}
% Heiko Oberdiek's ifpdf.sty is very useful if you need conditional
% compilation based on whether the output is pdf or dvi.
% usage:
% \ifpdf
%   % pdf code
% \else
%   % dvi code
% \fi
% The latest version of ifpdf.sty can be obtained from:
% http://www.ctan.org/pkg/ifpdf
% Also, note that IEEEtran.cls V1.7 and later provides a builtin
% \ifCLASSINFOpdf conditional that works the same way.
% When switching from latex to pdflatex and vice-versa, the compiler may
% have to be run twice to clear warning/error messages.






% *** CITATION PACKAGES ***
%
%\usepackage{cite}
% cite.sty was written by Donald Arseneau
% V1.6 and later of IEEEtran pre-defines the format of the cite.sty package
% \cite{} output to follow that of the IEEE. Loading the cite package will
% result in citation numbers being automatically sorted and properly
% "compressed/ranged". e.g., [1], [9], [2], [7], [5], [6] without using
% cite.sty will become [1], [2], [5]--[7], [9] using cite.sty. cite.sty's
% \cite will automatically add leading space, if needed. Use cite.sty's
% noadjust option (cite.sty V3.8 and later) if you want to turn this off
% such as if a citation ever needs to be enclosed in parenthesis.
% cite.sty is already installed on most LaTeX systems. Be sure and use
% version 5.0 (2009-03-20) and later if using hyperref.sty.
% The latest version can be obtained at:
% http://www.ctan.org/pkg/cite
% The documentation is contained in the cite.sty file itself.






% *** GRAPHICS RELATED PACKAGES ***
%
\ifCLASSINFOpdf
  \usepackage{graphicx}
  % declare the path(s) where your graphic files are
  \graphicspath{{imgs/}}
  % and their extensions so you won't have to specify these with
  % every instance of \includegraphics
  \DeclareGraphicsExtensions{.pdf}
\else
  % or other class option (dvipsone, dvipdf, if not using dvips). graphicx
  % will default to the driver specified in the system graphics.cfg if no
  % driver is specified.
  % \usepackage[dvips]{graphicx}
  % declare the path(s) where your graphic files are
  % \graphicspath{{../eps/}}
  % and their extensions so you won't have to specify these with
  % every instance of \includegraphics
  % \DeclareGraphicsExtensions{.eps}
\fi
% graphicx was written by David Carlisle and Sebastian Rahtz. It is
% required if you want graphics, photos, etc. graphicx.sty is already
% installed on most LaTeX systems. The latest version and documentation
% can be obtained at: 
% http://www.ctan.org/pkg/graphicx
% Another good source of documentation is "Using Imported Graphics in
% LaTeX2e" by Keith Reckdahl which can be found at:
% http://www.ctan.org/pkg/epslatex
%
% latex, and pdflatex in dvi mode, support graphics in encapsulated
% postscript (.eps) format. pdflatex in pdf mode supports graphics
% in .pdf, .jpeg, .png and .mps (metapost) formats. Users should ensure
% that all non-photo figures use a vector format (.eps, .pdf, .mps) and
% not a bitmapped formats (.jpeg, .png). The IEEE frowns on bitmapped formats
% which can result in "jaggedy"/blurry rendering of lines and letters as
% well as large increases in file sizes.
%
% You can find documentation about the pdfTeX application at:
% http://www.tug.org/applications/pdftex





% *** MATH PACKAGES ***
%
%\usepackage{amsmath}
% A popular package from the American Mathematical Society that provides
% many useful and powerful commands for dealing with mathematics.
%
% Note that the amsmath package sets \interdisplaylinepenalty to 10000
% thus preventing page breaks from occurring within multiline equations. Use:
%\interdisplaylinepenalty=2500
% after loading amsmath to restore such page breaks as IEEEtran.cls normally
% does. amsmath.sty is already installed on most LaTeX systems. The latest
% version and documentation can be obtained at:
% http://www.ctan.org/pkg/amsmath





% *** SPECIALIZED LIST PACKAGES ***
%
%\usepackage{algorithmic}
% algorithmic.sty was written by Peter Williams and Rogerio Brito.
% This package provides an algorithmic environment fo describing algorithms.
% You can use the algorithmic environment in-text or within a figure
% environment to provide for a floating algorithm. Do NOT use the algorithm
% floating environment provided by algorithm.sty (by the same authors) or
% algorithm2e.sty (by Christophe Fiorio) as the IEEE does not use dedicated
% algorithm float types and packages that provide these will not provide
% correct IEEE style captions. The latest version and documentation of
% algorithmic.sty can be obtained at:
% http://www.ctan.org/pkg/algorithms
% Also of interest may be the (relatively newer and more customizable)
% algorithmicx.sty package by Szasz Janos:
% http://www.ctan.org/pkg/algorithmicx




% *** ALIGNMENT PACKAGES ***
%
%\usepackage{array}
% Frank Mittelbach's and David Carlisle's array.sty patches and improves
% the standard LaTeX2e array and tabular environments to provide better
% appearance and additional user controls. As the default LaTeX2e table
% generation code is lacking to the point of almost being broken with
% respect to the quality of the end results, all users are strongly
% advised to use an enhanced (at the very least that provided by array.sty)
% set of table tools. array.sty is already installed on most systems. The
% latest version and documentation can be obtained at:
% http://www.ctan.org/pkg/array


% IEEEtran contains the IEEEeqnarray family of commands that can be used to
% generate multiline equations as well as matrices, tables, etc., of high
% quality.




% *** SUBFIGURE PACKAGES ***
%\ifCLASSOPTIONcompsoc
 \usepackage{subfig}
%\else
%  \usepackage[caption=false,font=footnotesize]{subfig}
%\fi
% subfig.sty, written by Steven Douglas Cochran, is the modern replacement
% for subfigure.sty, the latter of which is no longer maintained and is
% incompatible with some LaTeX packages including fixltx2e. However,
% subfig.sty requires and automatically loads Axel Sommerfeldt's caption.sty
% which will override IEEEtran.cls' handling of captions and this will result
% in non-IEEE style figure/table captions. To prevent this problem, be sure
% and invoke subfig.sty's "caption=false" package option (available since
% subfig.sty version 1.3, 2005/06/28) as this is will preserve IEEEtran.cls
% handling of captions.
% Note that the Computer Society format requires a larger sans serif font
% than the serif footnote size font used in traditional IEEE formatting
% and thus the need to invoke different subfig.sty package options depending
% on whether compsoc mode has been enabled.
%
% The latest version and documentation of subfig.sty can be obtained at:
% http://www.ctan.org/pkg/subfig




% *** FLOAT PACKAGES ***
%
%\usepackage{fixltx2e}
% fixltx2e, the successor to the earlier fix2col.sty, was written by
% Frank Mittelbach and David Carlisle. This package corrects a few problems
% in the LaTeX2e kernel, the most notable of which is that in current
% LaTeX2e releases, the ordering of single and double column floats is not
% guaranteed to be preserved. Thus, an unpatched LaTeX2e can allow a
% single column figure to be placed prior to an earlier double column
% figure.
% Be aware that LaTeX2e kernels dated 2015 and later have fixltx2e.sty's
% corrections already built into the system in which case a warning will
% be issued if an attempt is made to load fixltx2e.sty as it is no longer
% needed.
% The latest version and documentation can be found at:
% http://www.ctan.org/pkg/fixltx2e


\usepackage{stfloats}
% stfloats.sty was written by Sigitas Tolusis. This package gives LaTeX2e
% the ability to do double column floats at the bottom of the page as well
% as the top. (e.g., "\begin{figure*}[!b]" is not normally possible in
% LaTeX2e). It also provides a command:
%\fnbelowfloat
% to enable the placement of footnotes below bottom floats (the standard
% LaTeX2e kernel puts them above bottom floats). This is an invasive package
% which rewrites many portions of the LaTeX2e float routines. It may not work
% with other packages that modify the LaTeX2e float routines. The latest
% version and documentation can be obtained at:
% http://www.ctan.org/pkg/stfloats
% Do not use the stfloats baselinefloat ability as the IEEE does not allow
% \baselineskip to stretch. Authors submitting work to the IEEE should note
% that the IEEE rarely uses double column equations and that authors should try
% to avoid such use. Do not be tempted to use the cuted.sty or midfloat.sty
% packages (also by Sigitas Tolusis) as the IEEE does not format its papers in
% such ways.
% Do not attempt to use stfloats with fixltx2e as they are incompatible.
% Instead, use Morten Hogholm'a dblfloatfix which combines the features
% of both fixltx2e and stfloats:
%
% \usepackage{dblfloatfix}
% The latest version can be found at:
% http://www.ctan.org/pkg/dblfloatfix




% *** PDF, URL AND HYPERLINK PACKAGES ***
%
%\usepackage{url}
% url.sty was written by Donald Arseneau. It provides better support for
% handling and breaking URLs. url.sty is already installed on most LaTeX
% systems. The latest version and documentation can be obtained at:
% http://www.ctan.org/pkg/url
% Basically, \url{my_url_here}.




% *** Do not adjust lengths that control margins, column widths, etc. ***
% *** Do not use packages that alter fonts (such as pslatex).         ***
% There should be no need to do such things with IEEEtran.cls V1.6 and later.
% (Unless specifically asked to do so by the journal or conference you plan
% to submit to, of course. )


% correct bad hyphenation here
\hyphenation{op-tical net-works semi-conduc-tor}


\begin{document}
%
% paper title
% Titles are generally capitalized except for words such as a, an, and, as,
% at, but, by, for, in, nor, of, on, or, the, to and up, which are usually
% not capitalized unless they are the first or last word of the title.
% Linebreaks \\ can be used within to get better formatting as desired.
% Do not put math or special symbols in the title.
\title{\textit{msselectr}: Analysing the Selective Approaches for Performing Mutation Analysis}

% \author{\IEEEauthorblockN{Colton J. McCurdy}
% \IEEEauthorblockA{Department of Computer Science \\
% Allegheny College, USA\\
% \texttt{mccurdyc@allegheny.edu}}}

\author{
\IEEEauthorblockN{Phil McMinn}
\IEEEauthorblockA{Department of Computer Science \\
University of Sheffield, UK}
\and
\IEEEauthorblockN{Colton J. McCurdy and Gregory M. Kapfhammer}
\IEEEauthorblockA{Department of Computer Science \\
Allegheny College, USA}}

% conference papers do not typically use \thanks and this command
% is locked out in conference mode. If really needed, such as for
% the acknowledgment of grants, issue a \IEEEoverridecommandlockouts
% after \documentclass

% for over three affiliations, or if they all won't fit within the width
% of the page, use this alternative format:
% 
% \author{\IEEEauthorblockN{Michael Shell\IEEEauthorrefmark{1},
% Homer Simpson\IEEEauthorrefmark{2},
% James Kirk\IEEEauthorrefmark{3}, 
% Montgomery Scott\IEEEauthorrefmark{3} and
% Eldon Tyrell\IEEEauthorrefmark{4}}
% \IEEEauthorblockA{\IEEEauthorrefmark{1}School of Electrical and Computer Engineering\\
% Georgia Institute of Technology,
% Atlanta, Georgia 30332--0250\\ Email: see http://www.michaelshell.org/contact.html}
% \IEEEauthorblockA{\IEEEauthorrefmark{2}Twentieth Century Fox, Springfield, USA\\
% Email: homer@thesimpsons.com}
% \IEEEauthorblockA{\IEEEauthorrefmark{3}Starfleet Academy, San Francisco, California 96678-2391\\
% Telephone: (800) 555--1212, Fax: (888) 555--1212}
% \IEEEauthorblockA{\IEEEauthorrefmark{4}Tyrell Inc., 123 Replicant Street, Los Angeles, California 90210--4321}}




% use for special paper notices
%\IEEEspecialpapernotice{(Invited Paper)}

% make the title area
\maketitle

% As a general rule, do not put math, special symbols or citations
% in the abstract
\begin{abstract}
    Abstract will go here. It will be more clear and focus on the contributions of the tool
\end{abstract}

% no keywords
% For peer review papers, you can put extra information on the cover
% page as needed:
% \ifCLASSOPTIONpeerreview
% \begin{center} \bfseries EDICS Category: 3-BBND \end{center}
% \fi
%
% For peerreview papers, this IEEEtran command inserts a page break and
% creates the second title. It will be ignored for other modes.
\IEEEpeerreviewmaketitle

\section{Introduction}
    Key Contributions:
    \begin{itemize}
        \item Comprehensive tool for analysing selective mutation analysis.
        \item Released free and open source in a standardised format as an R package.
        \item Can easily be modified to analyse data from many domains because of the acceptance of a standardised general input format.
    \end{itemize}

\subsection{Mutation Testing}

% Mutation testing is a well-known procedure for determining the quality of test
% suites \cite{gopinath2015empirical}. Additionally, as McMinn et al. discovered, the quality of a database schema
% can be determined through the use of mutation testing \cite{mcminn2015effectiveness}. In
% this paper, I am using mutant data from performing mutation analysis on 34 well-known database schemas.
%
% Testing mutations---modified versions of source code containing faults---which is now
% referred to as mutation testing, was originally proposed by Richard A. DeMillo et al.
% in the 1970's \cite{demillo1978hints}. Each modified version of a program or schema contains
% exactly one fault. The goal is to have each modified, or faulty, version of the source code
% fail when executed against the tests \cite{offutt1993experimental}. If a test fails that was expected to fail
% for the given input, then the fault has been exploited and the mutation, also referred to as a mutant,
% was \textit{killed}. A mutant that has been killed is considered to be \textit{dead} and no
% longer needs to remain in the testing process since the faults represented by this type of
% mutant are confirmed as being detectable by the test suite \cite{offutt1993experimental}.
% Conversely, if a test passes, then the fault was not exploited and is considered to remain
% \textit{alive} and are analyzed in further testing \cite{mcminnvirtual}.
%
% After constructing all of the mutants for a given program, the user then adds the test cases
% to a mutation testing system such as \texttt{MAJOR} \cite{just2011major} or
% \texttt{Defects4J} \cite{just2014defects4j}. After adding the tests to a mutation testing system,
% the user then needs to ensure that the output is correct. If the output is incorrect, then
% a fault has been exploited and the mutant is considered to be killed \cite{offutt1993experimental}.
% After all test cases have been executed against all of the live mutants, remaining mutants are either
% considered to be \textit{equivalent} or the test cases are insufficient to kill that type of mutant \cite{gopinath2015empirical}. An equivalent
% mutant always produces the same output as the original program, making it undetectable by any test. On the contrary, a mutant
% is considered to be \textit{normal} if it is not equivalent.
% Finally, a score representing the quality of a test suite or database schema---referred to as the mutation score for the test data---can be calculated.

% \subsection{Calculating Mutation Score}
% Mutation score, displayed in Equation \ref{eq:ms}, is the ratio of killed \textit{normal} faults, denoted $k$ to the total number of \textit{normal} faults \cite{wright2013efficient}.
% The total number of normal mutants can be calculated by summing the number of normal mutants that are both killed and alive, denoted $k$ and $a$, repectively.
% Equivalent mutants are excluded because, by definition, they cannot be detected by a unit test \cite{inozemtseva2014coverage}.
%
% \begin{equation}
%     Mutation \, Score(k,a) = \frac{k}{k+a}
%     \label{eq:ms}
% \end{equation}
%
% Test data with a 100\% mutation score is said to be \textit{mutation adequate} \cite{offutt1993experimental}.
% Mutation adequacy can mean one of two things: the test suite
% or database schema is of high quality or the introduced faults are easily
% discovered and are not stringent predictors of the test suite or schema's quality.

\subsection{The Cost of Mutation Testing}
% As previously noted, mutation testing is not able
% to be used in practice due to its computational expense.
% The primary computational cost is incurred from running the mutated versions
% of the source code or database schema against the test cases or integrity
% constraints \cite{offutt1993experimental}. Due to the fact that each generated
% mutant needs to be run against at least one, possibly many, test cases, mutation
% testing requires a large number of computations \cite{offutt1993experimental}.
%
% There has been a substantial amount of work to find a reduced approach
% for performing mutation testing \cite{gopinath2015empirical, mcminnvirtual, gopinath2015mutation, offutt1993experimental}.
% Offut and Untch categorize the existing mutation testing reduction approaches
% into three categories: do \textit{fewer}, do \textit{smarter} and do \textit{faster} \cite{offutt1993experimental}.
% In this paper, I am only focused on the do fewer categorization,
% specifically, selective mutation testing.

% \begin{figure*}[!ht]
% \centering
% \subfloat[Uniform random sampling]{\includegraphics[width=2.5in]{inventory_perc_v_ms}
% \label{fig:iv_s}}
% \hfil
% \subfloat[Stratified random sampling over mutation operators]{\includegraphics[width=2.5in]{inventory_op_perc_v_ms}
% \label{fig:iv_o}}
% \caption{Comparing uniform random sampling to stratified random sampling over
% mutation operators for the Inventory schema.}
% \label{fig:comp}
% \end{figure*}


\subsection{Selective Mutation Testing}
% Mathur suggests an approach to reducing the cost of performing mutation testing by
% reducing the modified or mutant versions of the source code \cite{mathur1991performance}.
% Gopinath et al. discuss two existing approaches for performing selective
% mutation analysis. The two strategies are
% sampling criteria and operator selection \cite{gopinath2015mutation}. Sampling criteria is where I am
% currently focused, specifically on unified random sampling and stratified
% random sampling over mutation operators.
%
% Uniform random sampling is the most trivial sampling approach, proposed by Timothy A. Budd,
% where an arbitrary fraction of the full set of mutants is chosen \cite{budd1980mutation}.
% Zhang et al. analyzed the performance of uniform random sampling as a reduction technique
% and found that sampling as few as 5\% of mutants can be very highly correlated with the full mutation
% score \cite{zhang2010operator, zhang2013operator}.
%
% Wong et al. first suggested sampling the same fraction of mutants from each operator, known as
% stratified random sampling over mutation operators \cite{wong1995reducing}. It is similar
% to uniform random sampling in the way that the same arbitrary fraction is applied. However,
% instead of applying the fraction over all mutants for a schema, the fraction is applied for all operators. In other
% words, the reduced set contains all possible operators, but only a fraction of the mutants
% produced by that operator. Stratified random sampling over mutation operators is subtly different
% from uniform random sampling because each operator is still present in the reduced set, where
% in uniform random sampling this is not considered. For example, an operator that generates few
% mutants compared to an operator that generates many mutants is less likely to have mutants represented
% in the reduced set.
%
% At this point, these are the only reduction strategies present in my research and available for analyzation
% with my tool. As time goes on and I implement more approaches, I will discuss them in further detail.

\subsection{Analysing Database Schema Mutants}
refer to the AST2016 paper
still computationally expensive\\
able to utilize existing select approaches


\section{Comprehensiveness of \textit{msselectr}}
approaches contained in the tool

\section{Released in a Standardised Format}
\textit{note: do not innapropriately include too much detail}
talk about why we chose to use R \\
talk about the importance of free and open-source software \\
talk about why reproducibility is important

\section{Generalised Input Format}
not limited to a single domain
allows for use in old and new domains with little modification
    \subsection{application to databases}
    talk about mutation analysis of database schema mutants
    \subsection{application to programs}

\section{Experimental Setup}
% For analyzing the existing selective approaches discussed previously, I have created
% a tool, \texttt{selectms}. This tool reads in data, expecting the data to contain
% specific attributes: identifier, dbms, schema, operator, type, killed, time. The identifier attribute
% is a universally unique set of characters describing the combination of database management system (dmbs)
% and schema. The dbms attribute is as the name suggests, a character value representing the database management system under observation.
% The schema and operator attributes, both character values, are the schema and operator under observation. The type attribute, a set of characters,
% refers to the type of mutant: equivalent, stillborn or normal. However, as previously mentioned, we filter out all non-normal mutants.
% A boolean value, killed, is either true or false, if the mutant is dead or alive, respectively. Finally, the time attribute is an
% integer value representing the time it takes---in milliseconds---to create the mutant.

\subsection{$x\%$ random selection with $x$ increasing by 10}
% As previously mentioned, I am comparing uniform random sampling and stratified random sampling over mutation operators.
% The fractional values that I analyzed for both of the selective approaches are:
% $\{\frac{1}{100}, \frac{10}{100}, \frac{20}{100}, \frac{30}{100}, \frac{40}{100}, \frac{50}{100}, \frac{60}{100}, \frac{70}{100}, \frac{80}{100}, \frac{90}{100}\}$.
% I chose to increase by 10\% rather than analyzing each percentage due to the computation expense of the analysis if performed for each percentage.

continue discussing why we chose 10\% increments (refer to one paper from annotated bib)

\subsection{30 trials}
% We ran each possible configuration of fractional value for uniform random sampling and stratified random sampling
% for 30 trials as suggested by Traeger and Zadok to allow the results to approximate a normal distribution \cite{traeger2008nine}. Additionally,
% We kept the fractional values and number of trials for each configuration of sampling approach consistent throughout the analysis.

% \begin{figure*}[!hb]
% \centering
% \subfloat[Uniform random sampling correlation]{\includegraphics[width=2.5in]{rs_corr}
% \label{fig:iv_c}}
% \hfil
% \subfloat[Stratified random sampling over mutation operators correlation]{\includegraphics[width=2.5in]{op_corr}
% \label{fig:iv_op_c}}
% \caption{The correlation of mutation scores across percentages for the uniform random sampling and stratified random sampling
% across operators approaches, specifically for the Inventory schema.}
% \label{fig:comp_corr}
% \end{figure*}

\vspace*{-0.25cm}
\section{Analysis of Results}

include tables about the mutant data (alive, dead, total, type)
provide results from more than just Inventory schema

% The data analyzed in this paper was generated from performing mutation testing on
% 34 database schemas using the SQLite database management system. The data
% contain 118860 mutants, which as a whole, take 67287543 milliseconds to analyze using mutation analysis.
%
% Reducing this cost is the primary focus of this analysis. However, reducing cost by analyzing fewer mutants
% leads to weakly correlated mutation scores and higher mean absolute and root mean squared errors (displayed
% in Figures \ref{fig:comp_corr} and \ref{fig:comp_error}, respectively). In this paper, I will
% analyze the graphs comparing mutation score, error and correlation (on the y-axis) to percentage (on the x-axis),
% specifically for the Inventory schema.
%
% As previously mentioned, the mutation score associated with a test suite or database schema is a value
% corresponding to the quality of the test suite or schema. When reducing the mutant set by fractional values,
% we are able to receive mutation scores that are very close to the original mutation score (e.g., both uniform random sampling
% and stratified random sampling over operators on only $30\%$ of the original mutants produce mutation scores ranging between 0.70 and 0.80, where the
% original mutation score is 0.75). As displayed
% in Figure \ref{fig:comp} and previously noted by Budd et al. in \cite{budd1980mutation}, even as small
% as $10\%$ of the original set can produce mutation scores that are considered to be highly correlated
% ($\tau_b \approx 0.73$ for uniform random sampling and $\tau_b \approx 0.76$ for stratified random sampling over operators).
% The correlation between the mutation scores of both reduced sets from both approaches and the original set for all fractional values---using Kendall's $\tau_b$---can
% observed in Figure \ref{fig:comp_corr}.
%
% Inozemtseva and Holmes adopt the Guildford scale to describe correlation values. I will similarly
% describe correlations between sets' mutation score using the same scale.
% The Guildford scale describes the absolute value of a coefficient as ``low'' when it is less than 0.4,
% ``moderate'' when it is between 0.4 and 0.7, ``high'' when ranging from 0.7 to 0.9, and ``very high''
% when it is greater than 0.9 \cite{inozemtseva2014coverage}.
%
% Finally, when observing mean absolute and root mean squared errors displayed in Figure \ref{fig:comp_error},
% I noticed that both approaches experience a similar decrease in error as the reduced sets
% get larger---increase in fractional value. When analyzing the errors experienced for sets between
% $1 - 30\%$ of the original mutant set, both uniform random sampling and stratified random sampling over mutation operators
% encounter a rapid decrease in error, or increase in accuracy.
%
% After performing selective mutation analysis I have observed that all reduced sets from both
% uniform random sampling and stratified random sampling over mutation operators produce sets with mutation scores that are
% considered to be ``highly'' correlated, with the only exception
% being the reduced set produced by random sampling with $1\%$, which produces a reduced set with a mutation score that is considered
% to be weakly correlated to that of the mutation score of the entire set of mutations.
%
% Although we are able to produce
% reduced sets with highly correlated mutation scores, we are not able to produced sets with
% very highly correlated mutation scores using either of the selective approaches being analyzed.
%
% I chose only to provide the visualizations for the Inventory schema because it effectively displays
% the consequences that selective mutation analysis can have on the mutation score. Additionally,
% there are nine schemas that are mutation adequate, meaning that no matter how small the test data
% are, data for the corresponding schema will \textit{always} produce mutation adequate sets. This is what I consider
% to be an ineffective display of the cause of selective mutation analysis.
%
% \begin{figure*}[!ht]
% \centering
% \subfloat[Uniform random sampling errors]{\includegraphics[width=3in]{inventory_error}
% \label{fig:iv_e}}
% \hfil
% \subfloat[Stratified random sampling over mutation operators errors]{\includegraphics[width=3in]{inventory_op_error}
% \label{fig:iv_op_e}}
% \caption{The errors for uniform random sampling and stratified random sampling over operators for the Inventory schema.}
% \label{fig:comp_error}
% \end{figure*}

\section{Conclusion and Future Work}

% Determining the quality of a test suite or database schema is salient.
% By creating higher-quality test suites, in turn, we are able to ensure that the programs
% under test are of high quality. Similarly, ensuring that database schemas are tested makes certain
% that only expected data enter the database. This is important because databases are said to be an organization's single most
% valuable asset \cite{silberschatz1997database} and if it contains invalid data, this asset is unreliable.
%
% Testing the quality of both a program's test suite or a database's schema can be done
% through the use of mutation testing \cite{gopinath2015empirical, mcminn2015effectiveness}. However,
% testing all possible mutations of a program or schema is of great computational expense \cite{offutt1993experimental}.
% Therefore, researches have introduced approaches for reducing the computation cost
% of mutation testing. These approaches have been categorized by
% Offut and Untch as: do \textit{fewer}, do \textit{smarter} and do \textit{faster} \cite{offutt1993experimental}.
%
% Of similar saliency to reducing cost is ensuring that the selective approaches
% produce sets that are as effective at detecting faults as the original
% set of data. In this paper, I analyzed uniform random sampling and stratified random sampling
% over mutation operators and found agreeing results to those of Gopinath et al. regarding the
% ability to, in a stable manner, produce reduced sets with high mutation scores \cite{gopinath2015empirical}.

% In addition, similar to Budd et al. in \cite{budd1980mutation}, I found that reduced sets, as small as $10\%$,
% of the entire set can produce mutation scores that are considered to be highly correlated according
% to the Guildford scale. Correspondingly, the mean absolute and root mean squared errors for these
% reduced sets were low, as to be expected with a highly correlated set.

% In future work, I plan to determine whether the reduced sets with highly correlated mutation scores
% are actually stringent predictors of test suite quality. I am able to determine the quality of
% a set based on fault coverage using the \texttt{Defects4J} tool which provides fault data from mutation
% testing \cite{just2014defects4j}.

% An example of a floating figure using the graphicx package.
% Note that \label must occur AFTER (or within) \caption.
% For figures, \caption should occur after the \includegraphics.
% Note that IEEEtran v1.7 and later has special internal code that
% is designed to preserve the operation of \label within \caption
% even when the captionsoff option is in effect. However, because
% of issues like this, it may be the safest practice to put all your
% \label just after \caption rather than within \caption{}.
%
% Reminder: the "draftcls" or "draftclsnofoot", not "draft", class
% option should be used if it is desired that the figures are to be
% displayed while in draft mode.
%
%\begin{figure}[!t]
%\centering
%\includegraphics[width=2.5in]{myfigure}
% where an .eps filename suffix will be assumed under latex, 
% and a .pdf suffix will be assumed for pdflatex; or what has been declared
% via \DeclareGraphicsExtensions.
%\caption{Simulation results for the network.}
%\label{fig_sim}
%\end{figure}

% Note that the IEEE typically puts floats only at the top, even when this
% results in a large percentage of a column being occupied by floats.


% An example of a double column floating figure using two subfigures.
% (The subfig.sty package must be loaded for this to work.)
% The subfigure \label commands are set within each subfloat command,
% and the \label for the overall figure must come after \caption.
% \hfil is used as a separator to get equal spacing.
% Watch out that the combined width of all the subfigures on a 
% line do not exceed the text width or a line break will occur.
%
%\begin{figure*}[!t]
%\centering
%\subfloat[Case I]{\includegraphics[width=2.5in]{box}%
%\label{fig_first_case}}
%\hfil
%\subfloat[Case II]{\includegraphics[width=2.5in]{box}%
%\label{fig_second_case}}
%\caption{Simulation results for the network.}
%\label{fig_sim}
%\end{figure*}
%
% Note that often IEEE papers with subfigures do not employ subfigure
% captions (using the optional argument to \subfloat[]), but instead will
% reference/describe all of them (a), (b), etc., within the main caption.
% Be aware that for subfig.sty to generate the (a), (b), etc., subfigure
% labels, the optional argument to \subfloat must be present. If a
% subcaption is not desired, just leave its contents blank,
% e.g., \subfloat[].


% An example of a floating table. Note that, for IEEE style tables, the
% \caption command should come BEFORE the table and, given that table
% captions serve much like titles, are usually capitalized except for words
% such as a, an, and, as, at, but, by, for, in, nor, of, on, or, the, to
% and up, which are usually not capitalized unless they are the first or
% last word of the caption. Table text will default to \footnotesize as
% the IEEE normally uses this smaller font for tables.
% The \label must come after \caption as always.
%
%\begin{table}[!t]
%% increase table row spacing, adjust to taste
%\renewcommand{\arraystretch}{1.3}
% if using array.sty, it might be a good idea to tweak the value of
% \extrarowheight as needed to properly center the text within the cells
%\caption{An Example of a Table}
%\label{table_example}
%\centering
%% Some packages, such as MDW tools, offer better commands for making tables
%% than the plain LaTeX2e tabular which is used here.
%\begin{tabular}{|c||c|}
%\hline
%One & Two\\
%\hline
%Three & Four\\
%\hline
%\end{tabular}
%\end{table}


% Note that the IEEE does not put floats in the very first column
% - or typically anywhere on the first page for that matter. Also,
% in-text middle ("here") positioning is typically not used, but it
% is allowed and encouraged for Computer Society conferences (but
% not Computer Society journals). Most IEEE journals/conferences use
% top floats exclusively. 
% Note that, LaTeX2e, unlike IEEE journals/conferences, places
% footnotes above bottom floats. This can be corrected via the
% \fnbelowfloat command of the stfloats package.


% conference papers do not normally have an appendix

% use section* for acknowledgment
% trigger a \newpage just before the given reference
% number - used to balance the columns on the last page
% adjust value as needed - may need to be readjusted if
% the document is modified later
%\IEEEtriggeratref{8}
% The "triggered" command can be changed if desired:
%\IEEEtriggercmd{\enlargethispage{-5in}}

% references section

% can use a bibliography generated by BibTeX as a .bbl file
% BibTeX documentation can be easily obtained at:
% http://mirror.ctan.org/biblio/bibtex/contrib/doc/
% The IEEEtran BibTeX style support page is at:
% http://www.michaelshell.org/tex/ieeetran/bibtex/
\bibliographystyle{IEEEtran}
% argument is your BibTeX string definitions and bibliography database(s)
% \bibliography{bibliography.bib}
%
% <OR> manually copy in the resultant .bbl file
% set second argument of \begin to the number of references
% (used to reserve space for the reference number labels box)
\end{document}


