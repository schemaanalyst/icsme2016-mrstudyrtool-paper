%!TEX root=../icsme2016-mrstudyr.tex

% Define the layers to draw the diagram
\pgfdeclarelayer{background}
\pgfdeclarelayer{foreground}
\pgfsetlayers{background,main,foreground}

% Define block styles used later

\tikzstyle{sensor}=[text width=5em,
    text centered, minimum height=2.5em]
\tikzstyle{darksensor}=[draw, fill=black!10, text width=6em,
    text centered, minimum height=2.5em,drop shadow]

% Define distances for bordering
\def\blockdist{1.5}
\def\edgedist{2.5}

\begin{figure}[t]

  \vspace{-1em}

  \centering
\begin{tikzpicture}[thick,scale=0.85, every node/.style={scale=0.85}]
    \node [] at (0, 5) (map) {\textbf{Mutation Analysis Process}};
    \draw (-4, 4.75) -- (4, 4.75);
    \node [sensor] at (0, 4) (m1) {Generate Mutants};
    \node [sensor] at (0, 2.5) (m2) {Execute Generated Mutants};
    \node [sensor] at (3.25, 3.25) (t1) {Mutation Testing};
    \node [sensor] at (3.25, 1) (t2) {Mutant Analysis};
    \draw (2, 4.4) -- (2, 1.9);
    \draw (1.7, 4.4) -- (2, 4.4);
    \draw (1.7, 1.9) -- (2, 1.9);
    \node [darksensor] at (0, 1) (m3) {Retrospective Analysis};
    \draw (2, 1.7) -- (2, 0.25);
    \draw (1.7, 1.7) -- (2, 1.7);
    \draw (1.7, 0.25) -- (2, 0.25);

    \path [draw, ->] (m1.south) -- node [above] {}
        (m2.90);
    \path [draw, ->] (m2.south) -- node [above] {}
        (m3.90);
    % \path [draw] (m1.east) -| +(0.3,0) |-node [above] {}
    %     (m2.0);
\end{tikzpicture}
\caption{\label{fig:process}The phases of the mutation analysis process.}
\vspace{-1em}
\end{figure}
