%!TEX root=../icsme2016-mrstudyr.tex

\begin{figure}
\centering

\scalebox{\algorithmfigurescalefactor}{
\begin{tabular}{|l|}
\hline

\begin{tabular}{l}
\vspace{-2ex} \\

{\bf function} analyse\_\textit{redudction\_technique\_name}($data$) \\

\atab {\bf where} \\
\atab \atab \tinybullet $data$ is data collected from performing mutation testing \\

\atab {\it // Iterate over all schemas:} \\
\atab {\bf for} $s \in $ list($schemas$) \smallvspace \\

\hspace{1em} \atab {\it // Iterate over the chosen $x$ values:} \\
\hspace{1em} \atab {\bf for} $i \in $ list($x\_values$) \smallvspace \\

\hspace{3em} \atab {\it // For 30 trials:} \\
\hspace{3em} \atab {\bf for} $j \in $ 1:30 \smallvspace \\

\hspace{4em} \atab {\it // Perform reduction technique on data:} \\
\hspace{4em} \atab {select\_k\_percent \it optionally (\_per\_operator)} \smallvspace \\

\hspace{4em} \atab {\it // Perform effectiveness calculations:} \\
\hspace{4em} \atab {analyse\_mutation\_score}\smallvspace \\
\hspace{4em} \atab {analyse\_correlation}\smallvspace \\
\hspace{4em} \atab {analyse\_percents\_error}\smallvspace \\

\hspace{4em} \atab {\it // Append rows to a dataframe:} \\
\hspace{4em} \atab {\bf let} $df \leftarrow$ dataframe\smallvspace \\

\atab {\bf return} $df$ \smallvspace \\

{\bf end function} \\
\vspace{-0.5em} \\

\end{tabular} \\

\hline
\end{tabular}}

% NOTE: I have reduced the length of the main caption and then expanded the subcaption. This is now in line with what is
% in the captions for the graphs, a style which I adapted from the code that CJW originally wrote.


\caption{\label{fig:experiment-structure} High-level experiment function structure.}

\vspace{-.75em}
\end{figure}

