%!TEX root=../icsme2016-mrstudyr.tex

\begin{figure}
\centering

\vspace{-.75em}

\scalebox{\algorithmfigurescalefactor}{
\begin{tabular}{|l|}
\hline

% TODO: Review this algorithm and fix all of the other issues pointed out in the marked version of the paper
% --> How do you initialise d?
% --> How is the variable p incorporated?
% --> Does this algorithm closely follow the structure of the R code?
% --> Does this algorithm use consistent terminology for the variable x?

\begin{tabular}{l}
\vspace{-2ex} \\

{\bf function} analyse ($data$) \\

\atab {\bf where} \\
\atab \atab \tinybullet $\mbox{{\em data}}$ is data collected from performing mutation testing on all mutants \\
\atab \atab \tinybullet $x$ is the maximum fractional threshold for the number of mutants \\
\atab \atab \tinybullet {\em percentages} is the list of chosen $x$ values for mutant sampling \\
\atab \atab \tinybullet $t$ is the trial index of the experiment \\
\atab \atab \tinybullet {\em trials} is the total number of trials set for each configuration of technique \\
\atab \atab \tinybullet {\em red} is the reduced mutant data returned from a single reduction method \\
\atab \atab \tinybullet $d$ is the data from the configuration of a reduction method and $x$ value\\

\vspace*{-.75em} \\

% \hspace{1em} \atab {\bf initialise} $d$ \smallvspace \\

\hspace{1em} \atab $d \leftarrow \emptyset$, $t \leftarrow 0$ \smallvspace \\

% \hspace{1em} \atab $t \leftarrow 0$ \smallvspace \\

\hspace{1em} \atab {\bf for each} $x \in \mbox{{\em percentages}}$ \smallvspace \\

\hspace{3em} \atab {\bf for} $t \leq \mbox{{\em trials}}$ \smallvspace \\

\hspace{4em} \atab {\bf let} $red \leftarrow \mbox{{\em reduction\,technique}}\,(\mbox{{\em data}},\,p)$ \smallvspace \\

\hspace{4em} \atab {\bf let} $d \leftarrow$ {\bf append}($d$, $\mbox{{\em red}}$) \smallvspace \\

\hspace{4em} \atab $t \leftarrow t + 1$ \smallvspace \\

\hspace{1em} \atab {\bf return} $d$ \smallvspace \\

{\bf end function} \\
\vspace{-0.5em} \\

\end{tabular} \\

\hline
\end{tabular}}

\caption{\label{fig:experiment-structure} An experiment function for studying mutant sampling.}

\vspace{-1em}

\end{figure}

