\documentclass[conference]{IEEEtran}

% Packages that Colton added:
\usepackage{tikz}
\usetikzlibrary{shapes,arrows,shadows,positioning,calc}
\usepackage{listings}
\usepackage{xspace}
\lstset{
basicstyle=\small\ttfamily,
columns=flexible,
breaklines=true
}
\usepackage{hhline}

% Textual shortcuts
\newcommand{\mr}{{\it mrstudyr}\xspace}
\newcommand{\MR}{\mr}
\newcommand{\mrstudyr}{\mr}
%%%%%%%%%%%%%%%%%%%%%%%%%%%%%%%%%%%%%%%%%%%%%%%%%%%%%%%%%%%%%%%%%%%

% *** GRAPHICS RELATED PACKAGES ***
\ifCLASSINFOpdf
  \usepackage{graphicx}
  % declare the path(s) where your graphic files are
  \graphicspath{{imgs/}}
  % and their extensions so you won't have to specify these with
  % every instance of \includegraphics
  \DeclareGraphicsExtensions{.pdf}
\else
\fi

% *** SUBFIGURE PACKAGES ***
\usepackage{subfig}

\usepackage{stfloats}

% correct bad hyphenation here
\hyphenation{op-tical net-works semi-conduc-tor}

\begin{document}

\title{\mr: Retrospectively Studying the Effectiveness of Mutant Reduction Techniques}

\author{
\IEEEauthorblockN{Colton J. McCurdy}
\IEEEauthorblockA{Allegheny College}
\and
\IEEEauthorblockN{Phil McMinn}
\IEEEauthorblockA{University of Sheffield}
\and
\IEEEauthorblockN{Gregory M. Kapfhammer}
\IEEEauthorblockA{Allegheny College}}

\maketitle

\begin{abstract}
    Mutation testing is a well-known method for measuring a test suite's quality.
    However, due to its computational expense and demand for human interaction,
    mutation testing is often infeasible in practice. To control the demands of
    mutation testing, various reduction strategies have been proposed.
    Although reduction techniques for mutation testing have been described
    in written form, a tool for retrospectively analysing the performance of
    an approach is not available.
    Instead of integrating every reduction approach into an existing mutation testing system and
    individually executing each one, we apply each approach to the data
    collected from a single execution of the entire mutant set.
    We provide a tool, \mr, which enables this retrospective
    comparison of effectiveness for existing mutation testing reduction techniques.
\end{abstract}

\IEEEpeerreviewmaketitle

\section{Introduction}
Programmers make mistakes---often referred to as errors---when writing software
that can be harmful and even result in human fatality \cite{vicente2003programming}.
Running a suite of test cases helps detect many of the errors
that lead to a crash of a program or that cause a logical failure \cite{wagner2005comparing}.
However, simply testing a program is not enough. In order for the results
from a test suite to be substantive, the test suite needs to be of ``high quality''.
Determining the quality of a test suite illustrates how stringently the given
program was tested.

Mutation testing is a widely-recognized technique for determining the quality
of a test suite \cite{gopinath2015mutation}. While there are an immense number
of potential faults for a program, mutation testing primarily focuses on those
which are close to the correct version with the expectation that they will
be representative of all faults \cite{jia2011analysis}.
Of the software testing methods, mutation testing is widely considered the strongest
test criterion in terms of finding the most faults \cite{ammann2008introduction}.

Although mutation testing is the most effective in detecting faults, it possesses
major drawbacks in the computational cost and amount of human interaction required, making it
impractical to use in practice \cite{gopinath2015mutation, wong1995reducing, gopinath2015empirical}.
The major computational cost of mutation testing comes from executing the large number of mutants generated for a
program against a test suite \cite{offutt1993experimental}. Executing a smaller, representative
set of mutants against the test suite has been proposed as a technique to reduce
the cost of mutation testing \cite{jia2011analysis, wong1995reducing, offutt1993experimental, offutt2001mutation}.
This reduction strategy is categorized by Offut and Untuch as a `do fewer' approach \cite{offutt2001mutation}.

There are a number of mutant reduction techniques that can be categorized as do fewer approaches. Two of these techniques are Mutant Sampling and Selective Mutation \cite{jia2011analysis, wong1995reducing, offutt1993experimental}.
Mutant Sampling is the simplest reduction strategy, where a subset---containing a fraction of all mutants---is
chosen randomly \cite{wong1995reducing}. Selective Mutation, also called ``constrained mutation'',
reduces the number of mutants to be analysed by decreasing the operator count \cite{jia2011analysis, offutt1993experimental, mathur1991performance}.
For each of these approaches, a threshold for the maximum number of mutants is set by an arbitrary \(x\%\).

Work has been done in the realm of mutant reduction techniques to find the smallest $x$ that still produces
a representative set of mutant \cite{jia2011analysis, mathur1994empirical}. Most of the work
to minimize representative sets of mutants has been done by integrating an approach into
mutation testing systems and performing mutation testing on each subset of mutants \cite{demillo1988extended, king1991fortran}.

While this way of evaluating reduction techniques is accurate, it is both
expensive to implement and perform due to the number of times mutation testing
needs to be executed on the reduced sets. Retrospectively studying the data
collected from a single run of mutation testing is superior to implementing each
approach into a mutation testing system because it has minimal upfront costs and obviates the need
to understand the complexities of a target environment.

Since studying the data retrospectively removes the need to fathom the complexities of a target environment,
mutation testing can be extended to new domains (e.g., testing database schemas) \cite{mcminn2016virtual, mcminn2015effectiveness, wright2013efficient}.
Testing that a relational database's schema has correctly specified integrity constraints is important
because these restrictions ensure that only valid data enters the database. However, little work has
sought to test these integrity constraints \cite{mcminn2016virtual}.

Databases are utilized pervasively for storing everything from batting and pitching statistics of baseball
players \cite{lahmanbaseball} to fingerprints, scars and tattoos of criminals \cite{ngi}.
Although enterprise use of non-relational, or ``NoSQL'', databases is becoming more common,
relational databases endure ubiquitously.
Relational databases are an efficient and reliable way to store large amounts of data.
The acceptance of relationships between data entries, as well as strict structural constraints
upheld by the schema are what make relational databases unique.

The database's schema is the artifact responsible for specifying the data to be stored and
how it should be structured into tables. A series of integrity constraints
are what make the database's schema able to ensure values are unique, present,
subject to domain-specific conditions and maintain referential integrity \cite{mcminn2016virtual}.
Although the schema is the last line of defence for the data contained
within a database, little work has sought to test the correctness of the integrity constraints \cite{mcminn2015effectiveness}.

%!TEX root=../icsme2016-mrstudyr.tex

% Define the layers to draw the diagram
\pgfdeclarelayer{background}
\pgfdeclarelayer{foreground}
\pgfsetlayers{background,main,foreground}

% Define block styles used later

\tikzstyle{sensor}=[text width=5em,
    text centered, minimum height=2.5em]
\tikzstyle{darksensor}=[draw, fill=black!10, text width=6em,
    text centered, minimum height=2.5em,drop shadow]

% Define distances for bordering
\def\blockdist{1.5}
\def\edgedist{2.5}

\begin{figure}[t]

  \vspace{-0.75em}

  \centering
\begin{tikzpicture}[thick,scale=0.85, every node/.style={scale=0.85}]
    \node [] at (0, 5) (map) {\textbf{Mutation Analysis Process}};
    \draw (-4, 4.75) -- (4, 4.75);
    \node [sensor] at (0, 4) (m1) {Generate Mutants};
    \node [sensor] at (0, 2.5) (m2) {Execute Generated Mutants};
    \node [sensor] at (3.25, 3.25) (t1) {Mutation Testing};
    \node [sensor] at (3.25, 1) (t2) {Mutant Analysis};
    \draw (2, 4.4) -- (2, 1.9);
    \draw (1.7, 4.4) -- (2, 4.4);
    \draw (1.7, 1.9) -- (2, 1.9);
    \node [darksensor] at (0, 1) (m3) {Retrospective Analysis};
    \draw (2, 1.7) -- (2, 0.25);
    \draw (1.7, 1.7) -- (2, 1.7);
    \draw (1.7, 0.25) -- (2, 0.25);

    \path [draw, ->] (m1.south) -- node [above] {}
        (m2.90);
    \path [draw, ->] (m2.south) -- node [above] {}
        (m3.90);
    % \path [draw] (m1.east) -| +(0.3,0) |-node [above] {}
    %     (m2.0);
\end{tikzpicture}
\caption{\label{fig:process}The phases of the mutation analysis process.}
\vspace{-1em}
\end{figure}


The small amount of work testing the integrity constraints of a database's schema is not
because of a sufficient understanding of the topic. The 941,910 questions posted on
the technical question and answer website, StackExchange, indicate the demand for support
of database schemas.\footnote{http://goo.gl/eZF1mK} Therefore, ensuring that
the schemas that are created are also tested is important in upholding the integrity
of the data contained within the database.

Mutation testing is versatile and can be used for detecting errors made by programmers in many domains.
A domain that mutation testing has been more frequently utilized in recently is testing database
schemas \cite{mcminn2016virtual, mcminn2015effectiveness, wright2013efficient}. However,
again, due to its computation expense, it is not widely adopted in industry. Since we
are able to use mutation testing, we can take advantage of the many
existing reduction approaches \cite{jia2011analysis, wong1995reducing, offutt1993experimental, offutt2001mutation}
as well as this new notion of retrospective analysis.

Retrospective analysis is substantially cheaper than applying an
individual reduction strategy to mutation testing and then evaluating that
approach. This is the case because the expense of mutation testing in
retrospective analysis is incurred only once, rather than for
every reduction technique. Analysing the effectiveness of an approach in a reflective fashion
is possible because we collect all of the data necessary from
the single execution of mutation testing. We introduce the
first and only tool that analyses reduction techniques retrospectively
in \mr.

\mr is a tool for evaluating the effectiveness of mutation analysis reduction techniques
in retrospect. Using data in a generalised format from a single execution of mutation testing, \mr
employs each reduction strategy and performs calculations to determine the
effectiveness. By accepting a generalised input format, \mr is capable of
retrospectively analysing mutant reduction techniques from various domains.
Additionally, \mr is modifiable, well-documented and has been released on
GitHub under and open-source license.

    In summary, the key contributions of this work are as follows:
    \begin{itemize}
        \item A tool to study mutant reduction techniques in retrospect.
        \item A tool that accepts a generalised input format.
        \item A tool released as a free and open-source R package.
    \end{itemize}

\newcommand{\mx}[1]{\mathbf{\bm{#1}}} % Matrix command
\newcommand{\vc}[1]{\mathbf{\bm{#1}}} % Vector command

% Define the layers to draw the diagram
\pgfdeclarelayer{background}
\pgfdeclarelayer{foreground}
\pgfsetlayers{background,main,foreground}

% Define block styles used later

\tikzstyle{sensor}=[draw, fill=black!10, text width=5em,
    text centered, minimum height=2.5em,drop shadow]
\tikzstyle{small}=[draw, fill=black!10, text width=2em,
    text centered, minimum width=2em, minimum height=2.5em,drop shadow]
\tikzstyle{smallcir}=[draw, ellipse, fill=black!10, text width=5em,
    text centered, minimum height=2.5em, drop shadow]
\tikzstyle{circle}=[draw, ellipse, fill=black!10, text width=8em,
    text centered, minimum height=2.5em, drop shadow]
\tikzstyle{func}=[draw, fill=black!20, text width=5em,
    text centered, minimum height=2.5em,drop shadow]
\tikzstyle{mr}=[draw, fill=black!20, text width=5em,
    text centered, minimum height=13em, minimum width = 15em, drop shadow]
\tikzstyle{ann} = [above, text width=5em, text centered]
\tikzstyle{wa} = [sensor, text width=10em, fill=black!30,
    minimum height=6em, rounded corners, drop shadow]
\tikzstyle{sc} = [sensor, text width=13em, fill=red!30,
    minimum height=10em, rounded corners, drop shadow]


% Define distances for bordering
\def\blockdist{1.5}
\def\edgedist{2.5}

\begin{figure}[t]
  \centering
% SIMPLE
\begin{tikzpicture}[thick,scale=0.85, every node/.style={scale=0.85}]
    \node [mr] at (0, 0) (mr) {};
    \node [smallcir] at (0, 3.25) (md) {Mutant Data};
    \node [sensor] at (0, 1.5) (r) {Reduction Approach};
    \node [sensor] at (0, 0) (rmd) {Reduced Mutant Data};
    \node [small] at (-1.75, 0) (m1) {$ms'$};
    \node [small] at (1.75, 0) (m2) {$ms$};
    \node [func] at (0, -1.5) (calc) {Calculations};
    \node [circle] at (0, -3.25) (rec) {Policy Recommendation};

    \path [draw, ->] (md.south) -- node [above] {}
        (r.90);
    \path [draw, ->] (md.east) -| node [above] {}
        (m2.90);
    \path [draw, ->] (r.south) -- node [above] {}
        (rmd.90);
    \path [draw, ->] (rmd.west) -- node [above] {}
        (m1.0);
    \path [draw, ->] (m1.south) |- node [above] {}
        (calc.180);
    \path [draw, ->] (m2.south) |- node [above] {}
        (calc.0);
    \path [draw, ->] (calc.south) -- node [above] {}
        (rec.90);

% VERTICAL
% \begin{tikzpicture}[thick,scale=0.85, every node/.style={scale=0.85}]
%     \node [mr] at (0, 0) (mr) {};
%     % \node [] at (-3, -0.25) (left) {};
%     % \node [] at (0, 2.3) (text) {\textit{mrstudyr}};
%     \node [sensor] at (-1.25, 0) (g) {};
%     \node [sensor] at (1.25, 0) (gg) {};
%     \node [circle] at (0, 4.5) (md) {Mutant Data};
%     \node [circle] at (0, 2.75) (df) {Dataframe};
%     \node [] at (-1.25, 1.75) (a1) {$a_1$};
%     \node [] at (0, 1.75) (dots) {...};
%     \node [] at (1.25, 1.75) (a2) {$a_n$};
%     \node [] at (-1.75, 1) (t1) {$t_1$};
%     \node [] at (-1.25, 1) (dots2) {...};
%     \node [] at (-0.75, 1) (t30) {$t_{30}$};
%     \node [] at (0.75, 1) (tt1) {$t_1$};
%     \node [] at (1.25, 1) (dots3) {...};
%     \node [] at (1.75, 1) (tt30) {$t_{30}$};
%     \node [] at (-1.75, 0) (m1) {$ms_{1}'$};
%     \node [] at (-0.75, 0) (m2) {$ms_{30}'$};
%     \node [] at (0.75, 0) (mm1) {$ms_{1}'$};
%     \node [] at (1.75, 0) (mm2) {$ms_{30}'$};
%     \node [func] at (-1.75, -1.75) (corr) {Calculate Correlation};
%     \node [func] at (1.75, -1.75) (err) {Calculate Error};
%     \node [circle] at (0, -2.8) (df2) {Dataframe};
%
%     \path [draw, ->] (md.south) -- node [above] {}
%         (df.90);
%     \path [draw, ->] (df.south) -- node [above] {}
%         (a1.10);
%     \path [draw, ->] (df.south) -- node [above] {}
%         (a2.170);
%     \path [draw, ->] (a1.south) -- node [above] {}
%         (t1.50);
%     \path [draw, ->] (a1.south) -- node [above] {}
%         (t30.130);
%     \path [draw, ->] (a2.south) -- node [above] {}
%         (tt1.50);
%     \path [draw, ->] (a2.south) -- node [above] {}
%         (tt30.130);
%     \path [draw, ->] (t1.south) -- node [above] {}
%         (m1.90);
%     \path [draw, ->] (t30.south) -- node [above] {}
%         (m2.90);
%     \path [draw, ->] (tt1.south) -- node [above] {}
%         (mm1.90);
%     \path [draw, ->] (tt30.south) -- node [above] {}
%         (mm2.90);
%     \path [draw, ->] (g.south) -| +(0, -0.25) -| node [above] {}
%         (err.90);
%     \path [draw, ->] (gg.south) -| +(0, -0.25) -| node [above] {}
%         (corr.90);
%
\end{tikzpicture}
\caption{\label{fig:mrstudyr}The inputs and outputs of mutation analysis phase
of the mutation analysis process, performed by the \mr \,tool. \\
\textit{The darkest square in the background represents the \mr \,tool.
A rectangle represents a process. A rectangle with rounded edges represents a
calculation performed by \mr. An ellipse represents the output from a process.}}
\end{figure}


\section{\textit{mrstudyr}: Mutant Reduction Studier}

Performing the entire mutation analysis process, displayed in Figure \ref{fig:process},
is both expensive in terms of time and computational requirements. The mutation testing
phase, displayed in Figure \ref{fig:process}, of the mutation analysis
process consists of generating and executing the mutants. The primary
expense of the mutation analysis process is incurred in the mutation testing phase
due to the large number of mutants generated---even for programs with few lines of code \cite{offutt2001mutation}.

The next phase in the mutation analysis process is retrospective analysis. In this
phase, mutants are found to either be dead or alive. Where a mutant is considered
``dead''---or to have been killed---if when executing the modified version of the source
code the output differed from the original version. Conversely, a mutant remains alive
if the output from the modified version of the source code does not differ from the
original version. More importantly, in this phase of the mutation analysis process,
we are able to evaluate the effectiveness of various reduction techniques.

By retrospectively analysing each reduction technique's effectiveness, we are
able to avoid the cost of performing mutation testing more than once. To alleviate
the additional executions, data needs to be collected
from the first run. Displayed in Figure \ref{fig:mrstudyr},
the \mr tool uses the data collected from mutation testing to conduct further analyses
regarding the effectiveness of a reduction technique.

The data, at the very least, needs to include the status of
a mutant after testing (e.g., dead or alive), the type of mutant (e.g., normal, duplicate, equivalent, or stillborn)
and the operator used. In addition to the minimum requirements, the data can include much
more detail about the mutants. An example of additional data that may be collected are the database
management system or schema under test and the cost of generating each mutant.

As indicated previously, the \mr tool requires the status of a mutant after testing to be included
in the data provided.
Based on the number of dead and alive mutants after mutation testing,
the \mr tool calculates and associates a mutation score with a set of mutants.
The mutation score of a mutant set is a value representative of how well
a test suite is at identifying and killing mutants. The \mr tool calculates the
mutation scores for the non-reduced set of mutants and then later for each reduced set
to determine how well a reduced set represents the original set in terms of ability to kill mutants.

In order to calculate the effectiveness of a reduction technique, the technique
must first be exercised on the data collected from mutation testing. The \mr tool
performs the reduction techniques and returns the new, reduced mutant data,
displayed in Figure \ref{fig:mrstudyr}. The effectiveness---how well a reduced set is able to represent
the non-reduced set---of a reduction technique is then evaluated
based on four metrics: mutation score, the correlation of the reduced and non-reduced sets' mutation scores
and the mean absolute and root mean squared error.

DO I FURTHER DISCUSS THESE METRICS HERE?? \\
I think so.

\subsection{Released as a Standardised R Package}
\textit{note: do not inappropriately include too much detail}
talk about why we chose to use R (hitchhiker's guide paper / SBST2016)\\
talk about the importance of free and open-source software (Regression testing workshop paper)\\
talk about why reproducibility is important (Regression testing workshop paper)


\subsection{Generalised Input Format}
not limited to a single domain / extensible
allows for use in old and new domains with little modification
    \subsubsection{application to databases}
    talk about mutation analysis of database schema mutants
    \subsubsection{application to programs}


\section{Implementation}

The R programming language for statistical data analysis was specifically designed to
be a full statistical language for conducting empirical studies. Following its predecessor, the S programming
language, R is a powerful means for data analysts to express computations \cite{ihaka1996r}.
Additionally, in R, the fundamental way to share code is via a package.

An R package includes code, data, documentation, tests and are easy to distribute \cite{wickham2015r}.
Since the \mr tool has been released as an R package, four commands in the R console
is all that is needed to install the tool.
First, \texttt{install.packages("devtools")}, then \texttt{library(devtools)}
to install and load the \texttt{devtools}\cite{devtools} package, respectively.
The \texttt{devtools} package is necessary because it provides a framework for installing
and maintaining the tool. Finally, to obtain the
\mr\cite{mrstudyr} tool itself, use the \texttt{devtools} package to install it
from the popular Git repository hosting service, GitHub\cite{github},
using the following command: {\small\texttt{devtools::install\_github("mccurdyc/mrstudyr")}}
and then load it using \texttt{library(mrstudyr)}.
In addition to being easy to install, the \mr tool accepts a generalised input format making
it extensible to various domains.

Accepting a generalised input format allows the \mr tool to be used to retrospectively analyse mutation testing
data from familiar and also emerging domains, such as databases. Although mutation
testing can be utilized in various domains, similar data can be collected from any
domain (e.g., mutation operators, mutant statuses, type of mutant). While more data
can optionally be collected from mutation testing, the additional data is more
dependent on the domain under observation.

\subsection{Tool Design}

The \mr tool was designed to empirically study the effectiveness of
various reduction techniques. The two reduction techniques mentioned earlier---uniform random sampling
and uniform random sampling over operators---require an arbitrary percentage, $x$, to be chosen as the maximum threshold
for the number of mutants to be analysed.

Where Wong and Mathur in their studies \cite{mathur1994empirical}, \cite{wong1993mutation} conducted
an experiment using random sampling with $x$ from $10\%$ to $40\%$ increasing by steps of $5\%$, we
chose to start $x$ at $10\%$ and increase by intervals of $10\%$ up to $90\%$. By lowering
the granularity of the experiment to $10\%$ intervals instead of
$1\%$ or $5\%$, we are able to reduce the cost of performing the retrospective analysis while observing
similar trends.

While reducing the costs of performing the retrospective analysis were a primary concern, it does not
outweigh producing accurate results. Although we conducted our experiments with $x$ increasing by
$10\%$ intervals, we did not see a significant enough loss in accuracy and in the amount of data collected to make
it worth running the additional trials with the smaller intervals (i.e. $1\%$ or $5\%$).

When it came to determining the number of trials necessary, we followed
the recommendations of Traeger et al. and Arcuri, by running 30 trials for each configuration of a reduction approach \cite{traeger2008nine, arcuri2014hitchhiker}.
The general rule in many fields of science is to at least perform 30 observations to show with high confidence
that the obtained results are statistically significant and to allow for the results to approximate a normal
distribution.


\subsection{Tool Usage}

Releasing an R tool via a package makes installing and loading the tool
a matter of a few commands, but this does not guarantee that the usability
of the tool will be as simple. \mr~was designed to be simplistic,
yet perform stringent empirical analyses on mutant reduction techniques.
To display the results from \mr~analysing mutant reduction strategies for the
testing of real-world database schemas, the following commands will be tailored
toward the data that we collected from performing mutation testing.
The collected data can be read in via the following: {\small\texttt{data <- read\_data("sqlite-avmdefaults")}}.
This function expects the data to be located in the \texttt{inst/extdata} folder
and stored as a comma-separated value file.

Immediately following providing the data to \mr, analysis begins. Currently,
the two most common sub-techniques of mutant sampling are performed using the following:
\texttt{analyse\_random\_sampling(data)} and \texttt{analyse\_across\_operators(data)},
for performing random sampling and random sampling across operators, respectively.
These functions will both return a new dataframe containing the trial, evaluated
percentage, reduced and original numerator and denominator counts,
reduced and original time to generate mutants and the reduced and original mutation scores.
The output from these function is provided in the accompanying README file on the
tool's GitHub page~\cite{tool}. After performing a reduction technique its efficiency
and effectiveness is evaluated based on five calculation functions.

The five functions that evaluate the efficiency and effectiveness are:
mutation score, correlation coefficient, magnitude in reduction of creation cost
(\%) and error---MAE and RMSE---between the reduced and original mutation scores.
The calculation functions are as follows, in order:

\begin{itemize}
    \item {\small\texttt{analyse\_mutation\_score(data)}}
    \item {\small\texttt{analyse\_correlation(data)}}
    \item {\small\texttt{analyse\_reduction(data)}}
    \item {\small\texttt{analyse\_percents\_error(data)}}
\end{itemize}

These functions each return a dataframe including the result of the calculation.
Again, the output from the calculation functions can be found on
the tool's GitHub page~\cite{tool}. Following efficiency and effectiveness calculations
is human examination, only succeeded by the establishment of a policy recommendation.

The human examination phase still requires use of \mr~for clearly visualising the trends
in the data collected from the calculation phase. The \mr~tool takes advantage of the
widely-used graphing package, \texttt{ggplot2}~\cite{ggplot2}. All of the visualisation
functions provided and their output can be found on the GitHub page~\cite{tool}.
These visualisations will help human examiners to construct a policy recommendation as
to which reduction technique should be used for the specific domain and data under observation.



\vspace*{-0.25cm}
\section{\textit{mrstudyr} in Action (to be changed)}

include tables about the mutant data (alive, dead, total, type) \\
provide results from more than just Inventory schema \\
Table including characteristics of the schemas \\
Graph and explaining trends and meaning of everything in graph (QSIC2104)



% The data analyzed in this paper was generated from performing mutation testing on
% 34 database schemas using the SQLite database management system. The data
% contain 118860 mutants, which as a whole, take 67287543 milliseconds to analyze using mutation analysis.
%
% Reducing this cost is the primary focus of this analysis. However, reducing cost by analyzing fewer mutants
% leads to weakly correlated mutation scores and higher mean absolute and root mean squared errors (displayed
% in Figures \ref{fig:comp_corr} and \ref{fig:comp_error}, respectively). In this paper, I will
% analyze the graphs comparing mutation score, error and correlation (on the y-axis) to percentage (on the x-axis),
% specifically for the Inventory schema.
%
% As previously mentioned, the mutation score associated with a test suite or database schema is a value
% corresponding to the quality of the test suite or schema. When reducing the mutant set by fractional values,
% we are able to receive mutation scores that are very close to the original mutation score (e.g., both uniform random sampling
% and stratified random sampling over operators on only $30\%$ of the original mutants produce mutation scores ranging between 0.70 and 0.80, where the
% original mutation score is 0.75). As displayed
% in Figure \ref{fig:comp} and previously noted by Budd et al. in \cite{budd1980mutation}, even as small
% as $10\%$ of the original set can produce mutation scores that are considered to be highly correlated
% ($\tau_b \approx 0.73$ for uniform random sampling and $\tau_b \approx 0.76$ for stratified random sampling over operators).
% The correlation between the mutation scores of both reduced sets from both approaches and the original set for all fractional values---using Kendall's $\tau_b$---can
% observed in Figure \ref{fig:comp_corr}.
%
% Inozemtseva and Holmes adopt the Guildford scale to describe correlation values. I will similarly
% describe correlations between sets' mutation score using the same scale.
% The Guildford scale describes the absolute value of a coefficient as ``low'' when it is less than 0.4,
% ``moderate'' when it is between 0.4 and 0.7, ``high'' when ranging from 0.7 to 0.9, and ``very high''
% when it is greater than 0.9 \cite{inozemtseva2014coverage}.
%
% Finally, when observing mean absolute and root mean squared errors displayed in Figure \ref{fig:comp_error},
% I noticed that both approaches experience a similar decrease in error as the reduced sets
% get larger---increase in fractional value. When analyzing the errors experienced for sets between
% $1 - 30\%$ of the original mutant set, both uniform random sampling and stratified random sampling over mutation operators
% encounter a rapid decrease in error, or increase in accuracy.
%
% After performing selective mutation analysis I have observed that all reduced sets from both
% uniform random sampling and stratified random sampling over mutation operators produce sets with mutation scores that are
% considered to be ``highly'' correlated, with the only exception
% being the reduced set produced by random sampling with $1\%$, which produces a reduced set with a mutation score that is considered
% to be weakly correlated to that of the mutation score of the entire set of mutations.
%
% Although we are able to produce
% reduced sets with highly correlated mutation scores, we are not able to produced sets with
% very highly correlated mutation scores using either of the selective approaches being analyzed.
%
% I chose only to provide the visualizations for the Inventory schema because it effectively displays
% the consequences that selective mutation analysis can have on the mutation score. Additionally,
% there are nine schemas that are mutation adequate, meaning that no matter how small the test data
% are, data for the corresponding schema will \textit{always} produce mutation adequate sets. This is what I consider
% to be an ineffective display of the cause of selective mutation analysis.
%
% \begin{figure*}[!ht]
% \centering
% \subfloat[Uniform random sampling errors]{\includegraphics[width=3in]{inventory_error}
% \label{fig:iv_e}}
% \hfil
% \subfloat[Stratified random sampling over mutation operators errors]{\includegraphics[width=3in]{inventory_op_error}
% \label{fig:iv_op_e}}
% \caption{The errors for uniform random sampling and stratified random sampling over operators for the Inventory schema.}
% \label{fig:comp_error}
% \end{figure*}


\section{Related Work}
They ...
I ...
They ...
I ...


\section{Conclusions and Future Work}

Although mutation testing is well-recognized as a way to assess test suite
quality, is often too costly to practically use. As such, various techniques
have been developed to reduce the cost of mutation testing. Performing these
reduction techniques in the past has required researchers and experimenters to
incorporate a reduction method into an, often complex, mutation testing tool.
\mr~alleviates the burden of implementing each approach by retrospectively
analysing reduction techniques, a potentially more cost-effective way.

By retrospectively analysing the data collected from prior analysis of all
mutants, the \mr~tool is able to reduce the upfront costs and obviate the need
for researchers and industrialists to understand the domain complexities
associated with implementing an approach into a mutation testing system.
Furthermore, \mr~provides an easy-to-use and rapid way to assess the efficiency
and effectiveness of mutant reduction methods. In addition to being detailed in
this paper, \mr~has been released under an open-source license on a GitHub site
that features extensive documentation and a screencast~\cite{tool}. In future
work, we plan to extend the functionality of \mr~by integrating additional
mutant reduction techniques, thereby allowing for a more comprehensive
comparison of the techniques' efficiency and effectiveness.



\bibliographystyle{IEEEtran}
\bibliography{IEEEabrv,bibliography}
\end{document}
