%% bare_conf.tex
%% V1.4b
%% 2015/08/26
%% by Michael Shell
%% See:
%% http://www.michaelshell.org/
%% for current contact information.
%%
%% This is a skeleton file demonstrating the use of IEEEtran.cls
%% (requires IEEEtran.cls version 1.8b or later) with an IEEE
%% conference paper.
%%
%% Support sites:
%% http://www.michaelshell.org/tex/ieeetran/
%% http://www.ctan.org/pkg/ieeetran
%% and
%% http://www.ieee.org/

%%*************************************************************************
%% Legal Notice:
%% This code is offered as-is without any warranty either expressed or
%% implied; without even the implied warranty of MERCHANTABILITY or
%% FITNESS FOR A PARTICULAR PURPOSE! 
%% User assumes all risk.
%% In no event shall the IEEE or any contributor to this code be liable for
%% any damages or losses, including, but not limited to, incidental,
%% consequential, or any other damages, resulting from the use or misuse
%% of any information contained here.
%%
%% All comments are the opinions of their respective authors and are not
%% necessarily endorsed by the IEEE.
%%
%% This work is distributed under the LaTeX Project Public License (LPPL)
%% ( http://www.latex-project.org/ ) version 1.3, and may be freely used,
%% distributed and modified. A copy of the LPPL, version 1.3, is included
%% in the base LaTeX documentation of all distributions of LaTeX released
%% 2003/12/01 or later.
%% Retain all contribution notices and credits.
%% ** Modified files should be clearly indicated as such, including  **
%% ** renaming them and changing author support contact information. **
%%*************************************************************************


% *** Authors should verify (and, if needed, correct) their LaTeX system  ***
% *** with the testflow diagnostic prior to trusting their LaTeX platform ***
% *** with production work. The IEEE's font choices and paper sizes can   ***
% *** trigger bugs that do not appear when using other class files.       ***                          ***
% The testflow support page is at:
% http://www.michaelshell.org/tex/testflow/



\documentclass[conference]{IEEEtran}
% Some Computer Society conferences also require the compsoc mode option,
% but others use the standard conference format.
%
% If IEEEtran.cls has not been installed into the LaTeX system files,
% manually specify the path to it like:
% \documentclass[conference]{../sty/IEEEtran}

%%%%%%%%%%%%%%%%%%%%%%%%%%%%%%%%%%%%%%%%%%%%%%%%%%%%%%%%%%%%%%%%%%%
% Packages that Colton added:
\usepackage{tikz}
\usetikzlibrary{shapes,arrows,shadows,positioning,calc}
\usepackage{listings}
\usepackage{xspace}
\lstset{
basicstyle=\small\ttfamily,
columns=flexible,
breaklines=true
}
\usepackage{hhline}

% Textual shortcuts
\newcommand{\mr}{{\it mrstudyr}\xspace}
\newcommand{\MR}{\mr}
\newcommand{\mrstudyr}{\mr}
%%%%%%%%%%%%%%%%%%%%%%%%%%%%%%%%%%%%%%%%%%%%%%%%%%%%%%%%%%%%%%%%%%%

% Some very useful LaTeX packages include:
% (uncomment the ones you want to load)

% *** MISC UTILITY PACKAGES ***
%
%\usepackage{ifpdf}
% Heiko Oberdiek's ifpdf.sty is very useful if you need conditional
% compilation based on whether the output is pdf or dvi.
% usage:
% \ifpdf
%   % pdf code
% \else
%   % dvi code
% \fi
% The latest version of ifpdf.sty can be obtained from:
% http://www.ctan.org/pkg/ifpdf
% Also, note that IEEEtran.cls V1.7 and later provides a builtin
% \ifCLASSINFOpdf conditional that works the same way.
% When switching from latex to pdflatex and vice-versa, the compiler may
% have to be run twice to clear warning/error messages.

% *** CITATION PACKAGES ***
%
%\usepackage{cite}
% cite.sty was written by Donald Arseneau
% V1.6 and later of IEEEtran pre-defines the format of the cite.sty package
% \cite{} output to follow that of the IEEE. Loading the cite package will
% result in citation numbers being automatically sorted and properly
% "compressed/ranged". e.g., [1], [9], [2], [7], [5], [6] without using
% cite.sty will become [1], [2], [5]--[7], [9] using cite.sty. cite.sty's
% \cite will automatically add leading space, if needed. Use cite.sty's
% noadjust option (cite.sty V3.8 and later) if you want to turn this off
% such as if a citation ever needs to be enclosed in parenthesis.
% cite.sty is already installed on most LaTeX systems. Be sure and use
% version 5.0 (2009-03-20) and later if using hyperref.sty.
% The latest version can be obtained at:
% http://www.ctan.org/pkg/cite
% The documentation is contained in the cite.sty file itself.

% *** GRAPHICS RELATED PACKAGES ***
%
\ifCLASSINFOpdf
  \usepackage{graphicx}
  % declare the path(s) where your graphic files are
  \graphicspath{{imgs/}}
  % and their extensions so you won't have to specify these with
  % every instance of \includegraphics
  \DeclareGraphicsExtensions{.pdf}
\else
  % or other class option (dvipsone, dvipdf, if not using dvips). graphicx
  % will default to the driver specified in the system graphics.cfg if no
  % driver is specified.
  % \usepackage[dvips]{graphicx}
  % declare the path(s) where your graphic files are
  % \graphicspath{{../eps/}}
  % and their extensions so you won't have to specify these with
  % every instance of \includegraphics
  % \DeclareGraphicsExtensions{.eps}
\fi
% graphicx was written by David Carlisle and Sebastian Rahtz. It is
% required if you want graphics, photos, etc. graphicx.sty is already
% installed on most LaTeX systems. The latest version and documentation
% can be obtained at: 
% http://www.ctan.org/pkg/graphicx
% Another good source of documentation is "Using Imported Graphics in
% LaTeX2e" by Keith Reckdahl which can be found at:
% http://www.ctan.org/pkg/epslatex
%
% latex, and pdflatex in dvi mode, support graphics in encapsulated
% postscript (.eps) format. pdflatex in pdf mode supports graphics
% in .pdf, .jpeg, .png and .mps (metapost) formats. Users should ensure
% that all non-photo figures use a vector format (.eps, .pdf, .mps) and
% not a bitmapped formats (.jpeg, .png). The IEEE frowns on bitmapped formats
% which can result in "jaggedy"/blurry rendering of lines and letters as
% well as large increases in file sizes.
%
% You can find documentation about the pdfTeX application at:
% http://www.tug.org/applications/pdftex

% *** MATH PACKAGES ***
%
%\usepackage{amsmath}
% A popular package from the American Mathematical Society that provides
% many useful and powerful commands for dealing with mathematics.
%
% Note that the amsmath package sets \interdisplaylinepenalty to 10000
% thus preventing page breaks from occurring within multiline equations. Use:
%\interdisplaylinepenalty=2500
% after loading amsmath to restore such page breaks as IEEEtran.cls normally
% does. amsmath.sty is already installed on most LaTeX systems. The latest
% version and documentation can be obtained at:
% http://www.ctan.org/pkg/amsmath

% *** SPECIALIZED LIST PACKAGES ***
%
%\usepackage{algorithmic}
% algorithmic.sty was written by Peter Williams and Rogerio Brito.
% This package provides an algorithmic environment fo describing algorithms.
% You can use the algorithmic environment in-text or within a figure
% environment to provide for a floating algorithm. Do NOT use the algorithm
% floating environment provided by algorithm.sty (by the same authors) or
% algorithm2e.sty (by Christophe Fiorio) as the IEEE does not use dedicated
% algorithm float types and packages that provide these will not provide
% correct IEEE style captions. The latest version and documentation of
% algorithmic.sty can be obtained at:
% http://www.ctan.org/pkg/algorithms
% Also of interest may be the (relatively newer and more customizable)
% algorithmicx.sty package by Szasz Janos:
% http://www.ctan.org/pkg/algorithmicx

% *** ALIGNMENT PACKAGES ***
%
%\usepackage{array}
% Frank Mittelbach's and David Carlisle's array.sty patches and improves
% the standard LaTeX2e array and tabular environments to provide better
% appearance and additional user controls. As the default LaTeX2e table
% generation code is lacking to the point of almost being broken with
% respect to the quality of the end results, all users are strongly
% advised to use an enhanced (at the very least that provided by array.sty)
% set of table tools. array.sty is already installed on most systems. The
% latest version and documentation can be obtained at:
% http://www.ctan.org/pkg/array

% IEEEtran contains the IEEEeqnarray family of commands that can be used to
% generate multiline equations as well as matrices, tables, etc., of high
% quality.

% *** SUBFIGURE PACKAGES ***
%\ifCLASSOPTIONcompsoc
 \usepackage{subfig}
%\else
%  \usepackage[caption=false,font=footnotesize]{subfig}
%\fi
% subfig.sty, written by Steven Douglas Cochran, is the modern replacement
% for subfigure.sty, the latter of which is no longer maintained and is
% incompatible with some LaTeX packages including fixltx2e. However,
% subfig.sty requires and automatically loads Axel Sommerfeldt's caption.sty
% which will override IEEEtran.cls' handling of captions and this will result
% in non-IEEE style figure/table captions. To prevent this problem, be sure
% and invoke subfig.sty's "caption=false" package option (available since
% subfig.sty version 1.3, 2005/06/28) as this is will preserve IEEEtran.cls
% handling of captions.
% Note that the Computer Society format requires a larger sans serif font
% than the serif footnote size font used in traditional IEEE formatting
% and thus the need to invoke different subfig.sty package options depending
% on whether compsoc mode has been enabled.
%
% The latest version and documentation of subfig.sty can be obtained at:
% http://www.ctan.org/pkg/subfig

% *** FLOAT PACKAGES ***
%
%\usepackage{fixltx2e}
% fixltx2e, the successor to the earlier fix2col.sty, was written by
% Frank Mittelbach and David Carlisle. This package corrects a few problems
% in the LaTeX2e kernel, the most notable of which is that in current
% LaTeX2e releases, the ordering of single and double column floats is not
% guaranteed to be preserved. Thus, an unpatched LaTeX2e can allow a
% single column figure to be placed prior to an earlier double column
% figure.
% Be aware that LaTeX2e kernels dated 2015 and later have fixltx2e.sty's
% corrections already built into the system in which case a warning will
% be issued if an attempt is made to load fixltx2e.sty as it is no longer
% needed.
% The latest version and documentation can be found at:
% http://www.ctan.org/pkg/fixltx2e

\usepackage{stfloats}
% stfloats.sty was written by Sigitas Tolusis. This package gives LaTeX2e
% the ability to do double column floats at the bottom of the page as well
% as the top. (e.g., "\begin{figure*}[!b]" is not normally possible in
% LaTeX2e). It also provides a command:
%\fnbelowfloat
% to enable the placement of footnotes below bottom floats (the standard
% LaTeX2e kernel puts them above bottom floats). This is an invasive package
% which rewrites many portions of the LaTeX2e float routines. It may not work
% with other packages that modify the LaTeX2e float routines. The latest
% version and documentation can be obtained at:
% http://www.ctan.org/pkg/stfloats
% Do not use the stfloats baselinefloat ability as the IEEE does not allow
% \baselineskip to stretch. Authors submitting work to the IEEE should note
% that the IEEE rarely uses double column equations and that authors should try
% to avoid such use. Do not be tempted to use the cuted.sty or midfloat.sty
% packages (also by Sigitas Tolusis) as the IEEE does not format its papers in
% such ways.
% Do not attempt to use stfloats with fixltx2e as they are incompatible.
% Instead, use Morten Hogholm'a dblfloatfix which combines the features
% of both fixltx2e and stfloats:
%
% \usepackage{dblfloatfix}
% The latest version can be found at:
% http://www.ctan.org/pkg/dblfloatfix

% *** PDF, URL AND HYPERLINK PACKAGES ***
%
%\usepackage{url}
% url.sty was written by Donald Arseneau. It provides better support for
% handling and breaking URLs. url.sty is already installed on most LaTeX
% systems. The latest version and documentation can be obtained at:
% http://www.ctan.org/pkg/url
% Basically, \url{my_url_here}.

% *** Do not adjust lengths that control margins, column widths, etc. ***
% *** Do not use packages that alter fonts (such as pslatex).         ***
% There should be no need to do such things with IEEEtran.cls V1.6 and later.
% (Unless specifically asked to do so by the journal or conference you plan
% to submit to, of course. )

% correct bad hyphenation here
\hyphenation{op-tical net-works semi-conduc-tor}

\begin{document}
%
% paper title
% Titles are generally capitalized except for words such as a, an, and, as,
% at, but, by, for, in, nor, of, on, or, the, to and up, which are usually
% not capitalized unless they are the first or last word of the title.
% Linebreaks \\ can be used within to get better formatting as desired.
% Do not put math or special symbols in the title.
\title{\mr: Retrospectively Studying the Effectiveness of Mutant Reduction Techniques}

\author{
\IEEEauthorblockN{Colton J. McCurdy}
\IEEEauthorblockA{Allegheny College}
\and
\IEEEauthorblockN{Phil McMinn}
\IEEEauthorblockA{University of Sheffield}
\and
\IEEEauthorblockN{Gregory M. Kapfhammer}
\IEEEauthorblockA{Allegheny College}}

% make the title area
\maketitle

% As a general rule, do not put math, special symbols or citations
% in the abstract
\begin{abstract}
    Mutation testing is a well-known method for measuring a test suite's quality.
    However, due to its computational expense and demand for human interaction,
    mutation testing is often infeasible in practice. To control the demands of
    mutation testing, various reduction strategies have been proposed.
    Although reduction techniques for mutation testing have been described
    in written form, a tool for retrospectively analysing the performance of
    an approach is not available.
    Instead of integrating every reduction approach into an existing mutation testing system and
    individually executing each one, we apply each approach to the data
    collected from a single execution of the entire mutant set.
    We provide a tool, \mr, which enables this retrospective
    comparison of effectiveness for existing mutation testing reduction techniques.

\end{abstract}

% creates the second title. It will be ignored for other modes.
\IEEEpeerreviewmaketitle

%!TEX root=../icsme2016-mrstudyr.tex

\section{Introduction}

% Determining the quality of a test suite illustrates how stringently the given program was tested.

% Introduce software testing

Software developers may introduce errors into a program's source code that could result in a human
fatality~\cite{vicente2003programming}. Running a set of tests, frequently called a test suite, often aids in detecting
the faults that cause a program to function incorrectly~\cite{Kapfhammer2010}. Yet, simply testing a program is not
sufficient. In order for testing to establish a confidence in the correctness of the program under test, the test suite
needs to be of a high level of quality~\cite{Kapfhammer2004}.

% Introduce mutation testing and its benefits

Mutation testing is a widely-recognized technique for assessing the quality of a test suite~\cite{gopinath2015mutation}.
While there are many potential faults for a program, mutation testing focuses on those that are ``close'' to the correct
version, with the expectation that they will be representative of all faults~\cite{jia2011analysis}. Of the methods for
evaluating test quality, mutation testing is widely considered the strongest test criterion in terms of its capability
to necessitate the creation of tests that find many faults~\cite{ammann2008introduction}.

% Talk about the limitations of mutation testing

Although mutation testing effectively requires tests to detect faults, it has major drawbacks in its computational cost
and the amount of necessary human interaction, often making it impractical to use~\cite{gopinath2015mutation, Just2011a,
gopinath2015empirical}. A major computational cost of mutation testing comes from executing each test case in a test
suite for the many generated mutants~\cite{Just2012b, offutt1993experimental}. Executing a small, representative set of
mutants against the test suite has previously been proposed as a technique to reduce the cost of mutation
testing~\cite{jia2011analysis, wong1995reducing, offutt1993experimental}; this reduction strategy is categorized by
Offutt and Untch as a ``do fewer'' approach~\cite{offutt2001mutation}.

% Introduce the technique call random mutant sampling

% Due to its simplicity, mutant sampling is considered to be the most cost-effective reduction technique by avoiding
% implementation complexities~\cite{gopinath2015mutation}.  Although it is simple to implement, mutant sampling has still
% been empirically shown to outperform other techniques~\cite{zhang2010operator}.

% TODO: Before the paragraph used operator-based selection in a way that seemed to be incorrect; please check this new
% content. I think that only zhang2013 actually explains this point (and they call it something slightly different).

There are several mutant reduction techniques in the ``do fewer'' category, with mutant sampling being a simple method
that randomly selects a subset of all mutants~\cite{wong1995reducing}. In addition to being conceptually
simple~\cite{gopinath2015mutation}, mutant sampling has been experimentally shown to outperform other more sophisticated
methods~\cite{zhang2010operator}. Two sub-techniques within mutant sampling are called uniform random sampling and
sampling over operators~\cite{gopinath2015mutation, gopinath2015empirical, Zhang2013}. For both of these sub-techniques, a
threshold for the maximum percentage of selected mutants is set to $x$, which is then either applied to the entire set
of mutants or to each set of mutants produced by an operator~\cite{gopinath2015mutation, gopinath2015empirical,
Zhang2013}.

% GMK NOTE: Cut this sentence as it is not precisely the point that we want to main in this paper

% While this way of evaluating reduction techniques is accurate, it is both expensive to implement and perform due to the
% number of times mutation testing needs to be executed.

% GMK NOTE: Here is the source code analysis of the PIT tool using the cloc program (see script in the bin/ directory)

% ./cloc-git https://github.com/hcoles/pitest
% Cloning into 'temp-linecount-repo'...
% remote: Counting objects: 1240, done.
% remote: Compressing objects: 100% (931/931), done.
% remote: Total 1240 (delta 297), reused 557 (delta 122), pack-reused 0
% Receiving objects: 100% (1240/1240), 583.98 KiB | 0 bytes/s, done.
% Resolving deltas: 100% (297/297), done.
% Checking connectivity... done.
% ('temp-linecount-repo' will be deleted automatically)

     % 843 text files.
     % 833 unique files.
     %  63 files ignored.

% http://cloc.sourceforge.net v 1.60  T=1.79 s (456.9 files/s, 36462.9 lines/s)
% -------------------------------------------------------------------------------
% Language                     files          blank        comment           code
% -------------------------------------------------------------------------------
% Java                           780          11126           8087          43030
% Maven                           29            136             23           2287
% XML                              3             33             69            377
% CSS                              1              4              0             36
% Groovy                           2             10              0             32
% YAML                             1              4              1             21
% Bourne Shell                     2              0              0              2
% -------------------------------------------------------------------------------
% SUM:                           818          11313           8180          45785
% -------------------------------------------------------------------------------

% Discuss the challenges associated with these types of analyses, drawing on the code example of PIT

Prior work has found the smallest value of $x$ that still produces a representative set of
mutants~\cite{jia2011analysis, mathur1994empirical}. Yet, these efforts normally required the experimenters to integrate
a reduction technique into an existing mutation testing system before performing a mutation testing
experiment~\cite{demillo1988extended, king1991fortran}. Since mutation testing tools are often complex --- according to
the Count Lines of Code ({\tt cloc}) tool the PIT mutation testing system contains over 43,000 lines of non-commented
Java code and thousands of lines of build and configuration files --- this approach to studying mutant reduction methods
has a high upfront cost. That is, researchers in this field must grasp the complexities of a mutation testing tool
before they can experimentally evaluate new techniques for mutant reduction.

% Retrospectively studying the data collected from a single execution of mutation testing is superior because of its
% minimal upfront costs and the implementation complexities of a target environment are obviated.

% Retrospective analysis is more cost-effective than applying a mutant reduction technique and then evaluating it.  This
% is the case because the expense of mutation testing in retrospective analysis is incurred only once, rather than for
% every reduction technique. Analysing the effectiveness of an approach retrospectively is possible because the necessary
% data can be collected from a prior analysis of all mutants.  This paper introduces \mr, the first and only tool that
% analyses reduction techniques retrospectively.

% GMK NOTE: Even this sentence is not really needed!

% This paper advocates a new way to empirically study mutant reduction methods.

% Introduce the idea of retrospectively studying a mutant reduction method

As a means for obviating the need for researchers to grasp a complex mutation testing system, this paper advances the
idea of retrospectively studying mutant reducers.  After using a tool like PIT to collect data about which operators ran
and what mutants they produced, a retrospective analysis applies strategies like uniform random sampling to the mutant
data, thereby quickly facilitating an understanding of a reduction method's trade-offs. Only after researchers
comprehend how the mutant reducers work in the intended domain must they then grapple with the complexities of the
chosen tool.

% TODO: Is there a place in the paper where we support this claim of the tool being well-documented?

% Okay, now introduce the tool and explain why it is novel

Since retrospective analysis still requires tool support, this paper presents \mr, a tool for evaluating mutant
reduction techniques in retrospect. Accepting data in a generalised format from a single run of a mutation testing
system, \mr~applies mutant reduction strategies and calculates their efficiency and effectiveness. In addition to being
capable of retrospectively analysing mutant reduction techniques from various domains, \mr~is well-documented and has
been released on GitHub under an open-source license.

% GMK NOTE: I have cut all of this content, using some of it in the new streamlined content

% Databases are utilized pervasively for storing everything from batting and pitching statistics of baseball
% players~\cite{lahmanbaseball} to fingerprints and scars of criminals~\cite{ngi}.  Relational databases are an
% efficient and reliable way to store large amounts of data.  The acceptance of relationships between data entries, as
% well as strict structural constraints upheld by the schema are what make relational databases unique.

% The little work testing the integrity constraints of a schema is not because of a sufficient understanding of the topic.
% database schemas~\footnote{http://goo.gl/eZF1mK}.

% Therefore, ensuring that the schemas that are created are also tested is important in upholding the integrity of the
% data contained within the database.

% The database's schema is the artefact responsible for specifying the data to be stored and how it should be structured
% into tables. A series of integrity constraints is what make the database's schema able to ensure values are unique,
% present, subject to domain-specific conditions and maintain referential integrity~\cite{mcminn2016virtual}.  Although
% the schema is the last line of defence for the data contained within a database, little work has sought to test the
% correctness of the integrity constraints~\cite{mcminn2015effectiveness}.

% Using the tool allows you to study mutant reduction in new domains, so go ahead and introduce databases

As studying the mutant data retrospectively removes the need to comprehend the complexities of a target environment,
mutant reduction methods can be extended to new domains such as that of relational database
schemas~\cite{mcminn2016virtual, mcminn2015effectiveness, wright2013efficient}. Ensuring that a database's schema has
correctly specified integrity constraints is important because these entities ensure that only valid data enters the
database. Even though there are \numquestions~questions about databases on StackExchange, the technical question and
answer website~\cite{stackexchange}, little prior work has focused on testing these integrity
constraints~\cite{mcminn2016virtual}.

% A domain that mutation testing has become more prominent in recently is testing database
% schemas~\cite{mcminn2016virtual, mcminn2015effectiveness, wright2013efficient}. However, due to its computation expense,
% it is not widely adopted in industry to test database schemas.  Although mutation testing is used in a variety of
% domains, the reduction techniques can still be utilized, including retrospective analysis~\cite{jia2011analysis,
%   wong1995reducing, offutt1993experimental, offutt2001mutation}.

% GMK NOTE: I have ultimately added in the reference instead of the footnote

% of database schemas~\cite{stackexchange}. Therefore, ensuring that

Since it is important to assess the quality of tests for relational database schemas, recent work has proposed and
evaluated database-aware mutation analysis techniques~\cite{mcminn2016virtual, mcminn2015effectiveness,
wright2013efficient}. Although the presented method and tool are general, this paper illustrates the retrospective study
of mutant reducers and the use of \mr~in the area of mutation analysis for database schemas. In addition to describing
the implementation of \mr~and overviewing results from applying it to the retrospective study of database schema
mutation, this paper inaugurates the public release of this analysis tool. In summary, the key contributions of this
paper are as follows:

\begin{itemize}

  % Although this is a tool paper, perhaps it is acceptable to claim this ideal as a contribution?

  % \item A tool that supports performing mutant reduction methods retrospectively, a way to quickly identify trade-offs
      % without having to understand the complexities of a mutation testing tool.

  % Make a point about the tool and the features that it provides

  % \item A tool that accepts a generalised input format.
  % \item A tool released as a free and open-source R package.

  \item A well-documented and easy-to-use tool, \mr, that:

      \begin{itemize}

          \item supports using mutant reduction methods retrospectively as a way to study trade-offs in efficiency and
            effectiveness without having to understand the implementation of a complex mutation testing tool.

          \item accepts a generalised input format, is extendible to various domains, and is released as a free and
            open-source package in the R programming language.

    \end{itemize}

  % The experimental results (even though preliminary) use real-world database schemas

  \item Using database schemas taken from real-world database-centric applications, preliminary results from using
    \mr, highlighting the benefits of mutant reduction and the ease with which these results may be obtained.

\end{itemize}

%!TEX root=../icsme2016-mrstudyr.tex

% Define the layers to draw the diagram
\pgfdeclarelayer{background}
\pgfdeclarelayer{foreground}
\pgfsetlayers{background,main,foreground}

% Define block styles used later

\tikzstyle{sensor}=[text width=5em,
    text centered, minimum height=2.5em]
\tikzstyle{darksensor}=[draw, fill=black!10, text width=6em,
    text centered, minimum height=2.5em,drop shadow]

% Define distances for bordering
\def\blockdist{1.5}
\def\edgedist{2.5}

\begin{figure}[t]
  \centering
\begin{tikzpicture}[thick,scale=0.85, every node/.style={scale=0.85}]
    \node [] at (0, 5) (map) {\textbf{Mutation Analysis Process}};
    \draw (-4, 4.75) -- (4, 4.75);
    \node [sensor] at (0, 4) (m1) {Generate Mutants};
    \node [sensor] at (0, 2.5) (m2) {Execute Generated Mutants};
    \node [sensor] at (3.25, 3.25) (t1) {Mutation Testing};
    \node [sensor] at (3.25, 1) (t2) {Retrospective Analysis};
    \draw (2, 4.4) -- (2, 1.9);
    \draw (1.7, 4.4) -- (2, 4.4);
    \draw (1.7, 1.9) -- (2, 1.9);
    \node [darksensor] at (0, 1) (m3) {Retrospective Analysis};
    \draw (2, 1.7) -- (2, 0.25);
    \draw (1.7, 1.7) -- (2, 1.7);
    \draw (1.7, 0.25) -- (2, 0.25);

    \path [draw, ->] (m1.south) -- node [above] {}
        (m2.90);
    \path [draw, ->] (m2.south) -- node [above] {}
        (m3.90);
    % \path [draw] (m1.east) -| +(0.3,0) |-node [above] {}
    %     (m2.0);
\end{tikzpicture}
\caption{\label{fig:process}The phases of the mutation analysis process.}
\vspace{-1.5em}
\end{figure}



%!TEX root=../icsme2016-mrstudyr.tex

\newcommand{\mx}[1]{\mathbf{\bm{#1}}} % Matrix command
\newcommand{\vc}[1]{\mathbf{\bm{#1}}} % Vector command

% Define the layers to draw the diagram
\pgfdeclarelayer{background}
\pgfdeclarelayer{foreground}
\pgfsetlayers{background,main,foreground}

% Define block styles used later

\tikzstyle{sensor}=[draw, fill=black!5, text width=15em,
text centered, minimum height=2.5em,drop shadow]
\tikzstyle{smallsensor}=[draw, fill=black!5, text width=4em,
text centered, minimum height=2.5em,drop shadow]
\tikzstyle{box}=[draw, text width=8em,
text centered, minimum width=17.5em, minimum height=3.5em]
\tikzstyle{calc}=[draw, fill=black!5, text width=2.5em,
text centered, rounded corners, minimum height=2.5em,drop shadow]
\tikzstyle{circle}=[draw, ellipse, fill=black!5, text width=11em,
text centered, minimum height=2.5em, drop shadow]
\tikzstyle{mr}=[draw, fill=black!20, text width=5em,
text centered, minimum height=17em, minimum width = 18em, drop shadow]

% Define distances for bordering
\def\blockdist{1.5}
\def\edgedist{2.5}

\begin{figure}[t]

  \vspace{-.75em}

  \centering
  % SIMPLE
  \begin{tikzpicture}[thick,scale=0.85, every node/.style={scale=0.85}]
    \node [mr] at (0, 0) (mr) {};
    \node [circle] at (0, 3.7) (md) {Original Data};
    \node [sensor] at (0, 2.25) (r) {Reduction Techniques};
    \node [circle] at (0, 0.75) (rmd) {Cumulated Reduced Data};
    \node [sensor] at (0, -0.75) (ra) {Efficiency \& Effectiveness Analysis};
    \node [box] at (0, -2.25) (box) {};
    \node [calc] at (-2.25, -2.25) (ms) {MS};
    \node [calc] at (-0.75, -2.25) (err) {Red.};
    \node [calc] at (0.75, -2.25) (corr) {Corr.};
    \node [calc] at (2.25, -2.25) (err) {Err.};
    \node [sensor] at (0, -3.75) (he) {Human Examination};
    \node [circle] at (0, -5.25) (pr) {Policy Recommendation};

    \path [draw, ->] (md.south) -- node [above] {}
      (r.90);
    \path [draw, ->] (r.south) -- node [above] {}
      (rmd.90);
    \path [draw, ->] (rmd.south) -- node [above] {}
      (ra.90);
    \path [draw, ->] (ra.south) --  node [above] {}
      (box.90);
    \path [draw, ->] (box.south) --  node [above] {}
      (he.90);
    \path [draw, ->] (he.south) -- node [above] {}
      (pr.90);
  \end{tikzpicture}

  \caption{\label{fig:mrstudyr}The inputs and outputs of the \mr~tool.}

  \captionpara{0.5}{0.9}{0.5}{In this figure, the dark square represents the \mr~tool and its constituent parts, a
    rectangle stands for a process, a rectangle with rounded edges is a calculation performed by \mr, and an ellipse
  symbolises a process output.}

  \vspace{-1.8em}
\end{figure}



\section{\textit{mrstudyr}: Mutant Reduction Studier}

Performing the entire mutation analysis process, displayed in Figure \ref{fig:process},
is both expensive in terms of time and computational requirements. The mutation testing
phase, displayed in Figure \ref{fig:process}, of the mutation analysis
process consists of generating and executing the mutants. The primary
expense of the mutation analysis process is incurred in the mutation testing phase
due to the large number of mutants generated---even for programs with few lines of code \cite{offutt2001mutation}.

The next phase in the mutation analysis process is retrospective analysis. In this
phase, mutants are found to either be dead or alive. Where a mutant is considered
``dead''---or to have been killed---if when executing the modified version of the source
code the output differed from the original version. Conversely, a mutant remains alive
if the output from the modified version of the source code does not differ from the
original version. More importantly, in this phase of the mutation analysis process,
we are able to evaluate the effectiveness of various reduction techniques.

By retrospectively analysing each reduction technique's effectiveness, we are
able to avoid the cost of performing mutation testing more than once. To alleviate
the additional executions, data needs to be collected
from the first run. Displayed in Figure \ref{fig:mrstudyr},
the \mr tool uses the data collected from mutation testing to conduct further analyses
regarding the effectiveness of reduction techniques.

The data, at the very least, needs to include the status of
a mutant after testing (e.g., dead or alive), the type of mutant (e.g., normal, duplicate, equivalent, or stillborn)
and the operator used. In addition to the minimum requirements, the data can include much
more detail about the mutants. An example of additional data that may be collected are the database
management system or schema under test and the cost of generating each mutant.

Talk about what is the data used for? \\
refer back to Figure \\
Talk about process shown in Figure

\subsection{Released as a Standardised R Package}
\textit{note: do not innapropriately include too much detail}
talk about why we chose to use R (hitchhiker's guide paper / SBST2016)\\
talk about the importance of free and open-source software (Regression testing workshop paper)\\
talk about why reproducibility is important (Regression testing workshop paper)


\subsection{Generalised Input Format}
not limited to a single domain / extensible
allows for use in old and new domains with little modification
    \subsubsection{application to databases}
    talk about mutation analysis of database schema mutants
    \subsubsection{application to programs}



The R programming language for statistical data analysis was specifically designed to
be a full statistical language for conducting empirical studies. Following its predecessor,
R is a powerful means for data analysts to express computations~\cite{ihaka1996r}.
In R, the fundamental way to share code is via a package.

% TODO: This paragraph will need Dr. Kapfhammer's magic to fix (I cannot stand the spacing)

R packages are easy to distribute and include code, data, documentation and tests~\cite{wickham2015r}.  Since \mr~has
been released as an R package, installing the tool requires four commands in the R console.  First,
\texttt{install.packages("devtools")}, then {\small\texttt{library(devtools)}} to install and load the
{\small\texttt{devtools}}~\cite{devtools} package, respectively.  The \texttt{devtools} package is necessary because it
provides functions for installing and maintaining \mr~as well as its dependencies.  Finally, to install \mr~\cite{tool}
from the popular Git repository hosting service, GitHub~\cite{github}, use the following command:
{\small\texttt{devtools::install\_github("mccurdyc/mrstudyr")}}.  Then, finally, load \mr~using
\texttt{library(mrstudyr)}.

%!TEX root=../icsme2016-mrstudyr.tex

\begin{figure}
\centering

\vspace{-.75em}

\scalebox{\algorithmfigurescalefactor}{
\begin{tabular}{|l|}
\hline

% TODO: Review this algorithm and fix all of the other issues pointed out in the marked version of the paper
% --> How do you initialise d?
% --> How is the variable p incorporated?
% --> Does this algorithm closely follow the structure of the R code?
% --> Does this algorithm use consistent terminology for the variable x?

\begin{tabular}{l}
\vspace{-2ex} \\

{\bf function} analyse ($data$) \\

\atab {\bf where} \\
\atab \atab \tinybullet $\mbox{{\em data}}$ is data collected from performing mutation testing on all mutants \\
\atab \atab \tinybullet $x$ is the maximum fractional threshold for the number of mutants \\
\atab \atab \tinybullet {\em percentages} is the list of chosen $x$ values for mutant sampling \\
\atab \atab \tinybullet $t$ is the trial index of the experiment \\
\atab \atab \tinybullet {\em trials} is the total number of trials set for each configuration of technique \\
\atab \atab \tinybullet {\em red} is the reduced mutant data returned from a single reduction method \\
\atab \atab \tinybullet $d$ is the data from the configuration of a reduction method and $x$ value\\

\vspace*{-.75em} \\

% \hspace{1em} \atab {\bf initialise} $d$ \smallvspace \\

\hspace{1em} \atab $d \leftarrow \emptyset$, $t \leftarrow 0$ \smallvspace \\

% \hspace{1em} \atab $t \leftarrow 0$ \smallvspace \\

\hspace{1em} \atab {\bf for each} $x \in \mbox{{\em percentages}}$ \smallvspace \\

\hspace{3em} \atab {\bf for} $t \leq \mbox{{\em trials}}$ \smallvspace \\

\hspace{4em} \atab {\bf let} $red \leftarrow \mbox{{\em reduction\,technique}}\,(\mbox{{\em data}},\,p)$ \smallvspace \\

\hspace{4em} \atab {\bf let} $d \leftarrow$ {\bf append}($d$, $\mbox{{\em red}}$) \smallvspace \\

\hspace{4em} \atab $t \leftarrow t + 1$ \smallvspace \\

\hspace{1em} \atab {\bf return} $d$ \smallvspace \\

{\bf end function} \\
\vspace{-0.5em} \\

\end{tabular} \\

\hline
\end{tabular}}

\caption{\label{fig:experiment-structure} An experiment for studying mutant sampling.}

\vspace{-1em}

\end{figure}


\subsection{Conducting Experiment Campaigns}

% The \mr~tool accepts a generalised input format enabling it to be used to retrospectively analyse mutation testing data
% from programs and also emerging domains such as databases.  Although domains under test may differ, like data can be
% collected (e.g., mutant status, mutant type and operator). While more data can optionally be collected from mutation
% testing, the additional data is more dependent on the domain under test. Data collected from performing mutation testing
% on the nine database schemas is displayed in Table~\ref{tbl:study-data}.  This is a snippet of the data used in the
% preliminary study of effectiveness of \mr.  The additional data are unique identifier for the database management system
% / schema combination (Identifier), DBMS (DBMS), schema (Schema) and mutant creation time (Time).

% The general rule in many fields of science is to minimally perform 30 observations to show with high confidence that the
% results are statistically significant and to allow for the results to approximate a normal distribution.

% The structure of the experiment function for the two reduction sub-techniques is displayed in

% Explain how to conduct a compaign of experiments using the figure with an algorithm's psuedo code
% TODO: Fix the fact that the last sentence in this paragraph does not take up enough space

% Experiment Structure Diagram: This diagram in its current state is pretty specific to our experiment and testing database
% schemas. As you will notice, it iterates over a list of schemas. Additionally, it is specific to mutant sampling
% with the diagram showing that a technique iterates over a list of percentages. We can keep the diagram the way it
% is or remove it. I think by making it generalised (removing list of schemas, and percentages) it reduces it to the
% point where it isn't worth showing because it will then consist of iterate through a number of trials, perform a reduction
% technique, return the data collected from every trial. If it were simplified, I think it would just be easier to
% describe the technique it writing rather than taking up so much space with a diagram.

% I know in GH we talked about objective-level discussion then increase granularity. With that in mind, this first
% sentence might be too specific for this section and may need to move to a section specific to the results displayed
% in this paper. Or, going back to the previous comment, we could remove the 'iterate over schemas' from the diagram
% and say that this is the structure for the campaign of experiments for mutant sampling with mrstudyr (I will do this!)

% CJM: This paragraph will need major revisions now with the updated experiment structure diagram
Figure~\ref{fig:experiment-structure} gives the structure for the campaign of experiments that \mr~conducts to collect
the reduced mutant data from performing various reduction techniques. The ``analyse'' algorithm expects the mutant data from performing
mutation testing on all mutants. Since the presented tool currently focuses on mutant
sampling, the \textit{reduction technique} algorithm requires as input an arbitrary $x$, to be chosen as the maximum fractional threshold for the
number of mutants to be analysed from a set. Following the recommendations of Traeger~\etal~and Arcuri and Briand,
for the preliminary results presented in this paper, the maximum threshold for the number of trials that \mr~runs for each configuration of a reduction approach is set to 30~\cite{traeger2008nine, arcuri2014hitchhiker}, thereby
controlling for the randomness inherent in a reduction method's behavior and execution time.

\subsection{Implementation as an R Package}

The R programming language for statistical data analysis was specifically designed to be a full statistical language for
conducting empirical studies. Following its predecessor, R is a powerful means for data analysts to express
computations~\cite{ihaka1996r}.  In R, the fundamental way to share code is via a package.  R packages are easy to
distribute and include code, data, documentation and tests~\cite{wickham2015r}.  Since \mr~has been released as an R
package, installing the tool requires four commands in the R console.  First, \texttt{install.packages("devtools")},
then {\small\texttt{library(devtools)}} to install and load the {\small\texttt{devtools}}~\cite{devtools} package,
respectively.  The \texttt{devtools} package is necessary because it provides functions for installing and maintaining
\mr~as well as its dependencies.  Finally, to install \mr~\cite{tool} from the popular Git repository hosting service,
GitHub~\cite{github}, use the following command: {\small\texttt{devtools::install\_github("mccurdyc/mrstudyr")}}.  Then,
finally, load \mr~using \texttt{library(mrstudyr)}.

%!TEX root=../icsme2016-mrstudyr.tex

\begin{figure}
\centering

\vspace{-.75em}

\scalebox{\algorithmfigurescalefactor}{
\begin{tabular}{|l|}
\hline

% TODO: Review this algorithm and fix all of the other issues pointed out in the marked version of the paper
% --> How do you initialise d?
% --> How is the variable p incorporated?
% --> Does this algorithm closely follow the structure of the R code?
% --> Does this algorithm use consistent terminology for the variable x?

\begin{tabular}{l}
\vspace{-2ex} \\

{\bf function} analyse ($data$) \\

\atab {\bf where} \\
\atab \atab \tinybullet $\mbox{{\em data}}$ is data collected from performing mutation testing on all mutants \\
\atab \atab \tinybullet $x$ is the maximum fractional threshold for the number of mutants \\
\atab \atab \tinybullet {\em percentages} is the list of chosen $x$ values for mutant sampling \\
\atab \atab \tinybullet $t$ is the trial index of the experiment \\
\atab \atab \tinybullet {\em trials} is the total number of trials set for each configuration of technique \\
\atab \atab \tinybullet {\em red} is the reduced mutant data returned from a single reduction method \\
\atab \atab \tinybullet $d$ is the data from the configuration of a reduction method and $x$ value\\

\vspace*{-.75em} \\

% \hspace{1em} \atab {\bf initialise} $d$ \smallvspace \\

\hspace{1em} \atab $d \leftarrow \emptyset$, $t \leftarrow 0$ \smallvspace \\

% \hspace{1em} \atab $t \leftarrow 0$ \smallvspace \\

\hspace{1em} \atab {\bf for each} $x \in \mbox{{\em percentages}}$ \smallvspace \\

\hspace{3em} \atab {\bf for} $t \leq \mbox{{\em trials}}$ \smallvspace \\

\hspace{4em} \atab {\bf let} $red \leftarrow \mbox{{\em reduction\,technique}}\,(\mbox{{\em data}},\,p)$ \smallvspace \\

\hspace{4em} \atab {\bf let} $d \leftarrow$ {\bf append}($d$, $\mbox{{\em red}}$) \smallvspace \\

\hspace{4em} \atab $t \leftarrow t + 1$ \smallvspace \\

\hspace{1em} \atab {\bf return} $d$ \smallvspace \\

{\bf end function} \\
\vspace{-0.5em} \\

\end{tabular} \\

\hline
\end{tabular}}

\caption{\label{fig:experiment-structure} An experiment for studying mutant sampling.}

\vspace{-1em}

\end{figure}




\subsection{Tool Usage}

% CJM: Moving this figure here to allow reader to follow the process of the mrstudr tool

%!TEX root=../icsme2016-mrstudyr.tex

\newcommand{\mx}[1]{\mathbf{\bm{#1}}} % Matrix command
\newcommand{\vc}[1]{\mathbf{\bm{#1}}} % Vector command

% Define the layers to draw the diagram
\pgfdeclarelayer{background}
\pgfdeclarelayer{foreground}
\pgfsetlayers{background,main,foreground}

% Define block styles used later

\tikzstyle{sensor}=[draw, fill=black!5, text width=15em,
text centered, minimum height=2.5em,drop shadow]
\tikzstyle{smallsensor}=[draw, fill=black!5, text width=4em,
text centered, minimum height=2.5em,drop shadow]
\tikzstyle{box}=[draw, text width=8em,
text centered, minimum width=17.5em, minimum height=3.5em]
\tikzstyle{calc}=[draw, fill=black!5, text width=2.5em,
text centered, rounded corners, minimum height=2.5em,drop shadow]
\tikzstyle{circle}=[draw, ellipse, fill=black!5, text width=11em,
text centered, minimum height=2.5em, drop shadow]
\tikzstyle{mr}=[draw, fill=black!20, text width=5em,
text centered, minimum height=17em, minimum width = 18em, drop shadow]

% Define distances for bordering
\def\blockdist{1.5}
\def\edgedist{2.5}

\begin{figure}[t]

  \vspace{-.75em}

  \centering
  % SIMPLE
  \begin{tikzpicture}[thick,scale=0.85, every node/.style={scale=0.85}]
    \node [mr] at (0, 0) (mr) {};
    \node [circle] at (0, 3.7) (md) {Original Data};
    \node [sensor] at (0, 2.25) (r) {Reduction Techniques};
    \node [circle] at (0, 0.75) (rmd) {Cumulated Reduced Data};
    \node [sensor] at (0, -0.75) (ra) {Efficiency \& Effectiveness Analysis};
    \node [box] at (0, -2.25) (box) {};
    \node [calc] at (-2.25, -2.25) (ms) {MS};
    \node [calc] at (-0.75, -2.25) (err) {Red.};
    \node [calc] at (0.75, -2.25) (corr) {Corr.};
    \node [calc] at (2.25, -2.25) (err) {Err.};
    \node [sensor] at (0, -3.75) (he) {Human Examination};
    \node [circle] at (0, -5.25) (pr) {Policy Recommendation};

    \path [draw, ->] (md.south) -- node [above] {}
      (r.90);
    \path [draw, ->] (r.south) -- node [above] {}
      (rmd.90);
    \path [draw, ->] (rmd.south) -- node [above] {}
      (ra.90);
    \path [draw, ->] (ra.south) --  node [above] {}
      (box.90);
    \path [draw, ->] (box.south) --  node [above] {}
      (he.90);
    \path [draw, ->] (he.south) -- node [above] {}
      (pr.90);
  \end{tikzpicture}

  \caption{\label{fig:mrstudyr}The inputs and outputs of the \mr~tool.}

  \captionpara{0.5}{0.9}{0.5}{In this figure, the dark square represents the \mr~tool and its constituent parts, a
    rectangle stands for a process, a rectangle with rounded edges is a calculation performed by \mr, and an ellipse
  symbolises a process output.}

  \vspace{-1.8em}
\end{figure}



% CJM: I think we should move 'data' to the same line as the function call

Releasing an R tool via a package makes installing and loading the tool a matter of a few commands,
but does not guarantee tool usability. \mr~was designed to be simplistic, yet perform stringent
empirical analyses on mutant reduction techniques. To display the results from \mr~analysing mutant
reduction strategies for the testing of real-world database schemas, the following commands will be
tailored toward the data that we collected from performing mutation testing. The collected data can
be read in via the following: {\small\texttt{data <- read\_data("sqlite-avmdefaults.dat")}}. This
function expects the data to be located in the \texttt{inst/extdata} folder and stored as a
comma-separated value file.

% CJM: I noticed in section 2.C (Conducting Experiment Campaigns) that it already explains that mrstudyr
% focuses on mutant sampling. Therefore, I am going to remove the sentences in this section restating
% this fact. These sentences are commented-out immediately follow this comment.

% Currently, the two most common sub-techniques of mutant sampling are performed using the following:
% \texttt{analyse\_random\_sampling(data)} and \texttt{analyse\_across\_operators(data)},
% for random sampling and sampling across operators, respectively.

% CJM: Additionally, the attributes of the data that I describe are specific to mutant sampling. I will
% remove these attributes.

% CJM: Is including the following statement entirely necessary? I removed it for now.

% Sample output from the \texttt{analyse} function is provided in the accompanying README file on the
% tool's GitHub page~\cite{tool}.

The reduction techniques, reffered to in Figure~\ref{fig:mrstudyr}, are performed following the provision
of mutant data collected from mutation testing --- the ``Original Data'' --- to \mr. Using \mr~to perform
reduction techniques is a single command: \texttt{analyse(data)}. Each reduction technique performed by
the \texttt{analyse} function in \mr, returns reduced data which is then cumulated into a data set, shown
in Figure~\ref{fig:mrstudyr}, containing the data from all of the reduction approaches. This data includes
the trial, the total number of mutants analysed, the number of killed mutants, the generation time of each
mutant and the mutation scores for the reduced and original data. After performing the reduction techniques,
their efficiency and effectiveness is evaluated based on four calculations on the cumulated reduced data in
the ``E \& E Analysis'' phase.

% CJM: I want to present the evaluation metrics of efficiency and effectiveness, then define them, then finally,
% how to perform them with mrstudyr.
The evaluation calculations in E \& E analysis --- and their respective abreviations --- are: mutation score
(MS), reduction in creation cost (Red.), correlation (Corr.), and error (Err.). The mutation score is calculated
by dividing the number of killed mutants by the total number of mutants~\cite{wong1995reducing}. The reduction
in creation cost for a set of mutants is the cost of the reduced set subtracted from the cost of the original set,
divided by the cost of the original set. The original and reduced sets' mutation score correlation is calculated
using Kendall's~\taub, supported by the ``Kendall'' R package~\cite{mcleod2015kendall}. Finally, the error --- both
MAE and RMSE --- between the original and reduced sets' mutation score is calculated using the existing ``Metrics''
R package~\cite{metrics}.

The aforementioned metrics, with the exception of calculating correlation, are calculated by a single function
in \mr~: \texttt{analyse\_calculations(data)}. Where the input to \texttt{analyse\_calculations} is the cumulated
reduced data, displayed in Figure~\ref{fig:mrstudyr}, from performing the reduction techniques and returns a data
set with the values of the respective calculation. To calculate correlation, the \texttt{analyse\_correlation(data)}
function is used. Where the input data to this function is again the cumulated reduced data and the output is a
data set containing only the correlation values between the reduced and original mutation scores. The usability of
\mr~is supported by requiring experimenters to only call two functions to analyse the efficiency and effective of
reduction techniques.

% CJM: removed this paragraph because I no longer think that it is needed and to save space for preliminary study
% Additionally, the \texttt{analyse\_calculations} function can be used to calculate the
% efficiency and effectiveness for any number of reduction approaches; it is not structurally
% limited to a single reduction method. As input, the \texttt{analyse\_calculations} function
% expects the dataframe returned from performing a reduction approach.

The ``Human Examination'' phase, displayed in Figure~\ref{fig:mrstudyr}, still requires use of \mr~for clearly
visualising the trends in the data collected from the E \& E analysis. The \mr~tool takes advantage of the work
of Hadley Wickham with widely-used graphing package \texttt{ggplot2}~\cite{ggplot2}. All of the visualisation
functions provided by \mr~and their output can be found on the GitHub page~\cite{tool}. These visualisations
will help human examiners to construct a policy recommendation as to which reduction technique should be used
for the specific domain and data under observation.


\section{Related Work}
They ...
I ...
They ...
I ...


\section{Conclusions and Future Work}

% Re-introduce the challenges of mutation testing and summarize what this tool provides

Although mutation testing is well-recognized as a way to assess test suite quality, it may be too costly to practically
use. As such, various methods have been developed to decrease the cost of mutation testing. Performing these reduction
techniques in the past has required researchers and experimenters to incorporate a reduction method into an, often
complex, mutation testing tool. \mr~alleviates the burden of implementing each approach by retrospectively analysing
reduction techniques, a potentially more cost-effective method.

% TODO: It would be ideal if this paragraph could point out the experimental results in this paper

% Summarize some of the key features of the tool and then talk about how it is released and enables future work

By retrospectively analysing the data collected from prior analysis of all mutants, the \mr~tool is able to reduce the
upfront human-implementation costs and obviate the need for researchers and industrialists to understand the domain
complexities associated with understanding a mutation testing system. Furthermore, \mr~provides an
easy-to-use and rapid way to assess the efficiency and effectiveness of mutant reduction methods. In addition to being
detailed in this paper, \mr~has been released under an open-source license on a GitHub site that features extensive
documentation and a screencast~\cite{tool}. In future work, we plan to extend the functionality of \mr~by integrating
additional mutant reduction techniques, thereby allowing for a more comprehensive comparison of the techniques'
efficiency and effectiveness.



% An example of a floating figure using the graphicx package.
% Note that \label must occur AFTER (or within) \caption.
% For figures, \caption should occur after the \includegraphics.
% Note that IEEEtran v1.7 and later has special internal code that
% is designed to preserve the operation of \label within \caption
% even when the captionsoff option is in effect. However, because
% of issues like this, it may be the safest practice to put all your
% \label just after \caption rather than within \caption{}.
%
% Reminder: the "draftcls" or "draftclsnofoot", not "draft", class
% option should be used if it is desired that the figures are to be
% displayed while in draft mode.
%
%\begin{figure}[!t]
%\centering
%\includegraphics[width=2.5in]{myfigure}
% where an .eps filename suffix will be assumed under latex, 
% and a .pdf suffix will be assumed for pdflatex; or what has been declared
% via \DeclareGraphicsExtensions.
%\caption{Simulation results for the network.}
%\label{fig_sim}
%\end{figure}

% Note that the IEEE typically puts floats only at the top, even when this
% results in a large percentage of a column being occupied by floats.


% An example of a double column floating figure using two subfigures.
% (The subfig.sty package must be loaded for this to work.)
% The subfigure \label commands are set within each subfloat command,
% and the \label for the overall figure must come after \caption.
% \hfil is used as a separator to get equal spacing.
% Watch out that the combined width of all the subfigures on a 
% line do not exceed the text width or a line break will occur.
%
%\begin{figure*}[!t]
%\centering
%\subfloat[Case I]{\includegraphics[width=2.5in]{box}%
%\label{fig_first_case}}
%\hfil
%\subfloat[Case II]{\includegraphics[width=2.5in]{box}%
%\label{fig_second_case}}
%\caption{Simulation results for the network.}
%\label{fig_sim}
%\end{figure*}
%
% Note that often IEEE papers with subfigures do not employ subfigure
% captions (using the optional argument to \subfloat[]), but instead will
% reference/describe all of them (a), (b), etc., within the main caption.
% Be aware that for subfig.sty to generate the (a), (b), etc., subfigure
% labels, the optional argument to \subfloat must be present. If a
% subcaption is not desired, just leave its contents blank,
% e.g., \subfloat[].


% An example of a floating table. Note that, for IEEE style tables, the
% \caption command should come BEFORE the table and, given that table
% captions serve much like titles, are usually capitalized except for words
% such as a, an, and, as, at, but, by, for, in, nor, of, on, or, the, to
% and up, which are usually not capitalized unless they are the first or
% last word of the caption. Table text will default to \footnotesize as
% the IEEE normally uses this smaller font for tables.
% The \label must come after \caption as always.
%
%\begin{table}[!t]
%% increase table row spacing, adjust to taste
%\renewcommand{\arraystretch}{1.3}
% if using array.sty, it might be a good idea to tweak the value of
% \extrarowheight as needed to properly center the text within the cells
%\caption{An Example of a Table}
%\label{table_example}
%\centering
%% Some packages, such as MDW tools, offer better commands for making tables
%% than the plain LaTeX2e tabular which is used here.
%\begin{tabular}{|c||c|}
%\hline
%One & Two\\
%\hline
%Three & Four\\
%\hline
%\end{tabular}
%\end{table}


% Note that the IEEE does not put floats in the very first column
% - or typically anywhere on the first page for that matter. Also,
% in-text middle ("here") positioning is typically not used, but it
% is allowed and encouraged for Computer Society conferences (but
% not Computer Society journals). Most IEEE journals/conferences use
% top floats exclusively. 
% Note that, LaTeX2e, unlike IEEE journals/conferences, places
% footnotes above bottom floats. This can be corrected via the
% \fnbelowfloat command of the stfloats package.


% conference papers do not normally have an appendix

% use section* for acknowledgment
% trigger a \newpage just before the given reference
% number - used to balance the columns on the last page
% adjust value as needed - may need to be readjusted if
% the document is modified later
%\IEEEtriggeratref{8}
% The "triggered" command can be changed if desired:
%\IEEEtriggercmd{\enlargethispage{-5in}}

% references section

% can use a bibliography generated by BibTeX as a .bbl file
% BibTeX documentation can be easily obtained at:
% http://mirror.ctan.org/biblio/bibtex/contrib/doc/
% The IEEEtran BibTeX style support page is at:
% http://www.michaelshell.org/tex/ieeetran/bibtex/
\bibliographystyle{IEEEtran}
% argument is your BibTeX string definitions and bibliography database(s)
\bibliography{IEEEabrv,bibliography}
%
% <OR> manually copy in the resultant .bbl file
% set second argument of \begin to the number of references
% (used to reserve space for the reference number labels box)
\end{document}


